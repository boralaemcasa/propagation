\documentclass[12pt]{article}
\usepackage{amsmath}
\usepackage{amssymb} %mathbb
\usepackage{graphicx}
\usepackage{hyperref}
\usepackage[latin1]{inputenc}

\begin{document}

\Large

\begin{center}
Sistemas Nebulosos
\end{center}

\normalsize

\vspace{3mm}

Entrada: $u(t)$

Sa\'ida: $i(t)$

$u = Ri + L\, \cfrac{di}{dt}$

Varia\c{c}\~ao de Par\^ametros: $y' + p(t) y = q(t) \Rightarrow y = \exp(-\int p) \left[ \int \exp (\int p) q \,\mathrm{d}t + C \right]$

$y \leftarrow i$

$p(t) \leftarrow R/L \Rightarrow \int p = Rt/L$

$q(t) \rightarrow u/L$

Portanto,

\begin{align}
 i(u(t)) &= \exp(-Rt/L) \left[ \int (\exp Rt/L) u/L \,\mathrm{d}t + C \right] \\
 i(u(t)) &= \cfrac{1}{L \alpha^t} \left[ \int \alpha^t u \,\mathrm{d}t + C \right]
\end{align}

Outras restri\c{c}\~oes:

$u \leftarrow u[k]$

$u[k] \leftarrow \mu_1 q_1[k] + \mu_2 q_2[k]$

$q_1[k+1] = q_1[k] - \beta e \mu_1$

$q_2[k+1] = q_2[k] - \beta e \mu_2$

\vspace{3mm}

\textbf{(II)}

\textbf{Se beta, mu\_i variarem com k, podemos pegar os pontos}

\textbf{$(q_1, q_2)[k] = P_k \in \mathbb{R}^2$ e interpolar. No caso de polin\^omios: $P_0, P_1$ fazem reta de primeiro grau, $P_0, \cdots, P_n$ tem grau n.}

\vspace{3mm}

\textbf{(III)}

\textbf{Para todo u, existe i(u). Queremos uma curva que comece da origem e estabilize em y = constante no menor $t = t_0$ poss\'ivel. Vou dar exemplos de degraus suaves:}

\begin{align}
  2 \le t \le 3 \Rightarrow f_1(t) &= \cfrac{1}{1 - t} \Rightarrow -1 < y < -0.5 \\
  2 \le t \le 10 \Rightarrow f_2(t) &= \ln(t + 2) \Rightarrow 1.4 < y < 2.4 \\
  0 \le t \le 2 \Rightarrow f_3(t) &= 1 - \exp(-t) \Rightarrow 0 < y < 0.8
\end{align}

\textbf{Agora eu exijo que todas as derivadas \`a esquerda e \`a direita sejam zero, como no degrau-rampa-degrau.}

\begin{align}
  0 \le t \le 2 \Rightarrow f_4(t) &= \exp(-1/t) \Rightarrow 0 < y < 0.6 \\
  a \le t \le a + 1 \Rightarrow f_5(t) &= \exp(-1/(t - a)) \Rightarrow 0 < y < 0.36 \\
  a \le t \le b \Rightarrow f_6(t) &= \int_a^t \exp\left(\cfrac{1}{(x - a) (x - b)}\right)\,\mathrm{d}x \Rightarrow 0 < y < f_6(b)
\end{align}

\textbf{Esta \'ultima \'e crescente. Calcule o limite quando b tende a $a$.}

\begin{align}
  a \le t \le b \Rightarrow g(t) &= \cfrac{f_6(t)}{f_6(b)} (M - g_0) + g_0 \Rightarrow g_0 < y < M \\
  a \le t \le b \Rightarrow f_7(t) &= \int_t^b \exp\left(\cfrac{1}{(x - a) (x - b)}\right)\,\mathrm{d}x \Rightarrow 0 < f_7(a) \to y \to 0 \\
  a \le t \le b \Rightarrow h(t) &= \cfrac{f_7(t)}{f_7(a)} (M - h_0) + h_0 \Rightarrow h_0 < M \to y \to h_0
\end{align}

\vspace{3mm}

Out of charity, there is no salvation at all. With charity, there is evolution.

\vspace{3mm}

Vinicius Claudino FERRAZ, 5/Sep/2019, Release $1.0$

\end{document}

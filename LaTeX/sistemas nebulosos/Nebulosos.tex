\documentclass[12pt]{article}
\usepackage{amsmath}
\usepackage{amssymb} %mathbb
\usepackage{graphicx}
\usepackage{hyperref}
\usepackage{colortbl}
\usepackage[latin1]{inputenc}

\begin{document}

\Large

\begin{center}
\href{https://www.youtube.com/watch?v=2E0TYW42x1c}{\color{blue}\underline{Sistemas Nebulosos 1}}
\end{center}

\normalsize

\vspace{3mm}

Entrada: $u(t)$

Sa\'ida: $i(t)$

$u = Ri + L\, \cfrac{di}{dt}$

Varia\c{c}\~ao de Par\^ametros: $y' + p(t) y = q(t) \Rightarrow y = \exp(-\int p) \left[ \int \exp (\int p) q \,\mathrm{d}t + C \right]$

$y \leftarrow i$

$p(t) \leftarrow R/L \Rightarrow \int p = Rt/L$

$q(t) \rightarrow u/L$

Portanto,

\begin{align}
 i(u(t)) &= \exp(-Rt/L) \left[ \int (\exp Rt/L) u/L \,\mathrm{d}t + C \right] \\
 i(u(t)) &= \cfrac{1}{L \alpha^t} \left[ \int \alpha^t u \,\mathrm{d}t + C \right]
\end{align}

Outras restri\c{c}\~oes:

$u \leftarrow u[k]$

$u[k] \leftarrow \mu_1 q_1[k] + \mu_2 q_2[k]$

$q_1[k+1] = q_1[k] - \beta e \mu_1$

$q_2[k+1] = q_2[k] - \beta e \mu_2$

\vspace{3mm}

\textbf{(II)}

\textbf{Se beta, mu\_i variarem com k, podemos pegar os pontos}

\textbf{$(q_1, q_2)[k] = P_k \in \mathbb{R}^2$ e interpolar. No caso de polin\^omios: $P_0, P_1$ fazem reta de primeiro grau, $P_0, \cdots, P_n$ tem grau n.}

\vspace{3mm}

\textbf{(III)}

\textbf{Para todo u, existe i(u). Queremos uma curva que comece da origem e estabilize em y = constante no menor $t = t_0$ poss\'ivel. Vou dar exemplos de degraus suaves:}

\vspace{3mm}

Out of charity, there is no salvation at all. With charity, there is evolution.

\vspace{3mm}

Vinicius Claudino FERRAZ, 5/Sep/2019, Release $1.0$

\end{document}

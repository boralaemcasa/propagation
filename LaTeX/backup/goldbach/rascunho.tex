\documentclass[12pt]{article}
\usepackage{amsmath}
\usepackage{amssymb} %mathbb
\usepackage{graphicx}
\usepackage{hyperref}
\usepackage[latin1]{inputenc}
\usepackage[top=1.0cm,bottom=1.3cm,left=1.0cm,right=1.0cm]{geometry}

\begin{document}

Every even number is a sum of two prime numbers. That's Goldbach's unproven theory.

\begin{align}
  \forall n \in \mathbb{N},\, n \ge 2 \Rightarrow \exists (p, q) \in \mathbb{P}^2\,;\,2n &= p + q
\end{align}

\textbf{Demo:} $\blacktriangleright$ What about the odd numbers? They are the sum of a prime number with an even number, because there is a prime below.

\begin{align}
  2n + 1 = p + 2j,\,\, p < 2n,\,\,\exists j \in \mathbb{N}
\end{align}

For instance, for $p = 3$, that is obvious. Similarly, there is a prime above. 

Hence, every odd number is the difference of a prime number with an even number.

\begin{align}
  2n + 1 = q - 2k,\,\, q > 2n,\,\,\exists k \in \mathbb{N}
\end{align}

Joining both, the conjecture has become too obvious.

\begin{align}
  2n + 1 &= p + 2j \\
  2n + 1 &= q - 2k \\
  4n + 2 &= p + q + 2j - 2k \\
  2\ell &= p + q\,\,\blacksquare
\end{align}

Example 1: 

\begin{align}
  2\ell &= 88\,;\,k = 0 \\
  2n + 1 &= p + 2j \\
  2n + 1 &= q \\
  4n + 2 &= 88 + 2j \\
  2n + 1 &= 44 + j \in \mathbb{P} \Rightarrow j = 3 \wedge q = 47 \wedge p = 41  \\
  2n + 1 &= 47 \Rightarrow n = 23
\end{align}

Example 2: 

\begin{align}
  2\ell &= 90\,;\,k = 0 \\
  2n + 1 &= p + 2j \\
  2n + 1 &= q \\
  4n + 2 &= 90 + 2j \\
  2n + 1 &= 45 + j \in \mathbb{P} \Rightarrow j = 2 \wedge q = 47 \wedge p = 43 \\
  2n + 1 &= 47 \Rightarrow n = 23 
\end{align}

Example 3: 

\begin{align}
  2\ell &= 92\,;\,k = 0 \\
  2n + 1 &= p + 2j \\
  2n + 1 &= q \\
  4n + 2 &= 92 + 2j \\
  2n + 1 &= 46 + j \in \mathbb{P} \Rightarrow j = 15 \wedge q = 61 \wedge p = 31 \\
  2n + 1 &= 61 \Rightarrow n = 30 
\end{align}

Let's go by indution.

\begin{align}
\varphi(4): 8 &= 3 + 5 = p(4) + q(4) \\
5 &= p(4) \Rightarrow j(4) = 0 \\
5 &= q(4) - 2k(4) \Rightarrow k(4) = 2 \\
\varphi(5): 10 &= 5 + 5 = p(5) + q(5) \\
5 &= p(5) \Rightarrow j(5) = 0 \\
5 &= q(5) \Rightarrow k(5) = 0 \\
\varphi(2n): 4n &= p(2n) + q(2n) \\
2n + 1 &= p(2n) + 2j(2n) \\
2n + 1 &= q(2n) - 2k(2n) \\
\varphi(2n + 1): 4n + 2 &= p(2n + 1) + q(2n + 1) \\
2n + 1 &= p(2n + 1) + 2j(2n + 1) \\
2n + 1 &= q(2n + 1) - 2k(2n + 1)
\end{align}

We want to show that

\begin{align}
\varphi(2n + 2): 4n + 4 &= p(2n + 2) + q(2n + 2) \\
2n + 3 &= p(2n + 2) + 2j(2n + 2) \\
2n + 3 &= q(2n + 2) - 2k(2n + 2)
\end{align}


\vspace{6mm}

Out of charity, there is no salvation at all. With charity, there is Evolution.

Vinicius Claudino Ferraz. 2020/July/31th

\end{document}
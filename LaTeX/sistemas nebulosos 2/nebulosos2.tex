\documentclass[12pt]{article}
\usepackage{amsmath}
\usepackage{amssymb} %mathbb
\usepackage{graphicx}
\usepackage{hyperref}
\usepackage{colortbl}
\usepackage[latin1]{inputenc}
\usepackage[top=1.0cm,bottom=1.3cm,left=1.0cm,right=1.0cm]{geometry}

\begin{document}

\Large

\begin{center}
\href{https://www.youtube.com/watch?v=2E0TYW42x1c}{\color{blue}\underline{Sistemas Nebulosos 2}}
\end{center}

\normalsize

\vspace{3mm}

\section{Degraus Suaves [de]Crescentes}

\begin{flushleft}
Entrada: $u(t)$ \\
Sa\'ida: $i(t)$
\end{flushleft}

$u = Ri + L\, \cfrac{di}{dt}$

Varia\c{c}\~ao de Par\^ametros: $y' + p(t) y = q(t) \Rightarrow y = \exp(-\int p) \left[ \int \exp (\int p) q \,\mathrm{d}t + C \right]$

$y \leftarrow i$

$p(t) \leftarrow R/L \Rightarrow \int p = Rt/L$

$q(t) \rightarrow u/L$

Portanto,

\begin{align}
 i(u(t)) &= \exp(-Rt/L) \left[ \int (\exp Rt/L) u/L \,\mathrm{d}t + C \right] \\
 i(u(t)) &= \cfrac{1}{L \alpha^t} \left[ \int \alpha^t u \,\mathrm{d}t + C \right]
\end{align}

Outras restri\c{c}\~oes:

$u \leftarrow u[k]$

$u[k] \leftarrow \mu_1 q_1[k] + \mu_2 q_2[k]$

$q_1[k+1] = q_1[k] - \beta e \mu_1$

$q_2[k+1] = q_2[k] - \beta e \mu_2$

\vspace{3mm}

\textbf{(II)}

Se beta, mu\_i variarem com k, podemos pegar os pontos $(q_1, q_2)[k] = P_k \in \mathbb{R}^2$ e interpolar. No caso de polin\^omios: $P_0, P_1$ fazem reta de primeiro grau, $P_0, \cdots, P_n$ tem grau n.

\vspace{3mm}

\textbf{(III)}

Para todo u, existe i(u). Queremos uma curva que comece da origem e estabilize em y = constante no menor $t = t_0$ poss\'ivel. Vou dar exemplos de degraus suaves:

\begin{align}
  2 \le t \le 3 \Rightarrow f_1(t) &= \cfrac{1}{1 - t} \Rightarrow -1 < y < -0.5 \\
  2 \le t \le 10 \Rightarrow f_2(t) &= \ln(t + 2) \Rightarrow 1.4 < y < 2.4 \\
  0 \le t \le 2 \Rightarrow f_3(t) &= 1 - \exp(-t) \Rightarrow 0 < y < 0.8
\end{align}

Agora eu exijo que todas as derivadas \`a esquerda e \`a direita sejam zero, como no degrau-rampa-degrau.

\begin{align}
  0 \le t \le 2 \Rightarrow f_4(t) &= \exp(-1/t) \Rightarrow 0 < y < 0.6 \\
  a \le t \le a + 1 \Rightarrow f_5(t) &= \exp(-1/(t - a)) \Rightarrow 0 < y < 0.36 \\
  a \le t \le b \Rightarrow f_6(t) &= \int_a^t \exp\left(\cfrac{1}{(x - a) (x - b)}\right)\,\mathrm{d}x \Rightarrow 0 < y < f_6(b)
\end{align}

Esta \'ultima \'e crescente. Calcule o limite quando b tende a $a$.

\begin{align}
  a \le t \le b \Rightarrow g(t) &= \cfrac{f_6(t)}{f_6(b)} (M - g_0) + g_0 \Rightarrow g_0 < y < M \\
  a \le t \le b \Rightarrow f_7(t) &= \int_t^b \exp\left(\cfrac{1}{(x - a) (x - b)}\right)\,\mathrm{d}x \Rightarrow 0 < f_7(a) \to y \to 0 \\
  a \le t \le b \Rightarrow h(t) &= \cfrac{f_7(t)}{f_7(a)} (M - h_0) + h_0 \Rightarrow h_0 < M \to y \to h_0
\end{align}

Conclus\~ao: o valor ideal que queremos \'e dado ou por $g(t)$ ou por $h(t)$.

\section{Controlador RL de primeira ordem}

\begin{align}
  h &= 0,001\,;\,0 \le k \le 23000 h\,;\,a = 1 \le x \le 24 = 8b
\end{align}

Pela soma de Riemann,

\begin{align}
  i(u)(k) &= e^{-R/L (a + hk)}\cfrac{1}{L}\sum_{n > 0}^k e^{R/L (a + hn)} u(n) h \\  \mu &:= \mu_1 \Rightarrow \mu_2 = 1 - \mu \\
  i_{ref}(\mathbb{Z}) &= \text{degrau restrito} \\
  e &= i - i_{ref} \\
  e(k) &\le - E_1 \Rightarrow \mu(k) = 1 \\
  e(k) &\ge E_2 \Rightarrow \mu(k) = 0 \\
  - E_1 &\le e(k) \le E_2 \Rightarrow \mu(k) = p e(k) + q \\
  u_{ref} &= R i_{ref} + L \cfrac{d}{dt} i_{ref} \\
  v &= u - u_{ref}
\end{align}

Para cada alfa, conseguimos $| e | < \delta \Rightarrow | v | < \epsilon$. [Provar.]

A primeira quest\~ao \'e se podemos inverter. Para \'epsilon = 0,08, ser\'a que o valor de alfa = 1 vai for\c{c}ar que o delta de entrada seja 2,5 e portanto garantir que o m\'odulo do erro de sa\'ida seja sempre menor que \'epsilon???

\begin{align}
u(k+1) &= \mu_k q_1(k+1) + (1 - \mu_k) q_2(k+1) \\
&= \mu_k \sum_n \alpha \mu_n e + (1 - \mu_k) \sum_n \alpha (1 - \mu_n) e \\
u(i)(k+1) &= \sum_{n = 0}^k \alpha \varphi(n) [i(n) - i_{ref}(n)] \\
\varphi(n) &= 2 \mu_n \mu_k + 1 - \mu_k - \mu_n
\end{align}

A equa\c{c}~ao 13 passa de u para i. Combinamos com a equa\c{c}\~ao 24, que passa de i para u.

\begin{align}
i &\to i(k+ 1) = e^{-R/L (a + hk + h)}\cfrac{1}{L}\sum_{m > 0}^{k + 1} e^{R/L (a + hm)}  \sum_{n = 0}^{m - 1} \alpha \varphi(n) [i(n) - i_{ref}(n)] h \\
u &\to u(k + 1) = \sum_{m = 0}^k \alpha \varphi(m) \bigg[e^{-R/L (a + hm)}\cfrac{1}{L}\sum_{n > 0}^m e^{R/L (a + hn)} u(n) h - i_{ref}(m)\bigg]
\end{align}

Vamos dividir em 3 casos.

\subsection{Caso 1}

\begin{align}
  e(k) &\le - E_1 \Rightarrow \mu(k) = 1
\end{align}

\subsection{Caso 2}

\begin{align}
   e(k) &\ge E_2 \Rightarrow \mu(k) = 0
\end{align}

\subsection{Caso 3}

\begin{align}
  - E_1 &\le e(k) \le E_2 \Rightarrow \mu(k) = p e(k) + q
\end{align}

\vspace{3mm}

Out of charity, there is no salvation at all. With charity, there is evolution.

\vspace{3mm}

Vinicius Claudino FERRAZ, 15/Sep/2019, Release $1.0.1$

\end{document}

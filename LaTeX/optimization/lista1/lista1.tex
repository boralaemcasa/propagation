\documentclass{rbfin}
\usepackage{amsmath}
\usepackage{amssymb} %mathbb
\usepackage{gensymb} % \degree
\usepackage{graphicx}
\usepackage{hyperref}

\begin{document}
\selectlanguage{brazil}
\shorttitle{Otimização Não Linear 2021} % appears on header every other page
\rbfe{}
\autor{Vinícius Claudino Ferraz, 2021}


\begin{center}
\Large

\textbf{Lista 1}

\normalsize

Matrícula $= 2019435823$
\end{center}

\large

\textbf{Questão 1}

\normalsize

\vspace{6mm}

\textbf{a)} Problemas de otimização também são conhecidos como problemas de programação matemática.

FALSO. Existem problemas em teoria da decisão que não são convertíveis em números.

\vspace{3mm}

\textbf{b)} O número de restrições de igualdade pode ser maior que o número de variáveis de otimização.

VERDADEIRO. Exemplo: numa matriz, o número de linhas pode ser maior que o número de colunas.

\vspace{3mm}

\textbf{c)} Parâmetros de projeto são os valores fixos das informações de projeto em um problema de otimização.

VERDADEIRO. São predefinidos.

\vspace{3mm}

\textbf{d)} Os limites das variáveis de otimização não são relacionados com o funcionamento do sistema que se está otimizando.

FALSO. É o espaço de busca.

\vspace{3mm}

\textbf{e)} Um ponto na fronteira de uma restrição pode ser inviável.

VERDADEIRO. Rao exibe em sua figura os pontos de limite inaceitáveis. É interseção de 2 restrições, mas não é interseção de 3.

\vspace{3mm}

\textbf{f)} É necessário que algum $g_i(x) = 0$ no ponto ótimo.

FALSO. $g_i(x) \le 0$.

\vspace{6mm}

\large

\textbf{Questão 2} 

\normalsize

\vspace{6mm}

Qual estratégia utilizar para resolver um problema de maximização como um problema de minimização?

\vspace{3mm}

Podemos simplesmente multiplicar a função objetivo por $-1$ e resolver como um problema de minimização.

$\max f (x) = \min - f (x)$.

\newpage

\large

\textbf{Questão 3}

\normalsize

\vspace{6mm}

Qual a diferença entre um problema de programação linear e um problema de programação não linear?

\vspace{3mm}

O problema linear pode ser escrito como:

$x^* = \arg \min c^\top x$

com restrições de desigualdade: $A_1 x \le b_1$

e restrições de igualdade: $A_2 x = b_2$.

Já o não linear não é passível de ser reduzido a tais vetores $c, b_1, b_2 \in \mathbb{R}^n$ (para algum $n$) e tais matrizes $A_1, A_2$.

\vspace{6mm}

\large

\textbf{Questão 4}

\normalsize

\vspace{6mm}

Qual a diferença entre as variáveis de otimização e os parâmetros de projeto?

\vspace{3mm}

Variáveis de otimização: são as entidades que identificarão
uma configuração particular, e esses valores serão os que mudarão
ao longo da busca da melhor configuração. Descrevem as especificações da configuração, como valores de componentes,
tamanho de partes, ou quantidade de cada peça.

Parâmetros de projeto ou predefinidos: são os valores que não irão
mudar (fixos) quando compararmos diferentes configurações durante
o procedimento de busca pela melhor solução. São definidos no início
do problema.

\vspace{6mm}

\large

\textbf{Questão 5}

\normalsize

\vspace{6mm}

O que é o espaço de projeto?

\vspace{3mm}

É a região do espaço Euclidiano gerado pelas $n$ variáveis de otimização $x$.

Por exemplo, um produto cartesiano de intervalos.

Seja $x_i \in J_i = [a_i, b_i] \subset \mathbb{R}$, para cada $1 \le i \le n$.

\newpage

\large

\textbf{Questão 6}

\normalsize

\vspace{6mm}

Qual a diferença entre uma superfície de restrição e uma superfície de restrição
composta?

\vspace{3mm}

Superfície de restrição: O conjunto de pontos que resultam em $g_i(x) = 0$; $1 \le i \le p$. As superfícies de restrição dividem o
espaço de busca em duas regiões, $g_i(x) > 0$ e $g_i(x) < 0$. Por exemplo, uma curva em $\mathbb{R}^2$.

Superfície de restrição composta: Por exemplo, um triângulo em $\mathbb{R}^2$ divide o espaço de busca em interior e exterior e pode ser descrito como a interseção de 3 semiplanos.
Já a superfície de restrição deveria ser escrita de forma mais complexa como união de 3 segmentos.

\vspace{6mm}

\large

\textbf{Questão 7}

\normalsize

\vspace{6mm}

Qual a diferença entre um ponto de limite e um ponto factível?

\vspace{3mm}

Ponto de limite: Satisfaz somente $g_i(x) = 0$, para algum $i$.

Ponto factível: Satisfaz todas as restrições simultaneamente (igualdade e desigualdade).

\vspace{6mm}

\large

\textbf{Questão 8}

\normalsize

\vspace{6mm}

Escolha um sistema de seu interesse e apresente um problema de otimização para
tal sistema juntamente com uma descrição sucinta das funções consideradas.

\vspace{3mm}

Ainda no curso técnico do CEFET, caiu na prova de geometria espacial: \emph{Uma formiga anda sobre um cubo de lado L. Qual é a menor distância d que ela consegue caminhar entre dois vértices opostos?}

Eu resolvi da seguinte forma: precisamos passar por duas faces adjacentes. É claro que são duas retas. Mas a qual altura $h$ do primeiro quadrado ela vai passar?

Temos um triângulo retângulo de base $L$ e altura $h$, logo a hipotenusa é $\sqrt{L^2 + h^2}$.

Na outra face, temos um triângulo retângulo de base $L$ e altura $L - h$, logo a hipotenusa é $\sqrt{L^2 + (L - h)^2}$.

\newpage

Queremos minimizar a soma: $d = \min f(x)$, em que a função objetivo é $f : \mathbb{R} \to \mathbb{R}$; $f(x) = \sqrt{L^2 + x^2} + \sqrt{L^2 + (L - x)^2}$.

$x^* = \arg \min f(x)$ é um problema de minimização irrestrita. $L$ é um parâmetro fixo. O espaço de busca é uma altura $0 \le x^* \le L$.

(Daí depois da prova ela chega em sala, planifica e vira um retângulo de base $2L$ e altura $L$. Portanto, $d = \sqrt{(2L)^2 + L^2}$.)

\vspace{6mm}

Versão de 08/novembro/2021\footnote{Fora da caridade não há salvação.}  por Vinicius Claudino Ferraz.

\end{document}

\documentclass[11pt,a4paper]{article}
\usepackage{amssymb} %mathbb
\usepackage{amsmath} %align
\usepackage{graphicx} %jpg
\usepackage{cancel}
\usepackage{hyperref} %a href
\usepackage{gensymb}
\usepackage{colortbl}
\usepackage{amsthm}
\newtheorem{exercise}{Exercise}
\newtheorem{definition}{Definition}
\newtheorem{thm}{Theorem}
\newtheorem{example}{Example}
\newtheorem*{coro}{Main Corollary}
\usepackage[top=1.0cm,bottom=1.3cm,left=1.0cm,right=1.0cm]{geometry}

\begin{document}

Main Video on \href{https://www.youtube.com/watch?v=wio-6zjSsV4}{\color{blue}\underline{YouTube}}.

\section{Introduction}

We know, by Wolfram's and Alpha's powers that:

\begin{align}
\cos \cfrac{2 \pi}{5} &= \cfrac{\sqrt{5} - 1}{4} \Rightarrow \sin \cfrac{2 \pi}{5} = \cfrac{\sqrt{2}}{4} \sqrt{5 + \sqrt{5}} \\
\cos 30\degree &= 2 \cos^2 15\degree - 1 \\
a_1 &= 2 x_1^2 - 1 \Rightarrow x_1 = \sqrt{\cfrac{a_1 + 1}{2}} \\
\cos 72\degree &= 2 \cos^2 36\degree - 1 \\
a_2 &= 2 x_2^2 - 1 \Rightarrow x_2 = \sqrt{\cfrac{a_2 + 1}{2}} = \sqrt{\cfrac{\cfrac{\sqrt{5} - 1}{4} + 1}{2}} \\
\cos 36\degree &= 2 \cos^2 18\degree - 1 \\
a_3 &= 2 x_3^2 - 1 \Rightarrow x_3 = \sqrt{\cfrac{a_3 + 1}{2}} \\
\cos 18\degree &= 2 \cos^2 9\degree - 1 \\
a_4 &= 2 x_4^2 - 1 \Rightarrow x_4 = \sqrt{\cfrac{a_4 + 1}{2}} \\
\cos 27\degree &= 4 x_4^3 - 3 x_4 = x_5 \\
\cos 30\degree &= 1 - 2 \sin^2 15\degree \\
a_1 &= 1 - 2 y_1^2 \Rightarrow y_1 = \sqrt{\cfrac{1 - a_1}{2}}\,;\,y_2 = \sqrt{\cfrac{1 - a_2}{2}}\,;\,y_3 = \sqrt{\cfrac{1 - a_3}{2}}\,;\,y_4 = \sqrt{\cfrac{1 - a_4}{2}} \\
\sin 27\degree &= \sin 9\degree (3 - 4 \sin^2 9\degree) \\
y_5 &= y_4 (3 - 4 y_4^2) \\
\cos (30\degree - 27\degree) &= \cfrac{\sqrt{3}}{2} \cdot x_5 + \cfrac{1}{2} \cdot y_5 = x_6 \\
\sin (30\degree - 27\degree) &= \cfrac{1}{2} \cdot x_5 - \cfrac{\sqrt{3}}{2} \cdot y_5 = y_6 \\
\cos 3\degree &= 4 \cos^3 1\degree - 3 \cos 1\degree \\
4z^3 - 3z - x_6 &= 0
\end{align}

\vspace{3mm}

Now, let's use Cardano:

\begin{align}
x^3 + px + q &= 0 \\
p &= - \cfrac{3}{4} \\
q &= - \cfrac{1}{4} \cdot \left[\cfrac{\sqrt{3}}{2} \cdot \sqrt{\cfrac{a_4 + 1}{2}} \cdot \left(4 \cdot \cfrac{a_4 + 1}{2} - 3 \right) + \cfrac{1}{2} \cdot \sqrt{\cfrac{1 - a_4}{2}} \left(3 - 4 \cdot \cfrac{1 - a_4}{2} \right) \right] \\
q &= - \cfrac{\sqrt{6}}{16} \cdot \sqrt{a_4 + 1} \cdot (2a_4 - 1 ) - \cfrac{\sqrt{2}}{16} \cdot \sqrt{1 - a_4} \cdot (1 + 2 a_4)
\end{align}

\begin{align}
q &= - \cfrac{\sqrt{6}}{16} \cdot \sqrt{\sqrt{\cfrac{\sqrt{\cfrac{\cfrac{\sqrt{5} - 1}{4} + 1}{2}} + 1}{2}} + 1} \cdot \bigg(2\sqrt{\cfrac{\sqrt{\cfrac{\cfrac{\sqrt{5} - 1}{4} + 1}{2}} + 1}{2}} - 1 \bigg) -\\
  &- \cfrac{\sqrt{2}}{16} \cdot \sqrt{1 - \sqrt{\cfrac{\sqrt{\cfrac{\cfrac{\sqrt{5} - 1}{4} + 1}{2}} + 1}{2}}} \cdot \bigg(1 + 2 \sqrt{\cfrac{\sqrt{\cfrac{\cfrac{\sqrt{5} - 1}{4} + 1}{2}} + 1}{2}}\bigg) \\
\cfrac{q^2}{4} + \cfrac{p^3}{27} &= \cfrac{q^2}{4} - \cfrac{1}{64} = \cfrac{16q^2 - 1}{64} \\
w &= - \cfrac{q}{2} + i \cdot \cfrac{\sqrt{1 - 16 q^2}}{8} = z^3\,;\, \arg w = \theta \\
W &= \overline{w} = Z^3\,;\, \arg W = \overline{\theta} \\
|w|^2 &= \cfrac{q^2}{4} + \cfrac{1 - 16 q^2}{64} = \cfrac{1}{64} \Rightarrow |z| = \cfrac{1}{2} \\
\tan \theta &= - \cfrac{\sqrt{1 - 16q^2}}{4q}\,;\, \overline{\theta} = 360\degree - \theta \\
\arg z &\in \{ \varphi_1, \varphi_1 + 120\degree, \varphi_1 + 240\degree \}\,;\, \varphi_1 = \cfrac{\theta}{3} \\
\sqrt[3]{w} &= \cfrac{1}{2} \cdot \exp i (\varphi_1 + k_1) \\
\arg Z &\in \{ \varphi_2, \varphi_2 + 120\degree, \varphi_2 + 240\degree \}\,;\, \varphi_2 = \cfrac{360\degree - \theta}{3} \\
&= \{ 120\degree  - \varphi_1, 240\degree - \varphi_1, - \varphi_1 \} \\
\sqrt[3]{W} &= \cfrac{1}{2} \cdot \exp i (k_2 - \varphi_1) \\
x_1 &= \cfrac{1}{2} \cdot \exp i (1\degree) + \cfrac{1}{2} \cdot \exp i (-1\degree) = \cos 1\degree \\
x_2 &= \exp i (120\degree) \cdot \cfrac{1}{2} \cdot \exp i (1\degree) + \exp i (240\degree) \cdot \cfrac{1}{2} \cdot \exp i (-1\degree) = - \cos 59\degree \\
x_3 &= \exp i (240\degree) \cdot \cfrac{1}{2} \cdot \exp i (1\degree) + \exp i (120\degree) \cdot \cfrac{1}{2} \cdot \exp i (-1\degree) = - \cos 61\degree \\
\cos \cfrac{t}{3} &= \cfrac{1}{2} \cdot \text{principal value of } \sqrt[3]{\cos 3t + i \cdot \sin 3t} + \cfrac{1}{2} \cdot \text{principal value of } \sqrt[3]{\cos 3t + i \cdot \sin 3t} \\
\sin \cfrac{t}{3} &= \cfrac{1}{2} \cdot \text{principal value of } \sqrt[3]{- \sin 3t + i \cdot \cos 3t} + \cfrac{1}{2} \cdot \text{principal value of } \sqrt[3]{- \sin 3t - i \cdot \cos 3t} \,\,\blacksquare
\end{align}

\vspace{3mm}

\section{Trisection Theorems}

\begin{align}
x^3 - \cfrac{3}{4} \cdot x - \cfrac{\cos 3t}{4} = 0 \Leftrightarrow x &\in \{ \cos t, \cos (t + 120\degree), \cos (t + 240\degree) \} \\
x^3 - \cfrac{3}{4} \cdot x + \cfrac{\sin 3t}{4} = 0 \Leftrightarrow x &\in \{ \cos (90\degree - t), \cos (210\degree - t), \cos (330\degree - t) \} \\
x &\in \{ \sin t, \sin (t - 120\degree), \sin (t - 240\degree) \} \\
x^3 + \cfrac{3 \sin 3t}{4} \cdot x^2 + \cfrac{12 \sin^2 3t - 27}{64} \cdot x + \cfrac{\sin^3 3t}{64} = 0 \Leftrightarrow x &\in \{ \sin^3 t, \sin^3 (t - 120\degree), \sin^3 (t - 240\degree) \} \\
x^3 - \cfrac{3 \cos 3t}{4} \cdot x^2 + \cfrac{12 \cos^2 3t - 27}{64} \cdot x - \cfrac{\cos^3 3t}{64} = 0 \Leftrightarrow x &\in \{ \cos^3 t, \cos^3 (t + 120\degree), \cos^3 (t + 240\degree) \}
\end{align}

\vspace{100mm}

\textbf{Proof of the last equation: }

\begin{align}
\blacktriangleright\,\,(a + bi)^3 &= \cos 3t + i \sin 3t = k + si \\
a^3 - 3ab^2 &= k \Rightarrow b^2 = \cfrac{a^3 - k}{3a} \\
3a^2b - b^3 &= s \\
b^2(3a^2 -b^2)^2 &= s^2 \\
\cfrac{a^3 - k}{3a} \cdot \cfrac{(9a^3 - a^3 + k)^2}{9a^2} &= s^2\,;\,a^3 = x \\
(x - k)(8x + k)^2 &= 27s^2 x \\
(x - k)(64x^2 + 16kx + k^2) &= 27x - 27k^2x\,\,\blacksquare
\end{align}

\begin{exercise}
$(a + bi)^3 = \exp i(270\degree - 3t)$.
\end{exercise}

\begin{exercise}
$\cos^a (t/3)$ and $\sin^a (t/3)$.
\end{exercise}

\begin{example}
$x^3 + 3/8 \cdot x^2 - 3/8 \cdot x + 1/512 = 0$.
\end{example}

\begin{example}
$x^3 + 3/8 \cdot \sqrt{3} \cdot x^2 - 9/32 \cdot x + 3/512 \cdot \sqrt{3} = 0$.
\end{example}

\section{Any Degree | $n$-Section Theorem}

The equations below are soluble by radicals:

\begin{align}
\cos \cfrac{t}{n} &= \cfrac{1}{2} \cdot \text{principal value of } \sqrt[n]{\cos nt + i \cdot \sin nt} + \cfrac{1}{2} \cdot \text{principal value of } \sqrt[n]{\cos nt + i \cdot \sin nt} \\
\sin \cfrac{t}{n} &= \cfrac{1}{2} \cdot \text{principal value of } \sqrt[n]{- \sin nt + i \cdot \cos nt} + \cfrac{1}{2} \cdot \text{principal value of } \sqrt[n]{- \sin nt - i \cdot \cos nt} \\
n > 0 \Rightarrow \cos nt &= \sum_{i = 0}^{\lfloor n/2 \rfloor} (-1)^i \cdot n \cdot \cfrac{(n - i - 1)!}{i! (n - 2i)!} \cdot 2^{n - 2i - 1} \cdot (\cos t)^{n - 2i} = p(\cos t) \\
p_n(x) &= 0 \Leftrightarrow p(x) = \cos nt \Leftrightarrow x \in \left\{ \cos \left(t + \cfrac{2k\pi}{n} \right) \,;\, 0 \le k \le n - 1 \right\} \\
q_n(x) &= 0 \Leftrightarrow p(x) = \cos \left(\cfrac{n\pi}{2} - nt \right) \Leftrightarrow x \in \left\{ \sin \left(t - \cfrac{2k\pi}{n} \right) \,;\, 0 \le k \le n - 1 \right\} \\
p_n(x) &= \sum_{i = 0}^n \alpha(n, i)\cdot x^i
\end{align}

\begin{example}
$\cos 2t = 2x^2 - 1\,;\, \cos 3t = 4x^3 - 3x\,;\, \cos 4t = 8x^4 - 8x^2 + 1\,;\, \cos 5t = 16x^5 - 20x^3 + 5x$.
\end{example}

Study with more details on this link to \href{https://en.wikipedia.org/wiki/Chebyshev_polynomials}{\color{blue}\underline{WikiPedia}}.

\vspace{3mm}

Cossines's and Chebyshev's triangle:

\begin{equation}
\left(\begin{matrix}
- \cos 0t + 1 \\
- \cos 1t & 1 \\
- \cos 2t - 1 & 0  & 2 \\
- \cos 3t     & -3 & 0  & 4 \\
- \cos 4t + 1 & 0  & -8 & 0   & 8 \\
- \cos 5t     & 5 & 0  & -20 & 0 & 16 \\
- \cos 6t -1 & 0 & 18 & 0 & -48 & 0 & 32 \\
- \cos 7t & -7 & 0 & 56 & 0 & -112 & 0 & 64
\end{matrix}
\right)
\end{equation}

\begin{exercise}
$\cos 1t = x\,;\,\cos 0t = 1$.
\end{exercise}

\begin{exercise}
$\cos^a (t/n)$.
\end{exercise}

\begin{exercise}
$\sin^a (t/n)$.
\end{exercise}

Sines's and Chebyshev's triangle:

\begin{equation}
\left(\begin{matrix}
- \cos 0t + 1 \\
- \sin 1t & 1 \\
+ \cos 2t - 1 & 0  & 2 \\
+ \sin 3t     & -3 & 0  & 4 \\
- \cos 4t + 1 & 0  & -8 & 0   & 8 \\
- \sin 5t     & 5 & 0  & -20 & 0 & 16 \\
+ \cos 6t -1 & 0 & 18 & 0 & -48 & 0 & 32 \\
+ \sin 7t & -7 & 0 & 56 & 0 & -112 & 0 & 64
\end{matrix}
\right)
\end{equation}

\begin{exercise}
To construct two tetrahedra for $\cos^a (t/n)$ and $\sin^a (t/n)$. For each $(a, n)$, there is a polynomial.
\end{exercise}

\section{Girard's Equations on Trisection}

\begin{align}
\cos t + \cos (t + 120\degree) + \cos (t + 240\degree) &= 0 \\
\cos t \cos (t + 120\degree) + \cos t \cos (t + 240\degree) + \cos (t + 120\degree) \cos (t + 240\degree) &= - \cfrac{3}{4} \\
\cos t \cos (t + 120\degree) \cos (t + 240\degree) &= \cfrac{\cos 3t}{4} \\
\sin t + \sin (t - 120\degree) + \sin (t - 240\degree) &= 0 \\
\sin t \sin (t - 120\degree) + \sin t \sin (t - 240\degree) + \sin (t - 120\degree) \sin (t - 240\degree) &= - \cfrac{3}{4} \\
\sin t \sin (t - 120\degree) \sin (t - 240\degree) &= - \cfrac{\sin 3t}{4}
\end{align}

\vspace{3mm}

He who trisects $3t$, also trisects $270\degree - 3t$ and finds $\{ t + 30\degree, t + 150\degree, t + 270\degree, t, t + 120\degree, t + 240\degree \}$ too.

\section{Pentasection | Theorem and Girard}

The equations below are soluble by radicals:

\begin{align}
&x^5 - \cfrac{20}{16} \cdot x^3 + \cfrac{5}{16} \cdot x - \cfrac{\cos 5t}{16} = 0 \Leftrightarrow x \in \{ \cos t, \cos (t + 72\degree), \cos (t + 144\degree), \cos (t + 216\degree), \cos (t + 288\degree) \} \\
&x^5 - \cfrac{20}{16} \cdot x^3 + \cfrac{5}{16} \cdot x - \cfrac{\sin 5t}{16} = 0 \Leftrightarrow x \in \{ \sin t, \sin (t - 72\degree), \sin (t - 144\degree), \sin (t - 216\degree), \sin (t - 288\degree) \} \\
0 &= S_1 = \sum c_i = S_1' = \sum s_i \\
- \cfrac{20}{16} &= S_2 = \sum c_i c_j = S_2' = \sum s_i s_j \\
0 &= S_3 = S_3' \\
\cfrac{5}{16} &= S_4 = S_4'
\end{align}

\begin{align}
- \cfrac{\cos 5t}{16} &= \cos t \cos (t + 72\degree) \cos (t + 144\degree) \cos (t + 216\degree) \cos (t + 288\degree) \\
- \cfrac{\sin 5t}{16} &= \sin t \sin (t - 72\degree) \sin (t - 144\degree) \sin (t - 216\degree) \sin (t - 288\degree)
\end{align}

\vspace{3mm}

\begin{exercise}
$\cos^a (t/5)$.
\end{exercise}

Sketch of solution:

\begin{align}
x^5 &= \cfrac{20}{16} \cdot x^3 - \cfrac{5}{16} \cdot x + \cfrac{\cos 5t}{16} \\
x - \cos^5 t &= x - \cfrac{20}{16} \cdot \cos^3 t + \cfrac{5}{16} \cdot \cos t - \cfrac{\cos 5t}{16} \\
t' &= t + 72\degree \\
(x - \cos^5 t)(x - \cos^5 t')(x - \cos^5 t'')(x - \cos^5 t''')(x - \cos^5 t'''') &= x^5 - \sigma_1 x^4 + \sigma_2 x^3 - \sigma_3 x^2 + \sigma_4 x - \sigma_5 = 0 \\
a + b + c + d + e &= 0 \\
ab + \cdots + de &= -20/16 \\
abc + \cdots + cde &= 0 \\
abcd + \cdots + bcde &= 5/16 \\
abcde &= -1/16 \cdot \cos 5t \\
a^5 + b^5 + c^5 + d^5 + e^5 &= \sigma_1 \\
a^5b^5 + \cdots + d^5e^5 &= \sigma_2 \\
a^5b^5c^5 + \cdots + c^5d^5e^5 &= \sigma_3 \\
a^5b^5c^5d^5 + \cdots + b^5c^5d^5e^5 &= \sigma_4 \\
(abcde)^5 &= \sigma_5 = -1/2^{20} \cdot \cos^5 5t \\
\end{align}

\begin{exercise}
$\sin^a (t/5)$.
\end{exercise}

He who pentasects $5t$, also pentasects $450\degree - 5t$ and finds $t + k \degree\,;\, k \in \{ 0, 54, 72, 126, 144, 198, 216, 270, 288, 342 \}$ too.

\section{$(x - r_1^a)\cdots(x - r_n^a) = 0$}

\begin{thm}
Let the polynomial equation be $p(x) = 0\,;\,\deg p = n$, whose roots are $r_i$. Let $a \ge 2$ be a natural. Then, it's always possible to construct another polynomial $q(x) = 0$, with the same degree $n$, whose roots are $r_i^a$, without solving the equation $p(x) = 0$ and only by using Girard's relations.
\end{thm}

\section{Heptasection}

The equations below are soluble by radicals:

\begin{align}
&x^7 - \cfrac{112}{64} \cdot x^5 + \cfrac{56}{64} \cdot x^3 - \cfrac{7}{64} \cdot x - \cfrac{\cos 7t}{64} = 0 \Leftrightarrow x \in \left\{ \cos \left(t + \cfrac{2k\pi}{7} \right) \,;\, 0 \le k \le 6 \right\} \\
&x^7 - \cfrac{112}{64} \cdot x^5 + \cfrac{56}{64} \cdot x^3 - \cfrac{7}{64} \cdot x + \cfrac{\sin 7t}{64} = 0 \Leftrightarrow x \in \left\{ \sin \left(t - \cfrac{2k\pi}{7} \right) \,;\, 0 \le k \le 6 \right\}
\end{align}

\vspace{3mm}

\begin{coro}
$A = \{ \cos^a (t/q)\,;\,\sin^a (t/q)\,\vert\,t \in \arccos \mathbb{Q}\,;\, q \in \mathbb{Q}\,;\,a \in \mathbb{N} \}$ is a subset of the algebrical numbers.
\end{coro}

\begin{exercise}
Girard: \\
\begin{align}
0 &= S_1 \\
- \cfrac{112}{64} &= S_2 \\
0 &= S_3 \\
\cfrac{56}{64} &= S_4 \\
0 &= S_5 \\
- \cfrac{7}{64} &= S_6 \\
\cfrac{\cos 7t}{64} &= P
\end{align}
\end{exercise}

\begin{exercise}
$\cos^a (t/7)$.
\end{exercise}

\begin{exercise}
$\sin^a (t/7)$.
\end{exercise}

He who heptasects $7t$, also heptasects $630\degree - 7t$ and finds $t + k \degree\,;\, k \in \{0, 90/7, 360/7, 450/7, 720/7, 810/7, 1080/7,$

$1170/7, 1440/7, 1530/7, 1800/7, 270, 2160/7, 2250/7 \}$ too.

\section{Girard's Equations on $n$-section}

\begin{align}
S_1 &= \sum c_i = - \cfrac{\alpha(n, n - 1)}{2^{n - 1}} \\
S_2 &= \sum c_i c_j = + \cfrac{\alpha(n, n - 2)}{2^{n - 1}} \\
S_i &= (-1)^i \cdot \cfrac{\alpha(n, n - i)}{2^{n - 1}} \\
S_n &= P = (-1)^n \cdot \cfrac{\alpha(n, 0)}{2^{n - 1}}
\end{align}

\begin{exercise}
To construct a polygon of $n = 11$ sides. $n = 13, 17, 19, 23, \cdots$
\end{exercise}

\begin{align}
\cos nt = &+ 2^{n - 1} \cdot x^n + 0 \\
&- n \cdot 2^{n - 3} \cdot x^{n - 2} + 0 \\
&+ n(n-3)/2 \cdot 2^{n - 5} \cdot x^{n - 4} + 0 \\
&- n(n-4)(n-5)/6 \cdot 2^{n - 7} \cdot x^{n - 6} + 0 \\
&+ n(n-5)(n-6)(n-7)/24 \cdot 2^{n - 9} \cdot x^{n - 8} + 0 \\
&- n(n-6)(n-7)(n-8)(n-9)/5! \cdot 2^{n - 11} \cdot x^{n - 10} + 0 \\
&+ n(n-7,8,9,10,11)/6! \cdot 2^{n - 13} \cdot x^{n - 12} + 0 \\
&- n(n-8,9,10,11,12,13)/7! \cdot 2^{n - 15} \cdot x^{n - 14} + 0 \\
&+ n(n-9,10,11,12,13,14,15)/8! \cdot 2^{n - 17} \cdot x^{n - 16} + 0 \\
&- n(n-10,11,12,13,14,15,16,17)/9! \cdot 2^{n - 19} \cdot x^{n - 18} + 0 \\
&+ n(n-11,12,13,14,15,16,17,18,19)/10! \cdot 2^{n - 21} \cdot x^{n - 20} + 0 \\
&- n(n-12,13,14,15,16,17,18,19,20,21)/11! \cdot 2^{n - 23} \cdot x^{n - 22} + 0
\end{align}

\section{$\cos z \in \mathbb{C}$}

\begin{align}
\cos z &= a + bi \\
\cos(x + yi) &= \cos x \cosh y - i \sin x \sinh y \\
\cos x \cosh y &= a \\
- \sin x \sinh y &= b \\
\cosh^2 y - \sinh^2 y &= 1 = \cfrac{a^2}{\cos^2 x} - \cfrac{b^2}{\sin^2 x}\,;\,u = \cos^2 x \\
1 &= \cfrac{a^2}{u} + \cfrac{b^2}{u - 1} \\
u^2 - u &= a^2(u - 1) + b^2 u \\
u^2 + u(-a^2 - b^2 - 1) + a^2 &= 0 \\
\Delta &= (a^2 + b^2 + 1)^2 - 4 a^2 \\
\cos^2 x &= \cfrac{a^2 + b^2 + 1 \pm \sqrt{\Delta}}{2} \\
\cos x &\in \{c_1, c_2, c_3, c_4\} \\
x &\in \arccos c_i + 2 \pi \cdot \mathbb{Z} \\
\cosh y &= \cfrac{a}{c_i}
\end{align}

\begin{example}
$\arccos (3 + 4i) = \href{https://www.wolframalpha.com/input/?i=cos(cos\%5E(-1)(sqrt(13+\%2B+4+\%5Csqrt(10)))+\%2B+i+*+cosh\%5E(-1)(+3\%2Fsqrt(13+\%2B+4+\%5Csqrt\%7B10\%7D)))}{\color{blue}\underline{w}}$.
\end{example}

\begin{align}
\cos^2 x &= 13 \pm \sqrt{13^2 - 9} \\
\cos x &\in \{c_1, c_2, c_3, c_4\} \\
\cosh y &= \cfrac{3}{c} \\
x + yi &= \arccos \sqrt{13 + 4 \sqrt{10}} + i\cdot \text{argcosh } \left( \cfrac{3}{\sqrt{13 + 4 \sqrt{10}}} \right)
\end{align}

\begin{exercise}
$\cos nz = 3 + 4i = \sin nw$.
\end{exercise}

\section{Third Degree | Reduction of order}

\begin{equation*}
x^3 - Sx^2 + Qx - P = 0 \Leftrightarrow \left\{\begin{aligned}
x + y + z &= S \\
x^2 + y^2 + z^2 &= S^2 - 2Q \\
x^3 + y^3 + z^3 &= S^3 - 3SQ + 3P
\end{aligned}\right.
\end{equation*}

\begin{align}
z &= S - x - y \\
w &= x + y \\
x^2 + y^2 + \cancel{S^2} - 2Sw + w^2 &= \cancel{S^2} - 2Q \\
x^2 + y^2 - 2Sx - 2Sy + x^2 + 2xy + y^2 &= - 2Q
\end{align}

\begin{align}
2y^2 + y(- 2S + 2x) + 2x^2 - 2Sx + 2Q &= 0 \\
y^2 + y(x - S) + (x^2 - Sx + Q) &= 0 \\
\Delta &= x^2 - 2Sx + S^2 - 4x^2 + 4 Sx - 4Q
\end{align}

\vspace{3mm}

\begin{thm}
If we know a single root of the equation $x^3 - Sx^2 + Qx - P = 0$, \\
then the other two roots are:
\begin{align}
f_1^2(x) &= \cfrac{S - x \pm \sqrt{\Delta}}{2},
\end{align}
where $\Delta = -3x^2 + 2Sx + S^2 - 4Q$.
\end{thm}

\vspace{3mm}

\begin{example}
We know $\cos 1\degree$.
\end{example}

\begin{align}
x^3 + px + q &= 0 \\
S &= 0\,;\,Q = -\cfrac{3}{4}\,;\,P = - \cos 3\degree \\
\cos^2 1\degree + \cos^2 59\degree + \cos^2 61\degree &= \cfrac{3}{2} \\
\cos^3 1\degree - \cos^3 59\degree - \cos^3 61\degree &= \cfrac{3}{4} \cdot \cos 3\degree \\
x_0 &= \cos 1\degree \\
\Delta &= -3 \cos^2 1\degree + 3 \\
y &= \cfrac{- x \pm \sqrt{\Delta}}{2}
\end{align}

\begin{align}
- \cos 61\degree &= \cfrac{- \cos 1\degree + \sqrt{3 - 3 \cos^2 1\degree}}{2} \\
- \cos 59\degree &= \cfrac{- \cos 1\degree - \sqrt{3 - 3 \cos^2 1\degree}}{2}
\end{align}

\section{Any Degree | Reduction of Order}

\begin{thm}
By Galois's theorem, let $a \le 4$. If we know $b = n - a$ roots of the equation
\begin{align}
x^n - S_1 x^{n - 1} + \cdots + (-1)^{n - 1} S_{n - 1} x + (-1)^n S_n &= 0,
\end{align}
then the other $a$ roots are:
\begin{align}
f_{n - a}^i(x_1, \cdots, x_{n - a}) &\in \{ y_1, \cdots, y_a \}, \forall i \in \{ 1, \cdots, a \}.
\end{align}
\end{thm}

\vspace{3mm}

\textbf{Proof:} $\blacktriangleright\,\,$Divide the original $p = (1, -S_1, + S_2, \cdots)$ by $(x - x_1)$ to obtain $q = (1, -S_1', + S_2', \cdots)$,

where $S_1' = S_1 - x_1$, $S_2' = S_2 - x_1 S_1'$ and so on. Now, solve $q(x) = 0$.$\,\,\blacksquare$

\section{Variation of Parameters | From Polynomials to Equivalent Exponentials}

\begin{align}
\exp r_i t &= 1 f_1(t) + r_i f_2(t) + \cfrac{r_i^2}{2}\cdot f_3(t) + \cdots + \cfrac{r_i^n}{n!}\cdot f_{n + 1}(t) \\
E &= T\cdot F \\
\left[E^\top \right]_{1 \times n} &= [F^\top]_{1 \times n} \cdot \left[T^\top \right]_{n \times n} \\
[E']^\top &= [F']^\top\cdot T^\top \text{ and all other }n\text{ derivatives} \\
\left[E^\top \right]_{n \times n} &= [\exp M]_{n \times n} \cdot \left[T^\top \right]_{n \times n}
\end{align}

We have polynomial of $n$-th degree of $x$ with real coefficients equals to zero.

\begin{align}
p_n(x) &= 0 \Leftrightarrow x \in \{ r_1, \cdots, r_n \} = R
\end{align}

We have exponential of $xt$ plus polynomial of $(n-1)$-th degree of $x$ with coefficients $f_i(t)$ equals to zero.

\begin{align}
\varphi_{n - 1}(x, t) &= e^{xt} + p'_{n-1}(x, t) \\
\varphi_{n - 1}(x, t) &= 0 \Leftarrow x \in R, \forall t \in \mathbb{R} \\
\varphi_{n - 1}^{-1}(0) &\supset R \times \mathbb{R}
\end{align}

And for each term $a$ of the Taylor series $\exp rt = \sum \cfrac{r^a}{a!} \cdot t^a$,

\begin{align}
\psi_a(x, a) &= x^a + p''_{n-1}(x, n) \\
\psi_a(x, a) &= 0 \Leftarrow x \in R, \forall a \in \mathbb{N} \\
\psi_a^{-1}(0) &\supset R \times \mathbb{N}
\end{align}

\section{Hungerford's Algebra}

\begin{definition}
Let $E$ and $F$ be extension fields of a field $K$. A nonzero map $\sigma: E \to F$ which is both a field and a $K$-module homomorphism is called a $K$-homomorphism.
\end{definition}

\begin{definition}
If a field automorphism $\sigma \in$ Aut $F$ is a $K$-homomorphism, then $\sigma$ is called a $K$-automorphism of $F$.
\end{definition}

\begin{definition}
The group of $\{\sigma\,;\, \sigma$ is $K$-automorphism of $F\}$ is called the Galois group of $F$ over $K$ and is denoted $\text{Aut}_K\,F$.
\end{definition}

\begin{definition}
Let $F$ be a field and $f \in F[x]$ a polynomial of positive degree. $f$ is said to split in $F[x]$ if $f$ can be written as a product of linear factors in $F[x]$; that is, $f = u_0(x - u_1)\cdots(x - u_n)$, with $u_i \in F$.
\end{definition}

\begin{definition}
Let $K$ be a field and $f \in K[x]$ a polynomial of positive degree. An extension field $F$ of $K$ is said to be an splitting field over $K$ of the polynomial $f$ if $f$ splits in $F[x]$ and $F = K(u_1, \cdots, u_n)$, where $u_i$ are the roots of $f$ in $F$.
\end{definition}

\begin{definition}
Let $K$ be a field. The Galois group of a polynomial $f \in K[x]$ is the group $\text{Aut}_K\,F$, where $F$ is a splitting field of $f$ over $K$.
\end{definition}

\begin{thm}
If $p$ is prime and $f$ is an irreductible polynomial of degree $p$ over the field of rational numbers which has precisely two nonreal roots in the field of complex numbers, then the Galois group of $f$ is isomorphic to $S_p$, the group of permutations of $\{1, 2, \cdots, p \}$.
\end{thm}

\begin{thm}
The Galois group of $x^5 - 4x + 2 = 0$ is $S_5$.
\end{thm}

\begin{thm}
Every permutation is expressable as a composition of $n$ transpositions, where $n$ is even or odd.
\end{thm}

\begin{definition}
$A_n = \{\sigma \in S_n \,;\, \sigma \text{ is even} \}$
\end{definition}

\begin{definition}
The commutator of $A_n$ is defined as $A_n' = \{ ab a^{-1} b^{-1}\,;\,a, b \in A_n \}$.
\end{definition}

\begin{thm}
$A_5$ is not solvable.
\end{thm}

\textbf{Proof:} $\blacktriangleright\,\,ab = ba \Rightarrow ab a^{-1} b^{-1} = e$ \\
$A_5$ is not abelian. $A_5' \lhd A_5$. $A_5$ is simple. \\
Therefore, $A_5' = A_5$. If $A_5^{(n)}$ were eventually equal to $(e)$, then $A_5$ would be solvable. \\
Therefore, $A_5$ is not solvable. $\,\,\blacksquare$

\begin{thm}
$A_5 < S_5$ is not solvable.
\end{thm}

\begin{thm}
Let $K$ be a field and $f \in K[x]$ a polynomial of degree $n > 0$, where char $K$ does not divide $n!$ (which is always true when char $K\,= 0$). Then the equation $f(x) = 0$ is solvable by radicals if and only if the Galois group of $f$ is solvable.
\end{thm}

\vspace{3mm}

quod erat demonstrandum | Out of charity, there is no salvation at all.

\vspace{3mm}

Vinicius Claudino FERRAZ, 5/May/2019, Release 3.2.4

\end{document}

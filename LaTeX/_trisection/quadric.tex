\documentclass[11pt,a4paper]{article}
\usepackage{amssymb} %mathbb
\usepackage{amsmath} %align
\usepackage{graphicx} %jpg
\usepackage{cancel}
\usepackage[top=1.0cm,bottom=1.3cm,left=1.0cm,right=1.0cm,landscape]{geometry}

\begin{document}

\begin{equation*}
x^4 - Sx^3 + Qx^2 - Tx + P = 0 \Leftrightarrow \left\{\begin{aligned}
x + y + z + w &= S \\
x^2 + y^2 + z^2 + w^2 &= S^2 - 2 Q \\
x^3 + y^3 + z^3 + w^3 &= S^3 - 3SQ + 3T \\
x^4 + y^4 + z^4 + w^4 &= S^4 - 4S^2Q + 2Q^2 + 4ST - 4P
\end{aligned}\right.
\end{equation*}

\textbf{Theorem: }If we know a single root of the equation $x^4 - Sx^3 + Qx^2 - Tx + P = 0$, then the other three roots are:

\begin{align}
y(x) &= \zeta_1(x) + \zeta_2(x) - \cfrac{x + 17}{3} \\
z(x) &= u \cdot \zeta_1(x) + \overline{u} \cdot \zeta_2(x) - \cfrac{x + 17}{3} \\
w(x) &= \overline{u} \cdot \zeta_1(x) + u \cdot \zeta_2(x) - \cfrac{x + 17}{3}
\end{align}

Where

\begin{align}
f(x) &= (20x^3 + 255x^2 + 951 x + 1042)^2 + 4 (2x^2 + 17x + 14)^3 \\
\omega_1(x) &= \cfrac{-20x^3 - 255x^2 - 951 x - 1042 + \sqrt{f(x)}}{54} \\
\omega_2(x) &= \cfrac{-20x^3 - 255x^2 - 951 x - 1042 - \sqrt{f(x)}}{54} \\
u &= \exp i(120^o) \Rightarrow \overline{u} = \exp i(240^o) \\
\zeta_1(x) &= \text{principal value of } \sqrt[3]{\omega_1} \\
\zeta_2(x) &= \text{principal value of } \sqrt[3]{\omega_2} \\
\end{align}

Particularly,

\begin{align}
x + y + z + w &= -17 \\
w &= - x - y - z - 17 \\
x^2 + y^2 + z^2 + (x + y + z + 17)^2 &= 87 \\
2x^2 + 2y^2 + 2z^2 + 2xy + 2xz + 34x + 2yz + 34y + 34z + 202 &= 0 \\
x^2 + y^2 + z^2 + xy + xz + 17x + yz + 17y + 17z + 101 &= 0 \\
z^2 + (x + y + 17)z + x^2 + y^2 + xy + 17x + 17y + 101 &= 0
\end{align}

\begin{align}
\Delta &= x^2 + y^2 + 289 + 2xy + 34x + 34y - 4x^2 - 4 y^2 - 4xy - 68x - 68y - 404 \\
\Delta &= -3 x^2 -3 y^2 - 2xy - 34x - 34y - 115 \\
z &= \cfrac{- x - y - 17 + \epsilon \sqrt{\Delta}}{2} \\
w &= \cfrac{-x - y - 17 - \epsilon \sqrt{\Delta}}{2} \\
x^3 + y^3 - \cfrac{[x + y + 17 - \epsilon \sqrt{\Delta}]^3}{8} - \cfrac{[x + y + 17 + \epsilon \sqrt{\Delta}]^3}{8} &= - 503 \\
8 x^3 + 8 y^3 - (x^3 + y^3 + 4913 &- \epsilon \Delta \cancel{\sqrt{\Delta}} \\
+ 3x^2 y + 3x y^2 + 51 x^2 + 867x + 51 y^2 + 867y &- 3x^2 \epsilon \cancel{\sqrt{\Delta}} \\
+ 3x \Delta - 3y^2 \epsilon \cancel{\sqrt{\Delta}} + 3y \Delta - 867 \epsilon \cancel{\sqrt{\Delta}} &+ 51 \Delta \\
+ 102xy - 6 xy \epsilon \cancel{\sqrt{\Delta}} - 102x \epsilon \cancel{\sqrt{\Delta}} &- 102y \epsilon \cancel{\sqrt{\Delta}} \\
+ x^3 + y^3 + 4913 &+ \epsilon \Delta \cancel{\sqrt{\Delta}} \\
+ 3x^2 y + 3x y^2 + 51 x^2 + 867x + 51 y^2 + 867y &+ 3x^2 \epsilon \cancel{\sqrt{\Delta}} \\
+ 3x \Delta + 3y^2 \epsilon \cancel{\sqrt{\Delta}} + 3y \Delta + 867 \epsilon \cancel{\sqrt{\Delta}} &+ 51 \Delta \\
+ 102xy + 6 xy \epsilon \cancel{\sqrt{\Delta}} + 102x \epsilon \cancel{\sqrt{\Delta}} + 102y \epsilon \cancel{\sqrt{\Delta}}) &= -4024 \\
6 x^3 + 6 y^3 - (9826 + 6x^2 y + 6x y^2 + 102 x^2 + 1734x + 102 y^2 + 1734y + 204xy) &= - 4024 + \Delta (6x + 6y + 102) \\
&= - 4024 -18 x^3 -18 xy^2 - 12x^2y - 204x^2 - 204xy - 690x \\
&-18 x^2y -18 y^3 - 12xy^2 - 204xy - 204y^2 - 690y \\
&-306 x^2 -306 y^2 - 204xy - 3468x - 3468y - 11730 \\
24 y^3 + y^2 (24 x + 408) + y (24x^2 + 408x + 2424) + 24 x^3 + 408x^2 + 2424x + 5928 &= 0 \\
y^3 + y^2 (x + 17) + y (x^2 + 17 x + 101) + x^3 + 17x^2 + 101x + 247 &= 0 \\
y^3 + b y^2 + c y + d &= 0 \\
y &= t - \cfrac{b}{3} \\
t^3 + pt + q &= 0 \\
b &= x + 17 \\
c &= x^2 + 17 x + 101 \\
d &= x^3 + 17x^2 + 101x + 247
\end{align}

\begin{align}
3p &= 3c - b^2 = 3(x^2 + 17 x + 101) - (x + 17)^2 = 2x^2 + 17x + 14 \\
27q &= 2 b^3 - 9 bc + 27 d = 2 (x + 17)^3 - 9 (x + 17)(x^2 + 17 x + 101) + 27(x^3 + 17x^2 + 101x + 247) \\
27q &= 2x^3 + 102x^2 + 1734 x + 9826 - 9 x^3 - 306 x^2 - 3510 x - 15453 + 27 x^3 + 459 x^2 + 2727 x + 6669 \\
27q &= 20x^3 + 255x^2 + 951 x + 1042 \\
\delta &= \cfrac{q^2}{4} + \cfrac{p^3}{27} = \cfrac{(20x^3 + 255x^2 + 951 x + 1042)^2 + 4 (2x^2 + 17x + 14)^3}{2916} = \cfrac{f(x)}{2916} \\
\omega_1 &= -\cfrac{q}{2} + \sqrt{\delta} = \cfrac{-20x^3 - 255x^2 - 951 x - 1042 + \sqrt{f}}{54} \\
\omega_2 &= \cfrac{-20x^3 - 255x^2 - 951 x - 1042 - \sqrt{f}}{54} \\
u &= \exp i(120^o) \Rightarrow \overline{u} = \exp i(240^o) \\
\zeta_1 &= \text{principal value of } \sqrt[3]{\omega_1} \\
\zeta_2 &= \text{principal value of } \sqrt[3]{\omega_2} \\
y_1 &= \zeta_1 + \zeta_2 - \cfrac{x + 17}{3} \\
y_2 &= u \cdot \zeta_1 + \overline{u} \cdot \zeta_2 - \cfrac{x + 17}{3} \\
y_3 &= \overline{u} \cdot \zeta_1 + u \cdot \zeta_2 - \cfrac{x + 17}{3}
\end{align}

\vspace{3mm}

\textbf{Theorem: }If we know two roots $x, y$ of the equation $x^4 - Sx^3 + Qx^2 - Tx + P = 0$, then the other two roots are:

\begin{align}
z &= \cfrac{- x - y - 17 \pm \sqrt{\Delta}}{2} \\
\text{Where } \Delta &= -3 x^2 -3 y^2 - 2xy - 34x - 34y - 115
\end{align}

\vspace{3mm}

quod erat demonstrandum | Out of charity, there is no salvation at all.

\vspace{3mm}

I did not suppose that, but I was playing with $(x + 2)(x + 3)(x + 5)(x + 7) = 0$.

\vspace{3mm}

\textbf{Exercise: } Now, do the same without numbers.

\vspace{3mm}

Vinicius Claudino FERRAZ, $01^{\text{st}}$/May/2019, Release 1.1

\end{document}

\documentclass[12pt,a4paper]{article}
\usepackage{amsmath}
\usepackage{amssymb} %mathbb
\usepackage{graphicx}
\usepackage{hyperref}
\usepackage[top=1.0cm,bottom=1.3cm,left=1.0cm,right=1.0cm]{geometry}

\begin{document}

\Large

Liapunov

\normalsize

\vspace{6mm}

Procur\'avamos uma fun\c{c}\~ao de Liapunov para o degrau unit\'ario $y = u(t)$. O que encontramos foi:

\begin{align}
y^* &= y(\infty) = 1 \\
L(y) &= (y - y^*)^2 \ge 0 \\
\cfrac{dL}{dt} &= 2 (y - y^*) y' < 0
\end{align}

\vspace{6mm}

Procur\'avamos uma fun\c{c}\~ao de Liapunov para a corrente alternada $y = \cos t$. O que encontramos foi:

\begin{align}
t \ge 0\,;\, y^* &= y(\infty) \\
L(t) &= - \int_0^t \cfrac{\cos \tau}{\tau} \,\mathrm{d}\tau - \cfrac{t^3}{3} - 300t \\
\cfrac{dL}{dt} &= - \cfrac{\cos t}{t} - t^2 - 300 \\
t \le 0\,;\, y^* &= y(-\infty) \\
L(t) &= - \int_t^0 \cfrac{\cos \tau}{\tau} \,\mathrm{d}\tau - 1000t \ge 0 \\
\cfrac{dL}{dt} &= - \cfrac{\cos t}{t} - 1000 < 0
\end{align}

\vspace{3mm}

Exerc\'icio 1: $y = \cos \omega t$.

\vspace{3mm}

Exerc\'icio 2: $y = \sin \omega t$.

\vspace{3mm}

Exerc\'icio 3: $y = \cfrac{a_0}{2} + \sum a_n \cos \omega_n t + \sum b_n \sin \omega_n t$. Soma finita, depois s\'erie.

\vspace{6mm}

Fora da caridade n\~ao h\'a salva\c{c}\~ao. Com caridade, h\'a evolu\c{c}\~ao.

\vspace{6mm}

Vers\~ao 0.1 de 6/jan/2019 por Vinicius Claudino Ferraz.

\end{document}

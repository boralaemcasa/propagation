\documentclass[12pt]{article}
\usepackage[portuguese]{babel}
\usepackage[utf8x]{inputenc}
\usepackage{indentfirst}
\usepackage{amsmath}
\numberwithin{table}{section}
\usepackage{graphicx}
\usepackage[colorinlistoftodos]{todonotes}
\usepackage{acronym}
\usepackage{multirow}
\usepackage{array}
\usepackage{float}
\usepackage{units}
\usepackage{mathtools}
\usepackage{amsthm}
\usepackage{amsmath}
\usepackage{amssymb}
\usepackage{graphicx}
\usepackage{adjustbox}
\usepackage{setspace}
\usepackage{wasysym}
\usepackage{esint}
\usepackage{cancel}
\usepackage{tikz-cd}
\usepackage[colorlinks = true,
            linkcolor = red, % acrônimos
            urlcolor  = blue,
            citecolor = coolblack,
            anchorcolor = yellow]{hyperref}
%\usepackage[numbers]{natbib}
\usepackage[abbrvnat]{natbib}
\newcommand{\isParallel}{\mathbin{\!/\mkern-5mu/\!}}
\usepackage{geometry}
\bibliographystyle{abbrvnat}



\geometry{verbose,a4paper,tmargin=30mm,bmargin=20mm,lmargin=30mm,rmargin=20mm}
\begin{document}
\definecolor{coolblack}{rgb}{0.0, 0.18, 0.39}


% PDF informations
                  
\setstretch{1.5}
\newtheorem{definition}{Definição}
\newcommand\LG{\mathord{\mathcal{L}}}
\newcommand\LM{\mathord{\mathcal{L}}_m}


\begin{titlepage}

\newcommand{\HRule}{\rule{\linewidth}{0.4mm}} % Defines a new command for the horizontal lines, change thickness here

\center % Center everything on the page
 
%----------------------------------------------------------------------------------------
%	HEADING SECTIONS
%----------------------------------------------------------------------------------------

\textsc{\normalsize Universidade Federal de Minas Gerais}\\[0.05cm]
\begin{figure}[!h]
   \centering{
   \includegraphics[scale=0.65]{ufmglogo.png} \\[0.5cm]
   }
 \end{figure} 


%----------------------------------------------------------------------------------------
%	TITLE SECTION
%----------------------------------------------------------------------------------------

\HRule \\[0.4cm]
{ \huge \bfseries Projective Geometry Again}\\[0.4cm] 
\HRule \\[0.7cm]
 
%----------------------------------------------------------------------------------------
%	AUTHOR SECTION
%----------------------------------------------------------------------------------------
{\normalsize TEAM:
\begin{itemize}
    \item[] Vinicius Claudino Ferraz PPGEE-UFMG.
\end{itemize}
}


$\,$\\[2.0cm]
%----------------------------------------------------------------------------------------
%	DATE SECTION
%----------------------------------------------------------------------------------------

%{\large \today} % Date, change the \today to a set date if you want to be precise

%----------------------------------------------------------------------------------------
%	LOGO SECTION
%----------------------------------------------------------------------------------------

%\includegraphics[width=0.5\textwidth]{ufmgLogo.jpg}
%\includegraphics{ufmglogo.png}\\[1cm] %
 
%----------------------------------------------------------------------------------------
\vfill % Fill the rest of the page with whitespace

\end{titlepage}

%\begin{abstract}
%Your abstract.
%\end{abstract}

% List of Acronyms Definition



\section{The Conic Theorem}

\begin{align}
A &= X_1 X_5 \cap X_2 X_6 \\
B &= X_1 X_4 \cap X_3 X_6 \\
C &= X_2 X_4 \cap X_3 X_5 \in AB \label{eq3} \\
A &= (0,0) \\
B &= (0,a) \\
X_1 &= (x_1, bx_1) \\
X_5 &= (x_5, bx_5) \\
X_2 &= (x_2, cx_2) \\
X_6 &= (x_6, cx_6) \\
X_4 &= (x_4, d(x_4 - a)) \\
bx_1 &= d(x_1 - a) \Rightarrow d = \cfrac{b x_1}{x_1 - a} \\
X_3 &= (x_3, e(x_3 - a)) \\
cx_6 &= e(x_6 - a) \Rightarrow e = \cfrac{cx_6}{x_6 - a} \\
\alpha x_1^2 + \alpha (bx_1)^2 &= 1 \Rightarrow x_1^2 = \cfrac{1}{\alpha + \alpha b^2} = \lambda \\
\alpha x_2^2 + \alpha (cx_2)^2 &= 1 \Rightarrow  x_2^2 =\cfrac{1}{\alpha + \alpha c^2} = \xi\\  
\alpha x_5^2 + \alpha (bx_5)^2 &= 1\Rightarrow x_5^2 = \cfrac{1}{\alpha + \alpha b^2} = \lambda \therefore x_5 = \epsilon_1 x_1 \\
\alpha x_6^2 + \alpha (cx_6)^2 &= 1 \Rightarrow  x_6^2 =\cfrac{1}{\alpha + \alpha c^2} = \xi \therefore x_2 = \epsilon_6 x_6\\
\alpha x_3^2 + \alpha [e(x_3-a)]^2 &= 1 \Rightarrow e^2 = \cfrac{1 - \alpha x_3^2 }{\alpha (x_3-a)^2} \label{eq18}\\
\alpha x_4^2 + \alpha [d(x_4-a)]^2 &= 1 \Rightarrow d^2 = \cfrac{1 - \alpha x_4^2 }{\alpha (x_4-a)^2} \label{eq19}
\end{align}

\newpage

Therefore:

\begin{align}
\cfrac{-y_3 x_5 + y_5 x_3}{x_3 - x_5} &= \cfrac{-y_2 x_4 + y_4 x_2}{x_2 - x_4} \label{eq20} \\
\cfrac{-\cfrac{cx_6}{x_6 - a}(x_3-a) x_5 + b x_5 x_3}{x_3 - x_5} &= \cfrac{-cx_2 x_4 + \cfrac{bx_1}{x_1 - a}(x_4-a) x_2}{x_2 - x_4} 
\end{align}

Below, we try to reduce that to $0x = 0$:

The first intention is to eliminate $x_1$ and $x_6$.

\begin{align}
\zeta &= b x_3 x_5 (x_2 - x_4) + cx_2 x_4(x_3 - x_5) = \zeta_{51} x_5 + \zeta_{50} \\
\zeta_{51} &= bx_2x_3 - b x_3x_4 - cx_2 x_4 \\
\zeta_{50} &= cx_2 x_3 x_4 \\
\eta &= (x_4-a) (x_3 - x_5) = \eta_{51} x_5 + \eta_{50} \\
\eta_{51} &= a - x_4 \\
\eta_{50} &= -x_3 (a - x_4)\\
\kappa &= (x_3-a)^2 (x_2 - x_4)^2 =\kappa_{21} x_2 + \kappa_{20} \\
\kappa_{21} &= - (x_3-a)^2 x_4\\
\kappa_{20} &= (x_3-a)^2 x_4^2 + \xi (x_3-a)^2\\
\eta^2 &= (\eta_{51} x_5 + \eta_{50})^2 = \hat A x_5 + \hat B \\
\hat A &= 2\eta_{51}\eta_{50} \\
\hat B &= \lambda \eta_{51}^2 + \eta_{50}^2\\
\zeta^2 &= (\zeta_{51} x_5 + \zeta_{50})^2  = \hat C x_5 + \hat D\\
\hat C &= 2\zeta_{51}\zeta_{50} \\
\hat D &= \lambda \zeta_{51}^2 + \zeta_{50}^2 \\
\mu &= 4\alpha (x_3 - a)^4 (x_4 - a)^4(1 - \alpha x_4^2 ) \\
\nu &= (1 - \alpha x_4^2)(x_3-a)^2(x_4-a)\\
\pi &= \alpha (x_3-a)^2(x_4-a)^3
\end{align}
\begin{align}
\rho &= (1 - \alpha x_3^2 )(x_4-a)^3\\
\kappa x_5^2 \underbrace{e^2} &= (\zeta - \eta x_2 \underbrace{d})^2 \\ 
\kappa x_5^2 \cfrac{1 - \alpha x_3^2 }{\alpha (x_3-a)^2} &= \eta^2 x_2^2 \cfrac{1 - \alpha x_4^2 }{\alpha (x_4-a)^2} - 2 \zeta \eta x_2 \underbrace{d} + \zeta^2 \\
 \mu \zeta^2 \eta^2 x_2^2  &= (\eta^2 x_2^2 \nu  + \pi \zeta^2 - \kappa x_5^2 \rho)^2 
\end{align}

They are already eliminated: $(d,e)$. Let us eliminate $x_5$ and, only partially, $x_2$.

\begin{align}
\xi \mu (\hat A x_5 + \hat B)(\hat C x_5 + \hat D)  &= \left[ \xi\nu (\hat A x_5 + \hat B)  + \pi (\hat C x_5 + \hat D) - \lambda \rho \kappa \right]^2 \\
E x_5 + F  &= ( G x_5 + H )^2 = 2GH x_5 + \lambda G^2 + H^2 \\
\lambda (E - 2GH)^2  &= \left(\lambda G^2 + H^2 - F\right)^2 \\
E &= \xi \mu(\hat A\hat D + \hat B\hat C)\\
F &= \lambda \xi \mu [\hat A\hat C + \hat B\hat D(\alpha + \alpha b^2)] \\
G &= \xi \nu \hat A + \pi \hat C \\
H &= \xi \nu \hat B + \pi \hat D - \lambda \rho \kappa
\end{align}

Let us eliminate $x_2$ in $(\zeta,\kappa, \hat C, \hat D)$.

\footnotesize

\begin{align}
&\lambda \left[\xi \mu(\hat A\hat D + \hat B\hat C) - 2\left(\xi \nu \hat A + \pi \hat C\right) \left(\xi \nu \hat B + \pi \hat D - \lambda \rho \kappa\right) \right]^2 = \nonumber\\
&= \left\{\lambda \left(\xi \nu \hat A + \pi \hat C\right)^2 + \left(\xi \nu \hat B   + \pi \hat D - \lambda \rho \kappa\right) ^2 - \lambda \xi \mu [\hat A\hat C + \alpha\hat B\hat D(1 + b^2)]\right\}^2 
\end{align}

\normalsize

\begin{align}
\kappa(x_2) &= \kappa_{21} x_2 + \kappa_{20}\\
\hat C(x_2) &= 2(bx_2x_3 - b x_3x_4 - cx_2 x_4 )cx_2 x_3 x_4 = Ix_2 + J \\
\hat D(x_2) &= \lambda (bx_2x_3 - b x_3x_4 - cx_2 x_4 )^2 + c^2x_2^2 x_3^2 x_4^2 = K x_2 + L \\
I &= - 2b cx_3^2x_4^2\\
J &= \xi (2bcx_3^2 x_4 - 2c^2 x_3 x_4^2)\\
K &= \lambda (2bc x_3 x_4^2 - 2b^2x_3^2 x_4 )\end{align}
\begin{align}
L &= \lambda b^2 x_3^2 x_4^2 + \xi c^2x_3^2 x_4^2+\lambda \xi (bx_3 - c x_4)^2\\
M &= \xi \mu(\hat AK  + \hat BI ) \\
N &= \xi \mu(\hat AL + \hat BJ) \\
P &= \pi I \\
Q &= \xi \nu \hat A + \pi J \\
R &= \pi K - \lambda \rho \kappa_{21}  \\
S &=\xi \nu \hat B + \pi L - \lambda \rho \kappa_{20} \\ 
T &= - \lambda \xi \mu [\hat AI  + \alpha\hat BK  (1 + b^2)]\\
U &= - \lambda\xi \mu [\hat A J + \alpha\hat B L(1 + b^2)]\\
\lambda [Mx_2 + N -2 (Px_2 + Q)(Rx_2 &+ S) ]^2  = [\lambda\left(Px_2 + Q\right)^2 + \left(Rx_2 + S\right) ^2 + T x_2 + U]^2 \\
V &= M-2PS -2 QR\\
W &= N -2\xi PR-2 QS \\
Y &= 2\lambda VW\\
Z &= \lambda\xi V^2 + \lambda W^2 \\
\hat E &= 2\lambda PQ + 2RS + T\\
\hat F &= \lambda\xi P^2 +\xi R^2 + \lambda Q^2 + S^2 + U \\
Yx_2 + Z &= (\hat E x_2 + \hat F)^2 =  \xi \hat E^2 + \hat F^2 + 2\hat E\hat F x_2 \\
\xi (Y - 2\hat E\hat F)^2 &= \left(\xi \hat E^2 + \hat F^2 - Z\right)^2 \label{eq75}
\end{align}

We have the Equation (\ref{eq75}) in $\alpha, x_3, x_4, a, b, c$.

The main question is: is it an identity? Let us express it as a function of $(x_3,x_4) \equiv (x,y,z=x-a,w=y-a)$.

\begin{align}
V &= \xi \mu(\hat AK  + \hat BI )-2\pi I(\xi \nu \hat B + \pi L - \lambda \rho \kappa_{20}) \nonumber\\
& -2\xi \nu \pi \hat A K +2 \lambda \xi \nu \rho \kappa_{21} \hat A -2  \pi^2 J K +2 \lambda \pi \rho \kappa_{21} J\\
W &= \xi \mu(\hat AL + \hat BJ) -2\xi \pi I(\pi K - \lambda \rho \kappa_{21})\nonumber\\
&-2\xi^2 \nu^2 \hat A \hat B -2 \xi \nu \pi \hat A L +2 \lambda \xi \nu \rho \kappa_{20} \hat A-2 \xi \nu \pi \hat B J -2 \pi^2 J L +2 \lambda \pi \rho \kappa_{20} J
\end{align}

Let us substitute from $L$ and above, until $\eta$.

\begin{align}
\nu^2 &= z^4 ( w - \alpha y^3 + \bar k y^2)^2\nonumber \\
&= z^4 (w^2 + \alpha^2 y^6 + \bar k^2 y^4 - 2  \alpha y^3 w + 2 \bar k y^2 w - 2 \alpha \bar k y^5)
\end{align}

At this moment, we realize that we are next to a polynomial of a single variable, say $x$. Let us eliminate $z$.

For all conic section in $x\hat O y$, there is a circle in $x\hat O z$. The theorem is simplified because the inverse projection of 3 collinear points are 3 collinear points too.

\begin{align}
V &= x(x-a)^4 yw^2(\bar o_0 + \bar o_1 x + o_2 x^2 + o_3 x^3) \\
W &= x(x-a)^4(\bar p_0 + \bar p_1 x + \bar p_2 x^2 + \bar p_3 x^3)\\
Y &= 2\lambda x^2(x-a)^8 yw^2(\bar o_0 + \bar o_1 x + o_2 x^2 + o_3 x^3)(\bar p_0 + \bar p_1 x + \bar p_2 x^2 + \bar p_3 x^3)
\end{align}

If we did not want to give it all up, we ``would'' have:

\begin{align}
Z &= x^2(x-a)^8 [\lambda\xi  y^2w^4(\bar o_0 + \bar o_1 x + o_2 x^2 + o_3 x^3)^2 + \lambda (\bar p_0 + \bar p_1 x + \bar p_2 x^2 + \bar p_3 x^3)^2] \nonumber\\
&= \cdots
\end{align}

\begin{align}
S(\nu)&=\xi (z^2 w - \alpha y^3 z^2 + \bar k y^2 z^2) (\lambda w^2 + x^2 w^2) + \alpha z^2w^3 (\bar a x^2 y^2 + \bar b x^2 + \bar c y^2 + \bar d xy) \nonumber\\
&- \lambda (w^3 - \alpha x^2 w^3) (y^2 z^2 + \xi z^2)\\
&= \cdots
\end{align}

\begin{align}
Q(\nu)&=-2\xi xw^2 (z^2 w - \alpha y^3 z^2 + \bar k y^2 z^2) + \alpha z^2w^3 (\bar g xy^2 + \bar h x^2 y)\\
\hat E &= 2\lambda \pi I\overbrace{(\xi {\color{red}\nu} \hat A + \pi J)}^{Q} + 2(\pi K - \lambda \rho \kappa_{21})\overbrace{(\xi {\color{red}\nu} \hat B + \pi L - \lambda \rho \kappa_{20})}^{S}  \nonumber\\
&- \xi \mu (\lambda \hat AI  + \hat BK  )\\
\hat F &= \lambda\xi \pi^2 I^2 +\xi (\pi K - \lambda \rho \kappa_{21})^2 + \lambda (\xi {\color{red}\nu} \hat A + \pi J)^2 + (\xi {\color{red}\nu} \hat B + \pi L - \lambda \rho \kappa_{20})^2 \nonumber\\
&- \xi \mu (\lambda \hat A J + \hat B L) \\
L &= \bar a x^2 y^2 + \bar b x^2 + \bar c y^2 + \bar d xy \,;\,\bar a = \lambda b^2 + \xi c^2\,;\,\bar b = \lambda \xi b^2\,;\,\bar c = \lambda \xi c^2\,;\,\bar d = -2 \lambda \xi bc \\
K &= \bar e xy^2 + \bar f x^2 y \,;\,\bar e = 2 \lambda bc \,;\,\bar f = -2\lambda b^2 \\
J &= \bar g xy^2 + \bar h x^2 y \,;\,\bar g = -2\xi c^2 \,;\,\bar h = 2\xi bc \\
I &= \bar \iota x^2 y^2 \,;\, \bar \iota = -2bc
\end{align}

\begin{align}
(Y - 2\hat E \hat F)^2 &= \cdots \\
\hat E^2 &= \cdots \\
\hat F^2 &= \cdots 
\end{align}

\begin{align}
\kappa_{21} &= -yz^2 \\
\kappa_{20} &= y^2 z^2 + \xi z^2\\
\rho &= w^3 - \alpha x^2 w^3\\
\pi &= \alpha z^2w^3\\
\nu &= z^2 w - \alpha y^3 z^2 + \bar k y^2 z^2\,;\,\bar k = \alpha \cdot a\\
\mu &= \bar \ell z^4 w^4 + \bar m z^4 w^6 + \bar n z^4 w^5 \,;\,\bar \ell = 4\alpha   -4\alpha^2 a^2 \,;\,\bar m = -4\alpha^2 \,;\,\bar n = - 8\alpha^2 a \\
\hat B &= \lambda w^2 + x^2 w^2\\
\hat A &= -2xw^2 \\
\bar o_0 &= 4\lambda \xi w^3(  w - \alpha y^3  + \bar k y^2 ) -2 \lambda \alpha w^4 \bar g y^2 \\ 
\bar o_1 &= -2\bar e y (\xi\bar \ell w^4 + \xi\bar m w^6 + \xi\bar n w^5)
+ \lambda \bar\iota y(\xi\bar \ell  w^4 + \xi\bar m  w^6 + \xi\bar n  w^5)\nonumber\\
&-2\lambda\alpha \bar \iota yw^2 ( \xi  w - \xi \alpha y^3  + \xi \bar k y^2 ) - 2\alpha^2 \bar \iota y w^4 \bar c y^2 +2\lambda\alpha \bar \iota y  w^4 (y^2  + \xi )\nonumber\\
&+4\alpha\xi \bar e yw^3(  w - \alpha y^3  + \bar k y^2 ) -2\alpha^2 \bar g y w^4\bar e y^2 -2 \lambda \alpha w^4 \bar h y \\
\bar o_2 &= -2 \bar f  (\xi\bar \ell  w^4 + \xi\bar m  w^6 + \xi\bar n  w^5)- 2\alpha^2 \bar \iota y w^4 \bar d y +4\alpha\xi \bar f  w^3(  w - \alpha y^3  + \bar k y^2 )  \nonumber\\
&- 4\lambda \xi \alpha  w^3(  w - \alpha y^3  + \bar k y^2) -2\alpha^2 \bar g y w^4 \bar f  y -2\alpha^2 \bar h w^4\bar e y^2 + 2 \lambda \alpha^2 w^4 \bar g y^2 \\
\bar o_3 &= \bar\iota y(\xi\bar \ell  w^4 + \xi\bar m  w^6 + \xi\bar n  w^5)-2\alpha \bar \iota y w^3 ( \xi  w - \xi \alpha y^3  + \xi \bar k y^2 )  \nonumber\\
&- 2\alpha^2 \bar \iota \lambda y  w^4 (y^2  + \xi ) 
-2\alpha^2 \bar h w^4 \bar f  y
+ 2 \lambda \alpha^2 w^4 \bar h  y- 2\alpha^2 \bar \iota y w^4  (\bar a y^2 + \bar b )\\
\bar p_0 &=- 2\xi w^2\bar c y^2 (\bar \ell  w^4 + \bar m  w^6 + \bar n  w^5) + \xi \lambda w^2\bar g y^2 (\bar \ell  w^4 + \bar m  w^6 + \bar n  w^5)\nonumber\\
&+4\lambda\xi^2 w^4 (  w^2 + \alpha^2 y^6  + \bar k^2 y^4  - 2  \alpha y^3  w + 2 \bar k y^2 w - 2 \alpha y^5  \bar k) +  4 \xi\alpha w^6  \bar c y^2  \nonumber\\
&+ 4 \xi\alpha w^5\bar k y^2 \bar c y^2 +  w (-4 \lambda \xi w^5 y^2  -4 \lambda \xi w^5 \xi) - \alpha y^3  (-4 \lambda \xi w^5 y^2  -4 \lambda \xi w^5 \xi) \nonumber\\
&+ \bar k y^2  (-4 \lambda \xi w^5 y^2  -4 \lambda \xi w^5 \xi) -2 \xi\alpha w^3 \bar g y^2\lambda w^2 (  w - \alpha y^3  + \bar k y^2 )   \nonumber\\
&+ 2 \lambda \alpha w^6 (\bar g y^4 + \xi \bar g y^2) -2 \alpha^2 w^6\bar g y^2 \bar c y^2 - 4 \xi\alpha w^5\alpha y^3 \bar c y^2
\end{align}

\begin{align}
\bar p_1 &= - 2\xi w^2\bar d y (\bar \ell  w^4 + \bar m  w^6 + \bar n  w^5) + \xi \lambda w^2\bar h  y(\bar \ell  w^4 + \bar m  w^6 + \bar n  w^5)- 2\xi \alpha w^3 \bar \iota \lambda y^3 w^3  \nonumber\\
&+  4 \xi\alpha w^6  \bar d y - 4 \xi\alpha w^5\alpha y^3 \bar d y + 4 \xi\alpha w^5\bar k y^2 \bar d y -2 \lambda\xi\alpha \bar h  y w^5  (  w - \alpha y^3  + \bar k y^2 ) \nonumber\\
&-2 \alpha^2 w^6\bar g y^2 \bar d y 
-2 \alpha^2 w^6\bar h y \bar c y^2
+ 2 \lambda \alpha w^6 (\bar h  y^3 + \xi \bar h  y) \\
\bar p_2 &= -2\xi w^2\bar a  y^2(\bar \ell  w^4 + \bar m  w^6 + \bar n  w^5) -2\xi \alpha \bar \iota y^2\alpha w^6 \bar e y^2 - 2\xi w^2\bar b (\bar \ell  w^4 + \bar m  w^6 + \bar n  w^5)  \nonumber\\
&+4\xi^2 w^4 (  w^2 + \alpha^2 y^6  + \bar k^2 y^4  - 2  \alpha y^3  w + 2 \bar k y^2 w - 2 \alpha y^5  \bar k) +  4 \xi\alpha w^6  (\bar a  y^2 + \bar b ) \nonumber\\
&- 4 \xi\alpha w^5\alpha y^3 (\bar a  y^2 + \bar b ) + 4 \xi\alpha w^5\bar k y^2 (\bar a  y^2 + \bar b ) +  w (4 \lambda \xi \alpha  w^5 y^2  + 4 \lambda \xi \alpha  w^5 \xi ) \nonumber\\
&- \alpha y^3  (4 \lambda \xi \alpha  w^5 y^2  + 4 \lambda \xi \alpha  w^5 \xi ) + \bar k y^2  (4 \lambda \xi \alpha  w^5 y^2  + 4 \lambda \xi \alpha  w^5 \xi )  \nonumber\\
&-2 \alpha^2 z^4w^6\bar g y^2 (\bar a  y^2 + \bar b ) -2 \alpha^2 z^4w^6\bar h y \bar d y - 2 \lambda \alpha^2 w^6 (\bar g y^4 + \xi \bar g y^2)\nonumber\\
&-2 \xi\alpha w^3 \bar g y^2  w^2 (  w - \alpha y^3  + \bar k y^2 )+ \xi  w^2\bar g y^2 (\bar \ell  w^4 + \bar m  w^6 + \bar n  w^5) \\
\bar p_3 &= \xi w^2\bar h  y(\bar \ell  w^4 + \bar m  w^6 + \bar n  w^5)
-2\xi \alpha^2 \bar \iota y^2 w^6 \bar f  y 
- 2\xi \alpha w^3 \bar \iota \lambda y^3 \alpha  w^3 \nonumber\\
&-2 \xi\alpha \bar h  y w^5 (  w - \alpha y^3  + \bar k y^2 ) 
-2 \alpha^2 w^6\bar h y (\bar a  y^2 + \bar b ) - 2 \lambda \alpha^2 w^6 (\bar h  y^3 + \xi \bar h  y)
\end{align}

Below, let us explain the Equation (\ref{eq20}), expliciting $C$ on Equation (\ref{eq3}).

\begin{align}
    \begin{pmatrix}
        x_2 & 1 \\ x_4 & 1
    \end{pmatrix}  
    \begin{pmatrix}
        a \\ b
    \end{pmatrix} &=
    \begin{pmatrix}
        y_2 \\ y_4
    \end{pmatrix} \Rightarrow 
    \begin{pmatrix}
        a \\ b
    \end{pmatrix} = \cfrac{1}{x_2 - x_4}
    \begin{pmatrix}
        1 & -1 \\ -x_4 & x_2
    \end{pmatrix}  
    \begin{pmatrix}
        y_2 \\ y_4
    \end{pmatrix} \\
    \begin{pmatrix}
        x_3 & 1 \\ x_5 & 1
    \end{pmatrix}  
    \begin{pmatrix}
        c \\ d
    \end{pmatrix} &=
    \begin{pmatrix}
        y_3 \\ y_5
    \end{pmatrix} \Rightarrow 
    \begin{pmatrix}
        c \\ d
    \end{pmatrix} = \cfrac{1}{x_3 - x_5}
    \begin{pmatrix}
        1 & -1 \\ -x_5 & x_3
    \end{pmatrix}  
    \begin{pmatrix}
        y_3 \\ y_5
    \end{pmatrix} \\
    ax + b &= cx + d \Rightarrow x = \cfrac{d-b}{a-c} = 0 \therefore d = b
\end{align}

\newpage

\section{The Converse Conic Theorem}

\begin{align}
    A &= X_1 X_5 \cap X_2 X_6 \\
    B &= X_1 X_4 \cap X_3 X_6 \\
    C &= X_2 X_4 \cap X_3 X_5 \in AB \\
    A &= (0,0) \\
    B &= (0,a) \\
    X_1 &= (x_1, bx_1) \\
    X_5 &= (x_5, bx_5) \\
    X_2 &= (x_2, cx_2) \\
    X_6 &= (x_6, cx_6) \\
    X_4 &= (x_4, d(x_4 - a) \\
    bx_1 &= d(x_1 - a) \\
    X_3 &= e(x_3 - a) \\
    cx_6 &= e(x_6 - a) \\
    \cfrac{-y_3 x_5 + y_5 x_3}{x_3 - x_5} &= \cfrac{-y_2 x_4 + y_4 x_2}{x_2 - x_4} \\
    \text{Therefore, } \exists &\lambda\in\mathbb{R}^6\text{, such that:} \nonumber \\
    \begin{pmatrix}
        x_1^2 & y_1^2 & x_1 y_1 & x_1 & y_1 & 1 \\
        \vdots & \vdots & \vdots & \vdots & \vdots & \vdots \\
        x_6^2 & y_6^2 & x_6 y_6 & x_6 & y_6 & 1 
    \end{pmatrix}\cdot \lambda &= 0
\end{align}

\section{From Circle to Parabola}

\begin{align}
S': x'^2 + z'^2 &= R^2, y = 0 \\
v &= (\mathbb{R}, 0, R) \\
P &= (0, -p, R) \\
A' &= (x', 0, z') \in x\hat Oz \\
A'P: (x,y,z) &= (x' + \lambda x', \lambda p, z' + \lambda (z' - R) \\
B &= \pi(A')\,;\,\pi: x\hat Oz \to x\hat Oy \\
&= A'P \cap x\hat O y = (x_b, y_b, z_b = 0) \\
x_b &= \cfrac{Rx'}{R-z'}\,;\,y_b = \cfrac{pz'}{R-z'} \\
(R - z') y_b &= pz' \Rightarrow z' = \cfrac{Ry_b}{p+y_b} \\
x' &= \cfrac{R-z'}{R}\cdot x_b = \cfrac{px_b}{p+y_b} \\
\pi(S'): p^2 x_b^2 + \cancel{R^2 y_b^2} &= R^2 (p+y_b)^2 = p^2 R^2 + 2pR^2y_b + \cancel{R^2 y_b^2}\,;\,z_b=0 \\
\therefore y_b &= \cfrac{px_b^2 - p R^2}{2R^2}
\end{align}

That is a parabola that intercepts $\hat O y$ at $V = \left(0, -\cfrac{p}{2}, 0\right)$, and intercepts $\hat Ox$ at $(\pm R, 0, 0)$.

\subsection{From Circle to Hiperbola}

\begin{align}
S'': x'^2 &+ z'^2 = (R+q)^2, y = 0 \\
\pi(S''): p^2 x_b^2 + R^2 y_b^2 &= (R+q)^2 (p^2 + 2py_b +y_b^2)\,;\,z_b=0 \\
p^2x_b^2 &= (2Rq + q^2) y_b^2 + 2p(R+q)^2 y_b + p^2(R+q)^2\\
p^2(R+q)^4 - p^2q(R+q)^2(2R + q) &= \left[(2R + q)q y_b + p(R+q)^2\right]^2-(2R+q)p^2qx_b^2 \\
\cfrac{1}{A^2} \left[y_b+\cfrac{p(R+q)^2}{q(2R + q)}\right]^2-\cfrac{x_b^2}{B^2} &= 1\,;\,A = \cfrac{pR(R+q)}{q(2R+q)}\,;\,B = \cfrac{R(R+q)}{\sqrt{q(2R+q)}}
\end{align}

Whenever $x_b = 0$, we have $y_b + C(q) = \pm A$. Therefore, that's a vertical hyperbola.

We want to prove that while $q\to \infty$, the projection is a degenerated hyperbola.

\begin{align}
\cfrac{(y + c)^2}{a^2} - \cfrac{x^2}{b^2} &= 1\Rightarrow y = -c \pm a\sqrt{1 + \cfrac{x^2}{b^2}} \Rightarrow y' = \pm  \cfrac{ax}{\sqrt{1 + \cfrac{x^2}{b^2}} } \,\xrightarrow{x \to \infty}\, \pm \cfrac{a}{b} \\
y &= \pm \cfrac{ax}{b} -c\because y(0) = - c \\
\cfrac{A}{B} &= \cfrac{p}{\sqrt{q(2R+q)}}\,\xrightarrow{q \to \infty}\,0
\end{align}

\subsection{From Circle to Ellipsis}

\begin{align}
S''': x'^2 + z'^2 &= (R-q)^2, y = 0 \\
\pi(S'''): \cfrac{1}{A^2} \left[y_b-\cfrac{p(R-q)^2}{q(2R - q)}\right]^2+\cfrac{x_b^2}{B^2} &= 1 \,;\,A = \cfrac{pR(R-q)}{q(2R-q)}\,;\,B = \cfrac{R(R-q)}{\sqrt{q(2R-q)}}
\end{align}

Whenever $y_b + C(-q) = 0$, we have $x_b = \pm B$.

\subsection{Invariant Straight Lines}

\begin{align}
t': Ax' + Bz' &= C \\
\pi(t'): A px_b + B R y_b &= C(p + y_b)\text{, which is a straight line.} \\
r': x' &= 0\,;\,s': z' = Ax' \\
\pi(r'): x_b &= 0\,;\,\pi(s'): Ry_b = Ap x_b\\ 
\therefore 0 &= r' \cap s' \Rightarrow 0 = \pi(r') \cap \pi(s')
\end{align}

\subsection{Parallelism}

\begin{align}
s'': z' &= Ax' + R \\
\pi(s''): \cancel{Ry_b} &= Ap x_b + R(p + \cancel{y_b}) \Rightarrow x_b = - \cfrac{R}{A} \\
\therefore r' \cap s'' &= (0,R) \in v \Rightarrow \pi(r') \isParallel \pi(s'')
\end{align}

\section{Degenerated Section}

A degenerated conic section are 2 straight lines, but we do not want 2 distinct proofs. 

How do we merge a theorem $T_1$ on a circle and a theorem $T_2$ about 2 straight lines?

We want to prove that: $T(S)\Leftrightarrow$ ``three points are collinear'' holds in a conic section if and only if $T(r,s)\Leftrightarrow$ ``three points are collinear'' holds degeneratedly in 2 straight lines too.

Our way is to intercept a cone $K:z^2 = a^2(x^2 + y^2)$ by a plane.

$L = K\cap \pi_R: y = 0$, for two straight lines $\Rightarrow z = \pm ax$;

$H = K\cap \pi_H: y = c$, for a hyperbola;

$E = K\cap \pi_E: z = bx + c$; $b < b_E$ for an ellipsis;

$P = K\cap \pi_P: z = bx + c$; $b_E < b < b_H$ for a parabola. Here, it suffices that $\pi_P$ has a slope greater than for an ellipsis and less than for an hyperbola.

$\blacksquare$

\vspace{12mm}

Out of charity, there is no salvation at all. With charity, we evolve.

June, the 23th, 2024.
%\clearpage

%\bibliography{references.bib}
\end{document}
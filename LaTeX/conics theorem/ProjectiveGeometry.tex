%%%%%%%%%%%%%%%%%%%%%%%%%%%%%%%%%%%%%%%%%
% University Assignment Title Page 
% LaTeX Template
% Version 1.0 (27/12/12)
%
% This template has been downloaded from:
% http://www.LaTeXTemplates.com
%
% Original author:
% WikiBooks (http://en.wikibooks.org/wiki/LaTeX/Title_Creation)
%
% License:
% CC BY-NC-SA 3.0 (http://creativecommons.org/licenses/by-nc-sa/3.0/)
% 
% Instructions for using this template:
% This title page is capable of being compiled as is. This is not useful for 
% including it in another document. To do this, you have two options: 
%
% 1) Copy/paste everything between \begin{document} and \end{document} 
% starting at \begin{titlepage} and paste this into another LaTeX file where you 
% want your title page.
% OR
% 2) Remove everything outside the \begin{titlepage} and \end{titlepage} and 
% move this file to the same directory as the LaTeX file you wish to add it to. 
% Then add \input{./title_page_1.tex} to your LaTeX file where you want your
% title page.
%
%%%%%%%%%%%%%%%%%%%%%%%%%%%%%%%%%%%%%%%%%
%\title{Title page with logo}
%----------------------------------------------------------------------------------------
%	PACKAGES AND OTHER DOCUMENT CONFIGURATIONS
%----------------------------------------------------------------------------------------

\documentclass[12pt]{article}
\usepackage[portuguese]{babel}
\usepackage[utf8x]{inputenc}
\usepackage{amsmath}
\numberwithin{table}{section}
\usepackage{graphicx}
\usepackage[colorinlistoftodos]{todonotes}
\usepackage{acronym}
\usepackage{multirow}
\usepackage{array}
\usepackage{float}
\usepackage{units}
\usepackage{mathtools}
\usepackage{amsthm}
\usepackage{amsmath}
\usepackage{amssymb}
\usepackage{graphicx}
\usepackage{adjustbox}
\usepackage{setspace}
\usepackage{wasysym}
\usepackage{esint}
\usepackage{tikz-cd}
\usepackage[colorlinks = true,
            linkcolor = red, % acrônimos
            urlcolor  = blue,
            citecolor = coolblack,
            anchorcolor = yellow]{hyperref}
%\usepackage[numbers]{natbib}
\usepackage[abbrvnat]{natbib}
\newcommand\citeay[1]{%
%  \citeauthor{#1}~[\citeyear{#1}]}
    \cite{#1}}
\newcommand\citeayp[1]{%
%  [\citeauthor{#1}, \citeyear{#1}]}
    \citep{#1}}

\usepackage{geometry}
\bibliographystyle{abbrvnat}



\geometry{verbose,a4paper,tmargin=30mm,bmargin=20mm,lmargin=30mm,rmargin=20mm}
\begin{document}
\definecolor{coolblack}{rgb}{0.0, 0.18, 0.39}


% PDF informations
                  
\setstretch{1.5}
\newtheorem{definition}{Definição}
\newcommand\LG{\mathord{\mathcal{L}}}
\newcommand\LM{\mathord{\mathcal{L}}_m}


\begin{titlepage}

\newcommand{\HRule}{\rule{\linewidth}{0.4mm}} % Defines a new command for the horizontal lines, change thickness here

\center % Center everything on the page
 
%----------------------------------------------------------------------------------------
%	HEADING SECTIONS
%----------------------------------------------------------------------------------------

\textsc{\normalsize Universidade Federal de Minas Gerais}\\[0.05cm]
\begin{figure}[!h]
   \centering{
   \includegraphics[scale=0.65]{ufmglogo.png} \\[0.5cm]
   }
 \end{figure} 


%----------------------------------------------------------------------------------------
%	TITLE SECTION
%----------------------------------------------------------------------------------------

\HRule \\[0.4cm]
{ \huge \bfseries Projective Geometry Again}\\[0.4cm] 
\HRule \\[0.7cm]
 
%----------------------------------------------------------------------------------------
%	AUTHOR SECTION
%----------------------------------------------------------------------------------------
{\normalsize TEAM:
\begin{itemize}
    \item[] Vinicius Claudino Ferraz PPGEE-UFMG.
\end{itemize}
}


$\,$\\[2.0cm]
%----------------------------------------------------------------------------------------
%	DATE SECTION
%----------------------------------------------------------------------------------------

%{\large \today} % Date, change the \today to a set date if you want to be precise

%----------------------------------------------------------------------------------------
%	LOGO SECTION
%----------------------------------------------------------------------------------------

%\includegraphics[width=0.5\textwidth]{ufmgLogo.jpg}
%\includegraphics{ufmglogo.png}\\[1cm] %
 
%----------------------------------------------------------------------------------------
\vfill % Fill the rest of the page with whitespace

\end{titlepage}

%\begin{abstract}
%Your abstract.
%\end{abstract}

% List of Acronyms Definition



\section{The Conic Theorem}

\begin{align}
A &= X_1 X_5 \cap X_2 X_6 \\
B &= X_1 X_4 \cap X_3 X_6 \\
C &= X_2 X_4 \cap X_3 X_5 \in AB \label{eq3} \\
A &= (0,0) \\
B &= (0,a) \\
X_1 &= (x_1, bx_1) \\
X_5 &= (x_5, bx_5) \\
X_2 &= (x_2, cx_2) \\
X_6 &= (x_6, cx_6) \\
X_4 &= (x_4, d(x_4 - a)) \\
bx_1 &= d(x_1 - a) \Rightarrow d = \cfrac{b x_1}{x_1 - a} \\
X_3 &= (x_3, e(x_3 - a)) \\
cx_6 &= e(x_6 - a) \Rightarrow e = \cfrac{cx_6}{x_6 - a} \\
\alpha x_1^2 + \beta (bx_1)^2 &= 1 \Rightarrow x_1^2 = \cfrac{1}{\alpha + \beta b^2} = \lambda \\
\alpha x_2^2 + \beta (cx_2)^2 &= 1 \Rightarrow  x_2^2 =\cfrac{1}{\alpha + \beta c^2} = \xi\\  
\alpha x_5^2 + \beta (bx_5)^2 &= 1\Rightarrow x_5^2 = \cfrac{1}{\alpha + \beta b^2} = \lambda \therefore x_5 = \epsilon_1 x_1 \\
\alpha x_6^2 + \beta (cx_6)^2 &= 1 \Rightarrow  x_6^2 =\cfrac{1}{\alpha + \beta c^2} = \xi \therefore x_2 = \epsilon_6 x_6\\
\alpha x_3^2 + \beta [e(x_3-a)]^2 &= 1 \Rightarrow e^2 = \cfrac{1 - \alpha x_3^2 }{\beta (x_3-a)^2} \label{eq18}\\
\alpha x_4^2 + \beta [d(x_4-a)]^2 &= 1 \Rightarrow d^2 = \cfrac{1 - \alpha x_4^2 }{\beta (x_4-a)^2} \label{eq19}
\end{align}

\begin{align}
&\text{Therefore:} \nonumber \\
\cfrac{-y_3 x_5 + y_5 x_3}{x_3 - x_5} &= \cfrac{-y_2 x_4 + y_4 x_2}{x_2 - x_4} \label{eq20} \\
\cfrac{-\cfrac{cx_6}{x_6 - a}(x_3-a) x_5 + b x_5 x_3}{x_3 - x_5} &= \cfrac{-cx_2 x_4 + \cfrac{bx_1}{x_1 - a}(x_4-a) x_2}{x_2 - x_4} 
\end{align}

Below, we try to reduce that to $0x = 0$:

The first intention is to eliminate $x_1$ and $x_6$.

\begin{align}
\zeta &= b x_3 x_5 (x_2 - x_4) + cx_2 x_4(x_3 - x_5) = \zeta_{51} x_5 + \zeta_{50} \\
\zeta_{51} &= bx_2x_3 - b x_3x_4 - cx_2 x_4 \\
\zeta_{50} &= cx_2 x_3 x_4 \\
\eta &= (x_4-a) (x_3 - x_5) = \eta_{51} x_5 + \eta_{50} \\
\eta_{51} &= a - x_4 \\
\eta_{50} &= -x_3 (a - x_4)\\
\kappa &= (x_3-a)^2 (x_2 - x_4)^2 =\kappa_{21} x_2 + \kappa_{20} \\
\kappa_{21} &= - (x_3-a)^2 x_4\\
\kappa_{20} &= (x_3-a)^2 x_4^2 + \xi (x_3-a)^2\\
\eta^2 &= (\eta_{51} x_5 + \eta_{50})^2 = \hat A x_5 + \hat B \\
\hat A &= 2\eta_{51}\eta_{50} \\
\hat B &= \lambda \eta_{51}^2 + \eta_{50}^2\\
\zeta^2 &= (\zeta_{51} x_5 + \zeta_{50})^2  = \hat C x_5 + \hat D\\
\hat C &= 2\zeta_{51}\zeta_{50} \\
\hat D &= \lambda \zeta_{51}^2 + \zeta_{50}^2 \\
\mu &= 4\beta (x_3 - a)^4 (x_4 - a)^4(1 - \alpha x_4^2 ) \\
\nu &= (1 - \alpha x_4^2)(x_3-a)^2(x_4-a)\\
\pi &= \beta (x_3-a)^2(x_4-a)^3
\end{align}
\begin{align}
\rho &= (1 - \alpha x_3^2 )(x_4-a)^3\\
\kappa x_5^2 \underbrace{e^2} &= (\zeta - \eta x_2 \underbrace{d})^2 \\ 
\kappa x_5^2 \cfrac{1 - \alpha x_3^2 }{\beta (x_3-a)^2} &= \eta^2 x_2^2 \cfrac{1 - \alpha x_4^2 }{\beta (x_4-a)^2} - 2 \zeta \eta x_2 \underbrace{d} + \zeta^2 \\
 \mu \zeta^2 \eta^2 x_2^2  &= (\eta^2 x_2^2 \nu  + \pi \zeta^2 - \kappa x_5^2 \rho)^2 
\end{align}

They are already eliminated: $(d,e)$. Let us eliminate $x_5$ and, only partially, $x_2$.

\begin{align}
\xi \mu (\hat A x_5 + \hat B)(\hat C x_5 + \hat D)  &= \left[ \xi\nu (\hat A x_5 + \hat B)  + \pi (\hat C x_5 + \hat D) - \lambda \rho \kappa \right]^2 \\
E x_5 + F  &= ( G x_5 + H )^2 = 2GH x_5 + \lambda G^2 + H^2 \\
\lambda (E - 2GH)^2  &= \left(\lambda G^2 + H^2 - F\right)^2 \\
E &= \xi \mu(\hat A\hat D + \hat B\hat C)\\
F &= \lambda \xi \mu [\hat A\hat C + \hat B\hat D(\alpha + \beta b^2)] \\
G &= \xi \nu \hat A + \pi \hat C \\
H &= \xi \nu \hat B + \pi \hat D - \lambda \rho \kappa
\end{align}

Let us eliminate $x_2$ in $(\zeta,\kappa, \hat C, \hat D)$.

\footnotesize

\begin{align}
&\lambda \left[\xi \mu(\hat A\hat D + \hat B\hat C) - 2\left(\xi \nu \hat A + \pi \hat C\right) \left(\xi \nu \hat B + \pi \hat D - \lambda \rho \kappa\right) \right]^2 = \nonumber\\
&= \left\{\lambda \left(\xi \nu \hat A + \pi \hat C\right)^2 + \left(\xi \nu \hat B   + \pi \hat D - \lambda \rho \kappa\right) ^2 - \lambda \xi \mu [\hat A\hat C + \hat B\hat D(\alpha + \beta b^2)]\right\}^2 
\end{align}

\normalsize

\begin{align}
\kappa(x_2) &= \kappa_{21} x_2 + \kappa_{20}\\
\hat C(x_2) &= 2(bx_2x_3 - b x_3x_4 - cx_2 x_4 )cx_2 x_3 x_4 = Ix_2 + J \\
\hat D(x_2) &= \lambda (bx_2x_3 - b x_3x_4 - cx_2 x_4 )^2 + c^2x_2^2 x_3^2 x_4^2 = K x_2 + L \\
I &= - 2b cx_3^2x_4^2\\
J &= \xi (2bcx_3^2 x_4 - 2c^2 x_3 x_4^2)\\
K &= \lambda (2bc x_3 x_4^2 - 2b^2x_3^2 x_4 )\end{align}
\begin{align}
L &= \lambda b^2 x_3^2 x_4^2 + \xi c^2x_3^2 x_4^2+\lambda \xi (bx_3 - c x_4)^2\\
M &= \xi \mu(\hat AK  + \hat BI ) \\
N &= \xi \mu(\hat AL + \hat BJ) \\
P &= \pi I \\
Q &= \xi \nu \hat A + \pi J \\
R &= \pi K - \lambda \rho \kappa_{21}  \\
S &=\xi \nu \hat B + \pi L - \lambda \rho \kappa_{20} \\ 
T &= - \lambda \xi \mu [\hat AI  + \hat BK  (\alpha + \beta b^2)]\\
U &= - \lambda\xi \mu [\hat A J + \hat B L(\alpha + \beta b^2)]\\
\lambda [Mx_2 + N &-2 (Px_2 + Q)(Rx_2 + S) ]^2  = [\lambda\left(Px_2 + Q\right)^2 + \left(Rx_2 + S\right) ^2 + T x_2 + U]^2 \\
V &= M-2PS -2 QR\\
W &= N -2\xi PR-2 QS \\
Y &= 2\lambda VW\\
Z &= \lambda\xi V^2 + \lambda W^2 \\
\hat E &= 2\lambda PQ + 2RS + T\\
\hat F &= \lambda\xi P^2 +\xi R^2 + \lambda Q^2 + S^2 + U \\
Yx_2 + Z &= (\hat E x_2 + \hat F)^2 =  \xi \hat E^2 + \hat F^2 + 2\hat E\hat F x_2 \\
\xi (Y - 2\hat E\hat F)^2 &= \left(\xi \hat E^2 + \hat F^2 - Z\right)^2 
\end{align}

We have an Equation in $\alpha, \beta, x_3, x_4, a, b, c$.

The question is: is it an identity? Let us express it as a function of $(x_3,x_4) \equiv (x,y,z=x-a,w=y-a)$.

\begin{align}
Y &= 2\lambda (\xi \mu(\hat AK  + \hat BI )-2\pi I(\xi \nu \hat B + \pi L - \lambda \rho \kappa_{20}) \\
& -2\xi \nu \pi \hat A K +2 \lambda \xi \nu \rho \kappa_{21} \hat A -2  \pi^2 J K +2 \lambda \pi \rho \kappa_{21} J)(\xi \mu(\hat AL + \hat BJ) \\
&-2\xi \pi I(\pi K - \lambda \rho \kappa_{21})\\
&-2\xi^2 \nu^2 \hat A \hat B -2 \xi \nu \pi \hat A L +2 \lambda \xi \nu \rho \kappa_{20} \hat A-2 \xi \nu \pi \hat B J -2 \pi^2 J L +2 \lambda \pi \rho \kappa_{20} J)
\end{align}

Let us substitute from $L$ and above, until $\eta$.

\begin{align}
Y &= 2\lambda \{-2\bar e x^2 y^2 w^2(\xi\bar \ell z^4 w^4 + \xi\bar m z^4 w^6 + \xi\bar n z^4 w^5)\\
&-2 \bar f x^3 y w^2(\xi\bar \ell z^4 w^4 + \xi\bar m z^4 w^6 + \xi\bar n z^4 w^5)\\
&+ \lambda \bar\iota x^2 y^2 w^2(\xi\bar \ell z^4 w^4 + \xi\bar m z^4 w^6 + \xi\bar n z^4 w^5)\\
&+ \bar\iota x^4y^2 w^2(\xi\bar \ell z^4 w^4 + \xi\bar m z^4 w^6 + \xi\bar n z^4 w^5)\\
&-2\beta z^2w^3 \bar \iota x^2 y^2\biggl[\xi ( z^2 w - \alpha y^3 z^2 + \bar k y^2 z^2) (\lambda w^2 + x^2 w^2) + \beta z^2w^3 (\bar a x^2 y^2 \\
&+ \bar b x^2 + \bar c y^2 + \bar d xy) - \lambda (w^3 - \alpha x^2 w^3) (y^2 z^2 + \xi z^2)\biggl] \\
& +4\beta\xi x z^2w^5( z^2 w - \alpha y^3 z^2 + \bar k y^2 z^2)   (\bar e xy^2 + \bar f x^2 y) \\
&+4 \lambda \xi xyz^2 w^2( z^2 w - \alpha y^3 z^2 + \bar k y^2 z^2) (w^3 - \alpha x^2 w^3)  \\
&-2  \beta^2 z^4w^6 (\bar g xy^2 + \bar h x^2 y) (\bar e xy^2 + \bar f x^2 y) \\
&-2 \lambda \beta y z^4w^3 (w^3 - \alpha x^2 w^3) (\bar g xy^2 + \bar h x^2 y)\} \cdot \\
&\cdot \{\xi (\bar \ell z^4 w^4 + \bar m z^4 w^6 + \bar n z^4 w^5)\biggl[-2xw^2(\bar a x^2 y^2 + \bar b x^2 
\\&+ \bar c y^2 + \bar d xy ) + (\lambda w^2 + x^2 w^2)(\bar g xy^2 + \bar h x^2 y)\biggl] \\
&-2\xi \beta z^2w^3 \bar \iota x^2 y^2\biggl[\beta z^2w^3 (\bar e xy^2 + \bar f x^2 y) + \lambda yz^2 (w^3 - \alpha x^2 w^3) \biggl]\\
&+4\xi^2 xw^2 ( z^2 w - \alpha y^3 z^2 + \bar k y^2 z^2)^{\color{red}2} (\lambda w^2 + x^2 w^2) \\
&+4 \xi\beta x z^2w^5 ( z^2 w - \alpha y^3 z^2 + \bar k y^2 z^2)  (\bar a x^2 y^2 + \bar b x^2 + \bar c y^2 + \bar d xy ) \\
&-4 \lambda \xi xw^2( z^2 w - \alpha y^3 z^2 + \bar k y^2 z^2) (w^3 - \alpha x^2 w^3) (y^2 z^2 + \xi z^2)  \\
&-2 \xi\beta z^2w^3 ( z^2 w - \alpha y^3 z^2 + \bar k y^2 z^2)  (\lambda w^2 + x^2 w^2) (\bar g xy^2 + \bar h x^2 y) \\
&-2 \beta^2 z^4w^6 (\bar g xy^2 + \bar h x^2 y) (\bar a x^2 y^2 + \bar b x^2 + \bar c y^2 + \bar d xy ) \\
&+2 \lambda \beta z^2w^3 (w^3 - \alpha x^2 w^3) (y^2 z^2 + \xi z^2) (\bar g xy^2 + \bar h x^2 y)\}
\end{align}

If we did not want to give it all up, we ``would'' have:

\begin{align}
Z &= \lambda\xi [\xi \mu(\hat AK  + \hat BI )-2\pi I(\xi \nu \hat B + \pi L - \lambda \rho \kappa_{20})  -2\xi \nu \pi \hat A K \\
&+2 \lambda \xi \nu \rho \kappa_{21} \hat A -2  \pi^2 J K +2 \lambda \pi \rho \kappa_{21} J]^2 + \lambda [\xi \mu(\hat AL + \hat BJ) \\
&-2\xi \pi I(\pi K - \lambda \rho \kappa_{21})\\
&-2\xi^2 \nu^2 \hat A \hat B -2 \xi \nu \pi \hat A L +2 \lambda \xi \nu \rho \kappa_{20} \hat A-2 \xi \nu \pi \hat B J -2 \pi^2 J L +2 \lambda \pi \rho \kappa_{20} J]^2 \\
\hat E &= 2\lambda \pi I(\xi \nu \hat A + \pi J) + 2(\pi K - \lambda \rho \kappa_{21})(\xi \nu \hat B + \pi L - \lambda \rho \kappa_{20})  \\
&- \xi \mu (\lambda \hat AI  + \hat BK  )\\
\hat F &= \lambda\xi \pi^2 I^2 +\xi (\pi K - \lambda \rho \kappa_{21})^2 + \lambda (\xi \nu \hat A + \pi J)^2 + (\xi \nu \hat B + \pi L - \lambda \rho \kappa_{20})^2 \\
&- \xi \mu (\lambda \hat A J + \hat B L) \\
L &= \bar a x^2 y^2 + \bar b x^2 + \bar c y^2 + \bar d xy \,;\,\bar a = \lambda b^2 + \xi c^2\,;\,\bar b = \lambda \xi b^2\,;\,\bar c = \lambda \xi c^2\,;\,\bar d = -2 \lambda \xi bc \\
K &= \bar e xy^2 + \bar f x^2 y \,;\,\bar e = 2 \lambda bc \,;\,\bar f = -2\lambda b^2 \\
J &= \bar g xy^2 + \bar h x^2 y \,;\,\bar g = -2\xi c^2 \,;\,\bar h = 2\xi bc \\
I &= \bar \iota x^2 y^2 \,;\, \bar \iota = -2bc \\
\kappa_{21} &= -yz^2 \\
\kappa_{20} &= y^2 z^2 + \xi z^2\\
\rho &= w^3 - \alpha x^2 w^3\\
\pi &= \beta z^2w^3\\
\nu &= z^2 w - \alpha y^3 z^2 + \bar k y^2 z^2\,;\,\bar k = \alpha \cdot a\\
\mu &= \bar \ell z^4 w^4 + \bar m z^4 w^6 + \bar n z^4 w^5 \,;\,\bar \ell = 4\beta   -4\alpha\beta a^2 \,;\,\bar m = -4\alpha\beta \,;\,\bar n = - 8\alpha\beta a \\
\hat B &= \lambda w^2 + x^2 w^2\\
\hat A &= -2xw^2 \\
\bar o &=
\end{align}

\newpage

Below, let us explain the Equation (\ref{eq20}), expliciting C on Equation (\ref{eq3}).

\begin{align}
    \begin{pmatrix}
        x_2 & 1 \\ x_4 & 1
    \end{pmatrix}  
    \begin{pmatrix}
        a \\ b
    \end{pmatrix} &=
    \begin{pmatrix}
        y_2 \\ y_4
    \end{pmatrix} \Rightarrow 
    \begin{pmatrix}
        a \\ b
    \end{pmatrix} = \cfrac{1}{x_2 - x_4}
    \begin{pmatrix}
        1 & -1 \\ -x_4 & x_2
    \end{pmatrix}  
    \begin{pmatrix}
        y_2 \\ y_4
    \end{pmatrix} \\
    \begin{pmatrix}
        x_3 & 1 \\ x_5 & 1
    \end{pmatrix}  
    \begin{pmatrix}
        c \\ d
    \end{pmatrix} &=
    \begin{pmatrix}
        y_3 \\ y_5
    \end{pmatrix} \Rightarrow 
    \begin{pmatrix}
        c \\ d
    \end{pmatrix} = \cfrac{1}{x_3 - x_5}
    \begin{pmatrix}
        1 & -1 \\ -x_5 & x_3
    \end{pmatrix}  
    \begin{pmatrix}
        y_3 \\ y_5
    \end{pmatrix} \\
    ax + b &= cx + d \Rightarrow x = \cfrac{d-b}{a-c} = 0 \therefore d = b
\end{align}

\newpage

\section{The Converse Conic Theorem}

\begin{align}
    A &= X_1 X_5 \cap X_2 X_6 \\
    B &= X_1 X_4 \cap X_3 X_6 \\
    C &= X_2 X_4 \cap X_3 X_5 \in AB \\
    A &= (0,0) \\
    B &= (0,a) \\
    X_1 &= (x_1, bx_1) \\
    X_5 &= (x_5, bx_5) \\
    X_2 &= (x_2, cx_2) \\
    X_6 &= (x_6, cx_6) \\
    X_4 &= (x_4, d(x_4 - a) \\
    bx_1 &= d(x_1 - a) \\
    X_3 &== e(x_3 - a) \\
    cx_6 &= e(x_6 - a) \\
    \cfrac{-y_3 x_5 + y_5 x_3}{x_3 - x_5} &= \cfrac{-y_2 x_4 + y_4 x_2}{x_2 - x_4} \\
    \text{Então } \exists &P, \theta, \alpha, \beta, \gamma, \delta \in \{0, 1 \}\text{ tais que:} \nonumber \\
    \alpha x_i^2 + \beta y_i^2 + x(y_i - \gamma x_i) &= \delta, \forall i \in \{1, 2, \cdots, 6\}
\end{align}

$\blacksquare$

Out of charity, there is no salvation at all. With charity, we evolve.

June, the 17th, 2024.
%\clearpage

%\bibliography{references.bib}
\end{document}
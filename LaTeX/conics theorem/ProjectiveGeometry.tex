\documentclass[12pt]{article}
\usepackage[portuguese]{babel}
\usepackage[utf8x]{inputenc}
\usepackage{indentfirst}
\usepackage{amsmath}
\usepackage{graphicx}
\usepackage{array}
\usepackage{float}
\usepackage{units}
\usepackage{mathtools}
\usepackage{amsthm}
\usepackage{amsmath}
\usepackage{amssymb}
\usepackage{graphicx}
\usepackage{cancel}
\usepackage{tikz-cd}
\usepackage[colorlinks = true,
            linkcolor = red, % acrônimos
            urlcolor  = blue,
            citecolor = coolblack,
            anchorcolor = yellow]{hyperref}
\newcommand{\isParallel}{\mathbin{\!/\mkern-5mu/\!}}
\newcommand{\degree}{^{\circ}}
\usepackage{geometry}
\geometry{verbose,a4paper,tmargin=30mm,bmargin=20mm,lmargin=30mm,rmargin=20mm}
\begin{document}
% PDF informations
                  
\begin{titlepage}

\newcommand{\HRule}{\rule{\linewidth}{0.4mm}} % Defines a new command for the horizontal lines, change thickness here

\center % Center everything on the page
 
%----------------------------------------------------------------------------------------
%	HEADING SECTIONS
%----------------------------------------------------------------------------------------

\textsc{\normalsize Universidade Federal de Minas Gerais}\\[0.05cm]
\begin{figure}[!h]
   \centering{
   \includegraphics[scale=0.65]{ufmglogo.png} \\[0.5cm]
   }
 \end{figure} 


%----------------------------------------------------------------------------------------
%	TITLE SECTION
%----------------------------------------------------------------------------------------

\HRule \\[0.4cm]
{ \huge \bfseries Projective Geometry Again}\\[0.4cm] 
\HRule \\[0.7cm]
 
%----------------------------------------------------------------------------------------
%	AUTHOR SECTION
%----------------------------------------------------------------------------------------
{\normalsize TEAM:
\begin{itemize}
    \item[] Vinicius Claudino Ferraz PPGEE-UFMG.
\end{itemize}
}


$\,$\\[2.0cm]
%----------------------------------------------------------------------------------------
%	DATE SECTION
%----------------------------------------------------------------------------------------

%{\large \today} % Date, change the \today to a set date if you want to be precise

%----------------------------------------------------------------------------------------
%	LOGO SECTION
%----------------------------------------------------------------------------------------

%\includegraphics[width=0.5\textwidth]{ufmgLogo.jpg}
%\includegraphics{ufmglogo.png}\\[1cm] %
 
%----------------------------------------------------------------------------------------
\vfill % Fill the rest of the page with whitespace

\end{titlepage}

%\begin{abstract}
%Your abstract.
%\end{abstract}

% List of Acronyms Definition

\section{The Conic Theorem}

\begin{align}
A &= X_1 X_5 \cap X_2 X_6 \\
B &= X_1 X_4 \cap X_3 X_6 \\
C &= X_2 X_4 \cap X_3 X_5 \in AB \\
X &= \begin{pmatrix}
    x_1 \\ y_1
\end{pmatrix} + t_1 \begin{pmatrix}
    x_5 - x_1 \\
    y_5 - y_1
\end{pmatrix} = \begin{pmatrix}
    x_2 \\ y_2
\end{pmatrix} + t_2 \begin{pmatrix}
    x_6 - x_2 \\
    y_6 - y_2
\end{pmatrix} \\
\begin{pmatrix}
    x_5 - x_1 & x_2 - x_6 \\
    y_5 - y_1 & y_2 - y_6
\end{pmatrix} &\cdot \begin{pmatrix}
    t_{1A} \\
    t_{2A}
\end{pmatrix} = X_2 - X_1 \Rightarrow t_A = M_{15}^{26} (X_2 - X_1)\\
M_{15}^{26} = &\cfrac{1}{(x_5 - x_1)(y_2 - y_6) - (x_2 - x_6)(y_5 - y_1)} \begin{pmatrix}
    y_2 - y_6 & -x_2 + x_6 \\
    -y_5 + y_1 & x_5 - x_1
\end{pmatrix} \\
t_B &= M_{14}^{36} (X_3 - X_1)\\
t_C &= M_{24}^{35} (X_3 - X_2) \\
C &= A + \lambda (B - A) = X_2 + t_{1C} (X_4 - X_2) \\
\lambda &= \cfrac{x_C - x_A}{x_B - x_A} = \cfrac{y_C - y_A}{y_B - y_A}
\end{align}

% \newpage

Therefore:

\begin{align}
\cfrac{x_2 + t_{1C} (x_4 - x_2) - M_{15}^{36}(x_2 - x_1)}{M_{14}^{36} (x_3 - x_1) - M_{15}^{36}(x_2 - x_1)} &=\cfrac{y_2 + t_{2C} (y_4 - y_2) - M_{15}^{36}(y_2 - y_1)}{ M_{14}^{36}(y_3 - y_1) - M_{15}^{36}(y_2 - y_1)} \label{eq11}
\end{align}

For all conic section in $x\hat O y$, there is a circle in $x\hat O z$. The theorem is simplified because the inverse projection of 3 collinear points are 3 collinear points too.

\subsection{Parallel Straight Lines}

\begin{align}
    y_1 &= \epsilon_1 a \\
    y_2 &= \epsilon_2 a \\
    y_3 &= \epsilon_3 a \\
    y_4 &= \epsilon_4 a \\
    y_5 &= \epsilon_5 a \\
    y_6 &= \epsilon_6 a 
\end{align}

Below, we try to reduce that to $0x = 0$.

\begin{align}
\cfrac{x_2 + t_{1C} (x_4 - x_2) - M_{15}^{36}(x_2 - x_1)}{M_{14}^{36} (x_3 - x_1) - M_{15}^{36}(x_2 - x_1)} &=\cfrac{\epsilon_2 a + t_{2C} (\epsilon_4 a - \epsilon_2 a) - M_{15}^{36}(\epsilon_2 a - \epsilon_1 a)}{M_{14}^{36}(\epsilon_3 a - \epsilon_1 a) - M_{15}^{36}(\epsilon_2 a - \epsilon_1 a)}
\end{align}

\subsection{Concurrent Straight Lines}

\begin{align}
    y_1 &= \epsilon_1 ax_1 \\
    y_2 &= \epsilon_2 ax_2 \\
    y_3 &= \epsilon_3 ax_3 \\
    y_4 &= \epsilon_4 ax_4 \\
    y_5 &= \epsilon_5 ax_5 \\
    y_6 &= \epsilon_6 ax_6 
\end{align}

Below, we try to reduce that to $0x = 0$.

\begin{align}
\cfrac{x_2 + t_{1C} (x_4 - x_2) - M_{15}^{36}(x_2 - x_1)}{M_{14}^{36} (x_3 - x_1) - M_{15}^{36}(x_2 - x_1)} &=\cfrac{\epsilon_2 ax_2 + t_{2C} (\epsilon_4 ax_4 - \epsilon_2 ax_2) - M_{15}^{36}(\epsilon_2 ax_2 - \epsilon_1 ax_1)}{M_{14}^{36}(\epsilon_3 ax_3 - \epsilon_1 ax_1) - M_{15}^{36}(\epsilon_2 ax_2 - \epsilon_1 ax_1)}
\end{align}

\subsection{Circle}

\begin{align}
  X_1 &= (\cos t_1, \sin t_1) \\ 
  X_2 &= (\cos t_2, \sin t_2) \\ 
  X_3 &= (\cos t_3, \sin t_3) \\ 
  X_4 &= (\cos t_4, \sin t_4) \\ 
  X_5 &= (\cos t_5, \sin t_5) \\ 
  X_6 &= (\cos t_6, \sin t_6) 
\end{align}

Below, we try to reduce that to $0x = 0$.

\footnotesize

\begin{align}
\cfrac{\cos t_2 + t_{1C} (\cos t_4 - \cos t_2) - M_{15}^{36}(\cos t_2 - \cos t_1)}{M_{14}^{36} (\cos t_3 - \cos t_1) - M_{15}^{36}(\cos t_2 - \cos t_1)} &=\cfrac{\sin t_2 + t_{2C} (\sin t_4 - \sin t_2) - M_{15}^{36}(\sin t_2 - \sin t_1)}{M_{14}^{36}(\sin t_3 - \sin t_1) - M_{15}^{36}(\sin t_2 - \sin t_1)}
\end{align}

\normalsize

\newpage

\section{The Converse Conic Theorem}

\begin{align}
    A &= X_1 X_5 \cap X_2 X_6 \\
    B &= X_1 X_4 \cap X_3 X_6 \\
    C &= X_2 X_4 \cap X_3 X_5 \in AB\Leftrightarrow (\ref{eq11}) \\
    \text{Therefore, } \exists &\lambda\in\mathbb{R}^6-\{0\}\text{, such that:} \nonumber \\
    \begin{pmatrix}
        x_1^2 & y_1^2 & x_1 y_1 & x_1 & y_1 & 1 \\
        \vdots & \vdots & \vdots & \vdots & \vdots & \vdots \\
        x_6^2 & y_6^2 & x_6 y_6 & x_6 & y_6 & 1 
    \end{pmatrix}\cdot \lambda &= 0
\end{align}

We want to show that:

\begin{align}
0 &=
    \begin{vmatrix}
        x_2^2 - x_1^2 & y_2^2 - y_1^2 & x_2 y_2 - x_1 y_1 & x_2 - x_1 & y_2 - y_1 \\
        x_3^2 - x_1^2 & y_3^2 - y_1^2 & x_3 y_3 - x_1 y_1 & x_3 - x_1 & y_3 - y_1 \\
        x_4^2 - x_1^2 & y_4^2 - y_1^2 & x_4 y_4 - x_1 y_1 & x_4 - x_1 & y_4 - y_1 \\
        x_5^2 - x_1^2 & y_5^2 - y_1^2 & x_5 y_5 - x_1 y_1 & x_5 - x_1 & y_5 - y_1 \\
        x_6^2 - x_1^2 & y_6^2 - y_1^2 & x_6 y_6 - x_1 y_1 & x_6 - x_1 & y_6 - y_1 
    \end{vmatrix} 
\end{align}

\scriptsize

\begin{align}
 0 &= (y_2 - y_1) \begin{vmatrix}
        x_3^2 - x_1^2 & y_3^2 - y_1^2 & x_3 y_3 - x_1 y_1 & x_3 - x_1  \\
        x_4^2 - x_1^2 & y_4^2 - y_1^2 & x_4 y_4 - x_1 y_1 & x_4 - x_1  \\
        x_5^2 - x_1^2 & y_5^2 - y_1^2 & x_5 y_5 - x_1 y_1 & x_5 - x_1 \\
        x_6^2 - x_1^2 & y_6^2 - y_1^2 & x_6 y_6 - x_1 y_1 & x_6 - x_1
        \end{vmatrix}
        - (y_3 - y_1) \begin{vmatrix}
        x_2^2 - x_1^2 & y_2^2 - y_1^2 & x_2 y_2 - x_1 y_1 & x_2 - x_1  \\
        x_4^2 - x_1^2 & y_4^2 - y_1^2 & x_4 y_4 - x_1 y_1 & x_4 - x_1  \\
        x_5^2 - x_1^2 & y_5^2 - y_1^2 & x_5 y_5 - x_1 y_1 & x_5 - x_1 \\
        x_6^2 - x_1^2 & y_6^2 - y_1^2 & x_6 y_6 - x_1 y_1 & x_6 - x_1
        \end{vmatrix}\nonumber\\
        &+ (y_4 - y_1)  \begin{vmatrix}
        x_2^2 - x_1^2 & y_2^2 - y_1^2 & x_2 y_2 - x_1 y_1 & x_2 - x_1  \\
        x_3^2 - x_1^2 & y_3^2 - y_1^2 & x_3 y_3 - x_1 y_1 & x_3 - x_1  \\
        x_5^2 - x_1^2 & y_5^2 - y_1^2 & x_5 y_5 - x_1 y_1 & x_5 - x_1 \\
        x_6^2 - x_1^2 & y_6^2 - y_1^2 & x_6 y_6 - x_1 y_1 & x_6 - x_1
        \end{vmatrix} 
        - (y_5 - y_1) \begin{vmatrix}
        x_2^2 - x_1^2 & y_2^2 - y_1^2 & x_2 y_2 - x_1 y_1 & x_2 - x_1  \\
        x_3^2 - x_1^2 & y_3^2 - y_1^2 & x_3 y_3 - x_1 y_1 & x_3 - x_1  \\
        x_4^2 - x_1^2 & y_4^2 - y_1^2 & x_4 y_4 - x_1 y_1 & x_4 - x_1 \\
        x_6^2 - x_1^2 & y_6^2 - y_1^2 & x_6 y_6 - x_1 y_1 & x_6 - x_1
        \end{vmatrix} \nonumber\\
        &+ (y_6 - y_1)  \begin{vmatrix}
        x_2^2 - x_1^2 & y_2^2 - y_1^2 & x_2 y_2 - x_1 y_1 & x_2 - x_1  \\
        x_3^2 - x_1^2 & y_3^2 - y_1^2 & x_3 y_3 - x_1 y_1 & x_3 - x_1  \\
        x_4^2 - x_1^2 & y_4^2 - y_1^2 & x_4 y_4 - x_1 y_1 & x_4 - x_1 \\
        x_5^2 - x_1^2 & y_5^2 - y_1^2 & x_5 y_5 - x_1 y_1 & x_5 - x_1
        \end{vmatrix}
\end{align}

\normalsize

\subsection{Beyond Conic Sections}

More generally,

\begin{align}
    A &= X_1 X_5 \cap X_2 X_6 \\
    B &= X_1 X_4 \cap X_3 X_6 \\
    C &= X_2 X_4 \cap X_3 X_5 \in AB\Leftrightarrow (\ref{eq11}) \Rightarrow \forall f : \mathbb{R}^2 \to \mathbb{R},\\
    f(x,y) &= \sum_{i,j\ge 0} a_{ij} x^i y^j\Rightarrow \left\{\begin{aligned}
        f(x_1,y_1) &= 0 \\
        f(x_2,y_2) &= 0 \\
        f(x_3,y_3) &= 0 \\
        f(x_4,y_4) &= 0 \\
        f(x_5,y_5) &= 0 \\
        f(x_6,y_6) &= 0 
    \end{aligned}\right. \\
\end{align}

\section{From Circle to Parabola}

\begin{align}
S': x'^2 + z'^2 &= R^2, y = 0 \\
v &= (\mathbb{R}, 0, R) \\
P &= (0, -p, R) \\
A' &= (x', 0, z') \in x\hat Oz \\
A'P: (x,y,z) &= (x' + \lambda x', \lambda p, z' + \lambda (z' - R) \\
B &= \pi(A')\,;\,\pi: x\hat Oz \to x\hat Oy \\
&= A'P \cap x\hat O y = (x_b, y_b, z_b = 0) \\
x_b &= \cfrac{Rx'}{R-z'}\,;\,y_b = \cfrac{pz'}{R-z'} \\
(R - z') y_b &= pz' \Rightarrow z' = \cfrac{Ry_b}{p+y_b} \\
x' &= \cfrac{R-z'}{R}\cdot x_b = \cfrac{px_b}{p+y_b} \\
\pi(S'): p^2 x_b^2 + \cancel{R^2 y_b^2} &= R^2 (p+y_b)^2 = p^2 R^2 + 2pR^2y_b + \cancel{R^2 y_b^2}\,;\,z_b=0 \\
\therefore y_b &= \cfrac{px_b^2 - p R^2}{2R^2}
\end{align}

That is a parabola that intercepts $\hat O y$ at $V = \left(0, -\cfrac{p}{2}, 0\right)$, and intercepts $\hat Ox$ at $(\pm R, 0, 0)$.

\subsection{From Circle to Hiperbola}

\begin{align}
S'': x'^2 &+ z'^2 = (R+q)^2, y = 0 \\
\pi(S''): p^2 x_b^2 + R^2 y_b^2 &= (R+q)^2 (p^2 + 2py_b +y_b^2)\,;\,z_b=0 \\
p^2x_b^2 &= (2Rq + q^2) y_b^2 + 2p(R+q)^2 y_b + p^2(R+q)^2\\
p^2(R+q)^4 - p^2q(R+q)^2(2R + q) &= \left[(2R + q)q y_b + p(R+q)^2\right]^2-(2R+q)p^2qx_b^2 \\
\cfrac{1}{A^2} \left[y_b+\cfrac{p(R+q)^2}{q(2R + q)}\right]^2-\cfrac{x_b^2}{B^2} &= 1\,;\,A = \cfrac{pR(R+q)}{q(2R+q)}\,;\,B = \cfrac{R(R+q)}{\sqrt{q(2R+q)}}
\end{align}

Whenever $x_b = 0$, we have $y_b + C(q) = \pm A$. Therefore, that's a vertical hyperbola.

We want to prove that while $q\to \infty$, the projection is a degenerated hyperbola.

\begin{align}
\cfrac{(y + c)^2}{a^2} - \cfrac{x^2}{b^2} &= 1\Rightarrow y = -c \pm a\sqrt{1 + \cfrac{x^2}{b^2}} \Rightarrow y' = \pm  \cfrac{ax}{\sqrt{1 + \cfrac{x^2}{b^2}} } \,\xrightarrow{x \to \infty}\, \pm \cfrac{a}{b} \\
y &= \pm \cfrac{ax}{b} + y_0\,;\,y_0 = -c\pm \cancel{a}\\
\cfrac{A}{B} &= \cfrac{p}{\sqrt{q(2R+q)}}\,\xrightarrow{q \to \infty}\,0\\-C\pm A &= -\cfrac{p(R+q)^2}{q(2R + q)}\pm \cfrac{pR(R+q)}{q(2R+q)}\xrightarrow{q \to \infty}\,-p
\end{align}

\subsection{From Circle to Ellipsis}

\begin{align}
S''': x'^2 + z'^2 &= (R-q)^2, y = 0 \\
\pi(S'''): \cfrac{1}{A^2} \left[y_b-\cfrac{p(R-q)^2}{q(2R - q)}\right]^2+\cfrac{x_b^2}{B^2} &= 1 \,;\,A = \cfrac{pR(R-q)}{q(2R-q)}\,;\,B = \cfrac{R(R-q)}{\sqrt{q(2R-q)}}
\end{align}

Whenever $y_b + C(-q) = 0$, we have $x_b = \pm B$.

\subsection{Invariant Straight Lines}

\begin{align}
t': Ax' + Bz' &= C \\
\pi(t'): A px_b + B R y_b &= C(p + y_b)\text{, which is a straight line.} \\
r': x' &= 0\,;\,s': z' = Ax' \\
\pi(r'): x_b &= 0\,;\,\pi(s'): Ry_b = Ap x_b\\ 
\therefore 0 &= r' \cap s' \Rightarrow 0 = \pi(r') \cap \pi(s')
\end{align}

\subsection{Parallelism}

\begin{align}
s'': z' &= Ax' + R \\
\pi(s''): \cancel{Ry_b} &= Ap x_b + R(p + \cancel{y_b}) \Rightarrow x_b = - \cfrac{R}{A} \\
\therefore r' \cap s'' &= (0,R) \in v \Rightarrow \pi(r') \isParallel \pi(s'')
\end{align}

\section{Degenerated Section}

A degenerated conic section are 2 straight lines, but we do not want 2 distinct proofs. 

How do we merge a theorem $T_1$ on a circle and a theorem $T_2$ about 2 straight lines?

We want to prove that: $T(S)\Leftrightarrow$ ``three points are collinear'' holds in a conic section if and only if $T(r,s)\Leftrightarrow$ ``three points are collinear'' holds degeneratedly in 2 straight lines too.

Our way is to intercept a cone $K:z^2 = a^2(x^2 + y^2)$ by a plane.

$L = K\cap \pi_R: y = 0$, for two straight lines $\Rightarrow z = \pm ax$;

$H = K\cap \pi_H: y = c$, for a hyperbola;

$E = K\cap \pi_E: z = bx + c$; $b < b_E$ for an ellipsis;

$P = K\cap \pi_P: z = bx + c$; $b_E < b < b_H$ for a parabola. Here, it suffices that $\pi_P$ has a slope greater than for an ellipsis and less than for an hyperbola.

\section{Higher Dimensions}

\begin{align}
A &= X_1 X_5 \cap X_2 X_6 X_7 \\
B &= X_1 X_4 \cap X_3 X_6 X_7 \\
C &= X_2 X_4 \cap X_3 X_5 X_7 \\
D &= X_4 X_7 \cap X_2 X_3 X_5 \in \langle ABC \rangle \\
X = \begin{pmatrix}
    x_1 \\ y_1 \\ z_1 
\end{pmatrix} + t_1 \begin{pmatrix}
    x_5 - x_1 \\
    y_5 - y_1 \\
    z_5 - z_1 
\end{pmatrix} &= \begin{pmatrix}
    x_2 \\ y_2 \\ z_2
\end{pmatrix} + t_2 \begin{pmatrix}
    x_6 - x_2 \\
    y_6 - y_2 \\
    z_6 - z_2 
\end{pmatrix} + t_3 \begin{pmatrix}
    x_7 - x_2 \\
    y_7 - y_2 \\
    z_7 - z_2
\end{pmatrix} \\
\begin{pmatrix}
    x_5 - x_1 & x_2 - x_6 & x_2 - x_7 \\
    y_5 - y_1 & y_2 - y_6 & y_2 - y_7 \\
    z_5 - z_1 & z_2 - z_6 & z_2 - z_7 
\end{pmatrix} &\cdot \begin{pmatrix}
    t_{1A} \\
    t_{2A} \\
    t_{3A} 
\end{pmatrix} = X_2 - X_1 \Rightarrow t_A = M_{15}^{267} (X_2 - X_1)\\
t_B &= M_{14}^{367} (X_3 - X_1)\\
t_C &= M_{24}^{357} (X_3 - X_2) \\
t_D &= M_{47}^{235} (X_2 - X_4) \\
D &= A + \lambda (B - A) + \mu (C - A) \\
\begin{pmatrix}
    x_B - x_A & x_C - x_A \\
    y_B - y_A & y_C - y_A
\end{pmatrix} \cdot \begin{pmatrix}
    \lambda \\ \mu
\end{pmatrix} &= \begin{pmatrix}
    x_D \\ y_D 
\end{pmatrix} \\
(z_B - z_A) \lambda + (z_C - z_A)\mu &= z_D \\
0 &= \langle (x^2, y^2, z^2, xy, xz, yz, x, y, z, 1), (\lambda_1, \cdots, \lambda_{10})\rangle
\end{align}

$\blacksquare$

\vspace{12mm}

Out of charity, there is no salvation at all. With charity, we evolve.

July, the 20th, 2024.
%\clearpage

%\bibliography{references.bib}
\end{document}
\documentclass[12pt]{article}
\usepackage{amsmath}
\usepackage{amssymb} %mathbb
\usepackage{graphicx}
\usepackage{hyperref}
\usepackage[latin1]{inputenc}

\begin{document}

\Large

\begin{center}
Mean 4 Clocks
\end{center}

\normalsize

\begin{verbatim}
-- mean_4_clocks - A system that calculates the mean over four clocks
-- Copyright (C) 2018  Digital Systems Group - UFMG
--
-- This program is free software; you can redistribute it and/or modify
-- it under the terms of the GNU General Public License as published by
-- the Free Software Foundation; either version 3 of the License, or
-- (at your option) any later version.
--
-- This program is distributed in the hope that it will be useful,
-- but WITHOUT ANY WARRANTY; without even the implied warranty of
-- MERCHANTABILITY or FITNESS FOR A PARTICULAR PURPOSE.  See the
-- GNU General Public License for more details.
--
-- You should have received a copy of the GNU General Public License
-- along with this program; if not, see <https://www.gnu.org/licenses/>.
--

library ieee;
use ieee.std_logic_1164.all;
use ieee.numeric_std.all;

entity mean_4_clocks is
    generic (
        W       :       integer := 16
    );
    port (
        CLK     : in    std_logic;
        RESET   : in    std_logic;
        INPUT   : in    std_logic_vector(W - 1 downto 0);
        OUTPUT  : out   std_logic_vector(W - 1 downto 0)
    );
end mean_4_clocks;

-- Implement the testbench and find the errors in this model.
-- Consider the following expected behavior:
--      Every rising edge of CLK input, the content of INPUT
--      is added to a register chain and used to calculate the
--      mean value over 4 clock periods

architecture arch of mean_4_clocks is
        signal var1 : unsigned(W - 1 downto 0);
        signal var2 : unsigned(W - 1 downto 0);
        signal var3 : unsigned(W - 1 downto 0);
        signal var4 : unsigned(W - 1 downto 0);
begin
	  var1 <= to_unsigned(0,W) when RESET = '1'
	  else unsigned(INPUT) SRL 2 when (rising_edge(CLK));

	  var2 <= to_unsigned(0,W) when RESET = '1'
	  else var1 when (rising_edge(CLK));

	  var3 <= to_unsigned(0,W) when RESET = '1'
	  else var2 when (rising_edge(CLK));

	  var4 <= to_unsigned(0,W) when RESET = '1'
	  else var3 when (rising_edge(CLK));

  OUTPUT <= std_logic_vector(var1 + var2 + var3 + var4);
end arch;
\end{verbatim}

\vspace{10mm}

\Large

\begin{center}
Test Bench
\end{center}

\normalsize

\begin{verbatim}
library ieee;
use ieee.std_logic_1164.all;
use ieee.numeric_std.all;

entity tb_alu is
--    generic (
--        myW       :       integer := 16
--    );
end tb_alu;

architecture teste of tb_alu is

component mean_4_clocks is
    generic (
        W       :       integer := 16
    );
    port (
        CLK     : in    std_logic;
        RESET   : in    std_logic;
        INPUT   : in    std_logic_vector(W - 1 downto 0);
        OUTPUT  : out   std_logic_vector(W - 1 downto 0)
    );
end component;

signal C, R : std_logic;
signal Inpt, otpt : std_logic_vector(16 - 1 downto 0);
begin
instancia_alu: mean_4_clocks     generic map (W => 16)	 port map(
   CLK=>C,
   RESET=>R,
   INPUT=>inpt,
   OUTPUT=>otpt);
C <= '0',
     '1' after 10 ns,
	  '0' after 20 ns,
	  '1' after 30 ns,
	  '0' after 40 ns,
	  '1' after 50 ns,
	  '1' after 60 ns;
R <= '1',
     '0' after 5 ns,
	  '0' after 20 ns,
	  '0' after 30 ns,
	  '0' after 40 ns,
	  '0' after 50 ns,
	  '0' after 60 ns;
inpt <= (x"00010101"),
      (x"00001111") after 15 ns,
		(x"00002222") after 25 ns,
		(x"00003333") after 35 ns,
		(x"00004444") after 45 ns,
		(x"00005555") after 55 ns,
		(x"0000CCCC") after 65 ns;
end teste;
\end{verbatim}


\vspace{3mm}

Out of charity, there is no salvation at all. With charity, there is evolution.

\vspace{3mm}

Vinicius Claudino FERRAZ, 13/Sep/2019, Release $1.0$

\end{document}

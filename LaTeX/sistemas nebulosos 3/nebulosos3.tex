\documentclass[12pt]{article}
\usepackage{amsmath}
\usepackage{amssymb} %mathbb
\usepackage{graphicx}
\usepackage{hyperref}
\usepackage{colortbl}
\usepackage[latin1]{inputenc}
\usepackage[top=1.0cm,bottom=1.3cm,left=1.0cm,right=1.0cm]{geometry}

\begin{document}

\Large

\begin{center}
\href{https://www.youtube.com/watch?v=2E0TYW42x1c}{\color{blue}\underline{Sistemas Nebulosos 3}}
\end{center}

\normalsize

\vspace{3mm}

\section{Controlador RL de primeira ordem}

\begin{align}
  h &= 0,001\,;\,0 \le k \le 23000 h\,;\,a = 1 \le x \le 24 = 8b
\end{align}

Pela soma de Riemann,

\begin{align}
  i(u)(k) &= e^{-R/L (a + hk)}\cfrac{1}{L}\sum_{n > 0}^k e^{R/L (a + hn)} u(n) h \\  \mu &:= \mu_1 \Rightarrow \mu_2 = 1 - \mu \\
  i_{ref}(\mathbb{Z}) &= \text{degrau restrito} \\
  e &= i - i_{ref} \\
  e(k) &\le - E_1 \Rightarrow \mu(k) = 1 \\
  e(k) &\ge E_2 \Rightarrow \mu(k) = 0 \\
  - E_1 &\le e(k) \le E_2 \Rightarrow \mu(k) = p e(k) + q \\
  u_{ref} &= R i_{ref} + L \cfrac{d}{dt} i_{ref} \\
  \Delta u &= u - u_{ref}
\end{align}

Para cada alfa, conseguimos $| \Delta u | < \delta \Rightarrow | e | < \epsilon$. [Provar.]

A primeira quest\~ao \'e se podemos inverter. Para \'epsilon = 0,08, ser\'a que o valor de alfa = 1 vai for\c{c}ar que o delta de entrada seja 2,5 e portanto garantir que o m\'odulo do erro de sa\'ida seja sempre menor que \'epsilon???

\textbf{Conjectura 1: }$\alpha = 1 \Rightarrow |\Delta u| < 2,5 \Rightarrow |e| < 0,08$.

\textbf{Conjectura 2: }$ |\Delta u| < 2,5\cdot \alpha \Rightarrow |e| < 0,08\cdot \alpha $.

\textbf{Conjectura 3: }$ |\Delta u| < \delta \Rightarrow |e| < 0,032\cdot \delta$.

\begin{align}
u(k+1) &= \mu_k q_1(k+1) + (1 - \mu_k) q_2(k+1) \\
&= \mu_k \sum_n \alpha \mu_n e + (1 - \mu_k) \sum_n \alpha (1 - \mu_n) e \\
u(i)(k+1) &= \sum_{n = 0}^k \alpha \varphi(n,k) [i(n) - i_{ref}(n)] \\
\varphi(n, k) &= 2 \mu_n \mu_k + 1 - \mu_k - \mu_n
\end{align}

A equa\c{c}\~ao 13 passa de u para i. Combinamos com a equa\c{c}\~ao 24, que passa de i para u.

\begin{align}
i &\to i(k+ 1) = e^{-R/L (a + hk + h)}\cfrac{1}{L}\sum_{m > 0}^{k + 1} e^{R/L (a + hm)}  \sum_{n = 0}^{m - 1} \alpha \varphi(n,m-1) [i(n) - i_{ref}(n)] h \\
u &\to u(k + 1) = \sum_{m = 0}^k \alpha \varphi(m,k) \bigg[e^{-R/L (a + hm)}\cfrac{1}{L}\sum_{n > 0}^m e^{R/L (a + hn)} u(n) h - i_{ref}(m)\bigg]
\end{align}

Escrevemos mi em fun\c{c}\~ao do degrau qui. Por falta de letra, \'e o delta de Kronecker de $t \ge 0$.

\begin{align}
  \chi[t] &= 1 \Leftrightarrow t \ge 0 \\
  \chi[t] &= 0 \Leftrightarrow t < 0 \\
  \mu(k) &= \chi[-e(k) - E_1] + \underbrace{[pe(k) + q]}_R\cdot \{ \chi[\underbrace{e(k) + E_1}_A] - \chi[\underbrace{e(k) - E_2}_B] \} \\
  \varphi(f, g) &= 2fg + 1 - f - g \\
  f(e(k)) &= \chi(-a) + R[\chi(a) - \chi(b)] \\
  g(e(n)) &= \chi(-c) + S[\chi(c) - \chi(d)] \\
  \text{IF }c > a &\Rightarrow d> b \Rightarrow acbd \\
  \varphi_1 &= 2 \chi(-a) + 2 S [\chi(a) - \chi(c)] + 2 RS [\chi(c) - \chi(b)] + 1 \\
&- 2 \chi(-a) - (S + 1) [\chi(a) - \chi(c)] - (R + S) [\chi(c) - \chi(b)] - S [\chi(b) - \chi(d)] \\
  \text{ELSE IF }d \le a &\Rightarrow cdab \\
    \varphi_2 &= 2g + 1 - 2 \chi(-c) - (S + 1) [\chi(c) - \chi(d)] - [\chi(d) - \chi(a)] - R [\chi(a) - \chi(b)] \\
  \text{ELSE IF }a < d \le b &\Rightarrow cadb \\
    \varphi_3 &= 2 \chi(-c) + 2 S [\chi(c) - \chi(a)] + 2 RS [\chi(a) - \chi(d)] + 1 \\
&- 2 \chi(-c) - (S + 1) [\chi(c) - \chi(a)] - (R + S) [\chi(a) - \chi(d)] - R [\chi(d) - \chi(b)] \\
  \text{ELSE IF }d > b &\Rightarrow cabd\\
  \varphi_4 &= 2 \chi(-c) + 2 S [\chi(c) - \chi(a)] + 2 RS [\chi(a) - \chi(b)] + 1\\
&- 2 \chi(-c) - (S + 1) [\chi(c) - \chi(a)] - (R + S) [\chi(a) - \chi(b)] - R [\chi(b) - \chi(d)]
\end{align}

Queremos eliminar $\varphi$ das equa\c{c}\~oes 26 e 27.

\section{Controlador RLC de segunda ordem}

\begin{flushleft}
Se o erro for proporcional como vimos na simula\c{c}\~ao em MatLab, basta repetir a invers\~ao. \\
Vamos trocar a equa\c{c}\~ao 20 por
\end{flushleft}

\begin{align}
 u_{ref} &= R i_{ref} + L \cfrac{d}{dt} i_{ref} + \int_{t_0}^t \cfrac{i_{ref}(\tau)}{C}\, \mathrm{d}\tau + K\,;\,u_{ref}(t_0) = 0 \\
  \cfrac{d}{dt} u_{ref} &= R \cfrac{d}{dt} i_{ref} + L \cfrac{d^2}{dt^2} i_{ref} + \cfrac{1}{C}i_{ref}
\end{align}

Basta resolver a equa\c{c}\~ao diferencial $y'' = a y + b y' + c, a = -\cfrac{1}{LC}, b = - \cfrac{R}{L}, c = \cfrac{u_{ref}}{L}$.

A equa\c{c}\~ao caracter\'istica \'e $\lambda^2 = a + b\lambda$.

\begin{align}
 y' &= z \\
 z' &= ay + bz + c \\
 \begin{pmatrix} y \\ z \end{pmatrix}' &= \begin{pmatrix} 0 & 1 \\ a & b \end{pmatrix}\begin{pmatrix} y \\ z \end{pmatrix} + \begin{pmatrix} 0 \\ c \end{pmatrix} \\
 Y' &= P Y + Q \\
 Y &= e^{-tP} \int_{t_0}^t e^{\tau P} Q(\tau)\, \mathrm{d}\tau + Y(t_0)
\end{align}

Os autovalores s\~ao ra\'izes da equa\c{c}\~ao caracter\'istica.

\begin{align}
\lambda^2 - b\lambda - a &= 0 \\
\lambda &= \cfrac{b \pm \sqrt{b^2 + 4a}}{2} = \cfrac{-RC \pm \sqrt{C}\sqrt{R^2 C - 4}}{2LC} \\
\exp \begin{pmatrix} a & 0 \\ 0 & b \end{pmatrix} &= \begin{pmatrix} e^a & 0 \\ 0 & e^b \end{pmatrix} \\
\exp \begin{pmatrix} a & -b \\ b & a \end{pmatrix} &= \begin{pmatrix} e^a\cos b & -e^a \sin b \\ e^a \sin b & e^a \cos b \end{pmatrix} \\
\exp \begin{pmatrix} a & 1 \\ 0 & a \end{pmatrix} &= \begin{pmatrix} e^a & 1 \\ 0 & e^a \end{pmatrix} \\
e^{tP} &= \begin{pmatrix} f(t) & g(t) \\ p(t) & q(t) \end{pmatrix} \\
 Y &= \cfrac{1}{L} \begin{pmatrix} f(-t) & g(-t) \\ p(-t) & q(-t) \end{pmatrix} \int_{t_0}^t \begin{pmatrix} f(\tau) & g(\tau) \\ p(\tau) & q(\tau) \end{pmatrix} \begin{pmatrix} 0 \\ u_{ref}(\tau) \end{pmatrix} \mathrm{d}\tau + Y(t_0) \\
 v(t) &= \int_{t_0}^t g(\tau) u_{ref}(\tau) \,\mathrm{d}\tau \\
 w(t) &= \int_{t_0}^t q(\tau) u_{ref}(\tau) \,\mathrm{d}\tau \\
 Y &= \cfrac{1}{L} \begin{pmatrix} f(-t) & g(-t) \\ p(-t) & q(-t) \end{pmatrix}  \begin{pmatrix} v(t) \\ w(t) \end{pmatrix} + Y(t_0) \\
 y(t) &= \cfrac{f(-t)v(t) + g(-t)w(t)}{L}  + y(t_0)
\end{align}

\section{Controlador Iterado}

\begin{flushleft}
Entrada: $u(t)$ \\
Sa\'ida: $y(t)$ \\
Regra:
\end{flushleft}

\begin{align}
y(k) &= 1.4 y(k-1) - 0.6 y(k-2) - 3 u^3(k -1) + 2u(k-1) - u^3(k-2) + 2u(k-2)
\end{align}

Ser\'a poss\'ivel inverter?

\vspace{3mm}

Out of charity, there is no salvation at all. With charity, there is evolution.

\vspace{3mm}

Vinicius Claudino FERRAZ, 16/Sep/2019, Release $1.0.2$

\end{document}

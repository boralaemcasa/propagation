\documentclass[12pt]{article}
\usepackage{amsmath}
\usepackage{amssymb} %mathbb
\usepackage{graphicx}
\usepackage{hyperref}
\usepackage{xcolor}
\usepackage{gensymb}
\usepackage[latin1]{inputenc}
\usepackage[top=1.0cm,bottom=1.3cm,left=1.0cm,right=1.0cm]{geometry}

\begin{document}

\Large

\begin{center} Resum\~ao | \href{https://drive.google.com/open?id=1GLKu3t37DBGDA6l8v-TJTNLkG7PeAojl}{\color{blue}\underline{Pol\'itopos}}
\end{center}

\vspace{3mm}

\begin{center} Fontes em | \href{https://drive.google.com/file/d/1phUQO-HT7DYhAXLXW-yqCiEGUZJ7kV9f/view?usp=sharing}{\color{blue}\underline{Latex}}
\end{center}

\vspace{3mm}

Today, my program proves that:

With 120 points, there are 600 regular tetrahedra.

With 600 centers, there is a dual polytope with 120 dodecahedra.

With 120 centers, there is a bidual polytope with 600 tetrahedra.

\vspace{3mm}

Gostando ou n\~ao, confira os fontes nos links: \href{https://drive.google.com/open?id=1M_F7Kbrt0EbAXIL11U0AngV0gmLhiYK6}{\color{blue}\underline{Google Drive}}\,\,\,\,\,\,\,\href{https://www.dropbox.com/s/ro3p3d6ye5ue0nu/PolyhedraGroups.zip?dl=0}{\color{blue}\underline{DropBox}}

\vspace{3mm}

Fora da caridade, n\~ao h\'a salva\c{c}\~ao. Com caridade, h\'a evolu\c{c}\~ao.

Vinicius Claudino Ferraz

\end{document}

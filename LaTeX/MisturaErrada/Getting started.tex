\documentclass[12pt]{article}
\usepackage{amsmath}
\usepackage{amssymb} %mathbb
\usepackage{graphicx}
\usepackage{hyperref}
\usepackage[latin1]{inputenc}
\usepackage[top=1.0cm,bottom=1.3cm,left=1.0cm,right=1.0cm,landscape]{geometry}

\begin{document}

\Large

\begin{center}
Misturas
\end{center}

\normalsize

Isto foi uma tentativa de dividir em 20 classes.

\begin{verbatim}
rm(list=ls())
library(’plot3D’)
library(’mlbench’)
p<-mlbench.spirals(1000,cycles=1, sd=0.05)
plot(p)
n <- 1000
totalM <- matrix(0,ncol=3,nrow=n)
for (i in 1:n) {
totalM[i,1] <- p[1]$x[i,1]
totalM[i,2] <- p[1]$x[i,2]
totalM[i,3] <- p[2]$classes[i]
}
write.csv(totalM, file="e:\\_vinicius\\spirals.csv")
#FIZ ISSO PARA LEVAR PARA O LABORATORIO ONDE NAO FUNCIONA mlbench
heart <- read.csv("V:\\_46 reconhecimento padroes\\aula 10\\spirals.csv", sep=",",header=FALSE)
tudo <- as.matrix(heart)
n <- 1000
totalM <- matrix(0,ncol=3,nrow=n)
for (i in 1:n) {
totalM[i,1] <- tudo[i,2]
totalM[i,2] <- tudo[i,3]
}
nn <- 20
totalM <- kmeans(nn, totalM, n, 3)
cores <- rainbow(nn)
for (j in 1:nn) {
  contador <- 0
  for (i in 1:n)
    if (totalM[i, 3] == j) {
      contador <- contador + 1
    }
  teste <- matrix(0,ncol=2,nrow=contador)
  contador <- 1
  for (i in 1:n)
    if (totalM[i, 3] == j) {
      teste[contador,1] = totalM[i,1]
      teste[contador,2] = totalM[i,2]
      contador <- contador + 1
    }
  if (j == 1)
  plot(teste[,1],teste[,2],col = cores[j], xlim = c(-1,1),ylim = c(-1,1),xlab = '' , ylab= '' )
  if (j == 18)
  plot(teste[,1],teste[,2],col = cores[j], xlim = c(-1,1),ylim = c(-1,1),xlab = '' , ylab= '' )
  if (j < nn)
    par(new=T)
}
linhas <- 1000
colunas <- 2
final <- 3
acuracia <- matrix(0,nrow=10, ncol=nn)
#xc[,,1]
#cc[1]

    seqi<-seq(-1 + 0.06,1,0.06)
    seqj<-seq(-1 + 0.06,1,0.06)
    M1 <- matrix(0,nrow=length(seqi),ncol=length(seqj))

for (cluster in 1:nn) {

#cluster <- 1

offset <- 0
for (kk in 1:10) { #calcular acuracia[kk] da classe

  #kk <- 1

  cc <- matrix(0,nrow=nn)
  xc <- array(0, c(n,2,nn))
  for (jj in 1:nn)
    for (i in 1:linhas) {
      if (totalM[i, 3] == jj)
        cc[jj] <- cc[jj] + 1
      for (j in 1:2)
        xc[cc[jj], j, jj] <- totalM[i, j]
    }

    testeN <- trunc(cc[cluster] * 0.1)
    if (cc[cluster] %% 10 != 0)
      if (kk %% 2 == 0)
	  testeN <- testeN + 1
    if (offset + testeN > cc[cluster])
      testeN <- cc[cluster] - offset
    teste <- matrix(0,nrow=testeN,ncol=2)
    trein <- matrix(0,nrow=cc[cluster]-testeN,ncol=colunas)
    c <- 1
    t <- 1
    for (i in 1:linhas)
      if (totalM[i, 3] == cluster) {
	  flag <- 0
	  if ((c > offset) & (c - offset <= testeN)) {
	    for (j in 1:2)
		teste[c - offset, j] <- totalM[i, j]
	    flag <- 1
	  }
 	  if (flag == 0) {
	    if (t <= cc[cluster] - testeN)
	      for (j in 1:2)
		  trein[t, j] <- totalM[i, j]
	    t <- t + 1
	  }
	  c <- c + 1
      }

    par(new=T)
  plot(teste[,1],teste[,2],col = 'black', xlim = c(-1,1),ylim = c(-1,1),xlab = '' , ylab= '' )
    par(new=T)
  plot(trein[,1],trein[,2],col = 'green', xlim = c(-1,1),ylim = c(-1,1),xlab = '' , ylab= '' )


	media <- matrix(0,nrow=nn,ncol=2)
	desvio <- matrix(0,nrow=nn,ncol=2)
      maxk <- matrix(0,nrow=testeN)
	#MEU nn EH 20 ==> EU VOU TER 20 * 2 MEDIAS (X, Y)
	for (jj in 1:nn) {
	  somax <- 0
	  somay <- 0
	  for (j in 1:cc[jj]) {
	    somax <- somax + xc[j,1,jj]
	    somay <- somay + xc[j,2,jj]
	  }
	  media[jj,1] <- somax / cc[jj]
	  media[jj,2] <- somay / cc[jj]
	  for (j in 1:cc[jj]) {
          desvio[jj,1] <- desvio[jj,1] + sqr(xc[j,1,jj] - media[jj,1])
          desvio[jj,2] <- desvio[jj,2] + sqr(xc[j,2,jj] - media[jj,2])
	  }
        desvio[jj,1] <- sqrt(desvio[jj,1]/cc[jj])
        desvio[jj,2] <- sqrt(desvio[jj,2]/cc[jj])
	}

	p <- matrix(0,nrow=testeN,ncol=nn)
	#AO INVES DE p E q ANTIGOS, AGORA TEMOS COLUNAS 1 E 2 DE p (DE 1 A nn = 20)
	for (jj in 1:nn) {
	  K <- cov(xc[,,jj])
	  for (i in 1:testeN)
	    p[i,jj] <- pdfnvar(teste[i,], media[jj,], K, colunas)
	}
	acertos <- 0
      for (i in 1:testeN) {
 	  maximo <- p[i,1]
        maxk[i] <- 1
        for (jj in 2:nn)
	    if (p[i,jj] > maximo) {
		maximo <- p[i,jj]
		maxk[i] <- jj
	    }
   	  if (maxk[i] == cluster)
          acertos <- acertos + 1
      }
	acuracia[kk,cluster] <- acertos * 100 / testeN
	offset <- offset + testeN

  if (kk == 7) { # so falta trocar pelo maximo da acuracia[coluna cluster]
    ci<-0
    for (i in seqi) {
      ci<-ci+1
      cj<-0
      for(j in seqj) {
        cj<-cj+1

        pp <- matrix(0,nrow=nn)
        for (jj in 1:nn)
          pp[jj] <- pdf2var(i,j,media[jj,1],media[jj,2],desvio[jj,1],desvio[jj,2],0) # pdfnvar(x, media[jj,], K, colunas) * cc[jj]/linhas

 	  maximo <- pp[1]
        maxk <- 1
        for (jj in 2:nn)
	    if (pp[jj] > maximo) {
		maximo <- pp[jj]
		maxk <- jj
	    }
   	  flag <- (maxk == cluster)
        if (flag)
          M1[ci,cj]<- cluster
        #else
        #  M1[ci,cj]<- 0
      }
    }
  }
}

}

for (i in 1:nn) {
  print("media")
  print(mean(acuracia[,i]))
  print("desvio padrao")
  print(sd(acuracia[,i]))
}

 #plotando as superfícies de contorno  par(new=T)
 contour2D(M1,seqi,seqj,colkey = NULL)

dist <- function(x, y) { sqrt((x[1] - y[1])^2 + (x[2] - y[2])^2) }
kmeans <- function(k, M, linhas, index) {
centro <- matrix(0,ncol=2,nrow=k)
for (i in 1:k) {
a <- sample(1:linhas, 1)
centro[i,1] <- M[a,1]
centro[i,2] <- M[a,2]
}
distancia <- matrix(0,k)
novoCentro <- matrix(0,ncol=2,nrow=k)
while (TRUE) {
for (i in 1:linhas) {
for (j in 1:k)
distancia[j] <- dist(M[i,], centro[j,])
M[i,index] <- 1
min <- distancia[1]
for (j in 2:k)
if (distancia[j] < min) {
M[i,index] <- j
min <- distancia[j]
}
}
for (j in 1:k) {
sx <- 0
sy <- 0
contador <- 0
for (i in 1:linhas)
if (M[i, index] == j) {
sx <- sx + M[i,1]
sy <- sy + M[i,2]
contador <- contador + 1
}
novoCentro[j,1] <- sx / contador
novoCentro[j,2] <- sy / contador
}
4
flag <- TRUE
for (j in 1:k)
if ((novoCentro[j,1] != centro[j,1]) | (novoCentro[j,2] != centro[j,2]))
flag <- FALSE
if (flag)
return(M)
centro <- novoCentro
}
}
#funcao para estimativa da densidade de 2 variaveis
pdf2var<-function(x,y,u1,u2,s1,s2,p) {(1/(2*pi*s1*s2*sqrt(1-(p^2))))*
exp((-(1)/(2*(1-(p^2))))*((((x-u1)^2)/((s1)^2))+(((y-u2)^2)/((s2)^2))-
((2*p*(x-u1)*(y-u2))/(s1*s2))))
}
sqr<-function(x) {x*x}
\end{verbatim}


\vspace{3mm}

Out of charity, there is no salvation at all. With charity, there is evolution.

\vspace{3mm}

Vinicius Claudino FERRAZ, 21/Sep/2019, Release $1.0$

\end{document}

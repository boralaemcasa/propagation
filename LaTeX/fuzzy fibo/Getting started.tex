\documentclass[11pt]{article}
\usepackage{amsmath}
\usepackage{amssymb} %mathbb
\usepackage{graphicx}
\usepackage{hyperref}
\usepackage[latin1]{inputenc}
\usepackage[top=1.0cm,bottom=1.3cm,left=1.0cm,right=1.0cm]{geometry}

\begin{document}

\Large

\begin{center}
Fuzzy Systems | Fibonacci
\end{center}

\normalsize

Second derivative method. Why not third derivative?

\begin{align}
y(k) &= y(k-1) - y(k-2) \\
y(k+2) &= y(k+1) - y(k) \\
f'(a) &\sim \cfrac{f(a + h) - f(a)}{h} \Rightarrow f(a + h) \sim f(a) + h f'(a) + \cfrac{h^2}{2}\cdot f''(a) \\
y(k + 1) &\sim y(k) + y'(k) + \cfrac{1}{2}\cdot y''(k) \\
y(k + 2) &\sim y(k) + 2 y'(k) + 2 y''(k) \\
y(k) + 2 y'(k) + 2 y''(k) &= y(k) + y'(k) + \cfrac{1}{2}\cdot y''(k) - y(k) \\
(2 - 0.5) y''(k) &= - y(k) + (1 - 2) y'(k) \\
3 y''(k) + 2 y'(k) + 2 y(k) &= 0 \\
3 \lambda^2 + 2 \lambda + 2 &= 0 \\
\lambda &= \cfrac{-1 \pm i \sqrt{5}}{3} \\
y(k) &= A e^{\lambda_1 k} + B e^{\lambda_2 k} \\
y(1) &= 1 \Rightarrow A \underbrace{e^{\lambda_1}}_u + B \underbrace{e^{\lambda_2}}_v = 1 \Rightarrow Au + Bv = 1 \\
y(2) &= 1 \Rightarrow A e^{2 \lambda_1} + B e^{2 \lambda_2} = 1 \Rightarrow Au^2 + Bv^2 = 1 \\
A &= \cfrac{v^2 - v}{uv^2 - u^2v} \\
B &= \cfrac{u - u^2}{uv^2 - u^2v} \\
y(k) &= \cfrac{v^2 - v}{uv^2 - u^2v} \cdot u^k + \cfrac{u - u^2}{uv^2 - u^2 v} \cdot v^k \\
  u &= \exp \cfrac{-1 + i \sqrt 5}{3} = e^{-1/3} \cos \theta + i e^{-1/3} \sin \theta \\
  \theta &= \sqrt 5 / 3 \\
  v &= \overline{u} \\
  u^2 &=  e^{-2/3} \cos 2\theta + i e^{-2/3} \sin 2\theta \\
  uv^2 &= e^{-1/3 - 2/3} \cos \cfrac{\sqrt 5 - 2 \sqrt 5}{3} - i e^{-1} \sin \theta \\
  u^2 v &=  e^{-2/3 - 1/3} \cos \cfrac{2 \sqrt 5 - \sqrt 5}{3} + i e^{-1} \sin \theta \\
1/D &= \cfrac{ei}{2 \sin \theta} \\
  v^2-v &= e^{-2/3} \cos 2\theta - i e^{-2/3} \sin 2\theta - e^{-1/3} \cos \theta + i e^{-1/3} \sin \theta \\
  A &= i \bigg(  \cfrac{e^{1/3}}{2 \sin \theta} \cos 2\theta   - \cfrac{e^{2/3}}{2} \cot \theta \bigg) + \cfrac{e^{1/3}}{2 \sin \theta} \sin 2\theta - \cfrac{ e^{2/3}}{2 } = p + iq \\
  u^k &= e^{-k/3} \cos k\theta + i e^{-k/3} \sin k\theta = \alpha + i \beta \\
u - u^2 &= e^{-1/3} \cos \theta + i e^{-1/3} \sin \theta -  e^{-2/3} \cos 2\theta - i e^{-2/3} \sin 2\theta
\end{align}

\begin{align}
B &= i \bigg( \cfrac{e^{2/3}}{2} \cot \theta - \cfrac{e^{1/3}}{2 \sin \theta} \cos 2\theta \bigg) + \cfrac{e^{1/3}}{2 \sin \theta} \sin 2\theta  - \cfrac{e^{2/3}}{2} \\
  B &= p - iq \\
  v^k &= \alpha - i \beta \\
  Au^k + Bv^k &= (\alpha + i \beta)(p + iq) + (\alpha - i \beta)(p - iq) =2 \alpha p - 2 \beta q \\
  y(k) &= e^{-k/3} \cos k\theta ( 2e^{1/3} \cos \theta - e^{2/3} ) - e^{-k/3} \cfrac{\sin k\theta}{\sin \theta} (  e^{1/3} \cos 2\theta - e^{2/3} \cos \theta )
\end{align}

Compare with \href{https://wikimedia.org/api/rest\_v1/media/math/render/svg/bf676e167e853211636ae5862890a08ae78cb10a}{\underline{wikipedia}}. Every derivative is not zero.

		\begin{align}
F_n &= \cfrac{1}{\sqrt 5} \bigg( \cfrac{1 + \sqrt 5}{2} \bigg)^n - \cfrac{1}{\sqrt 5} \bigg( \cfrac{1 - \sqrt 5}{2} \bigg)^n
		\end{align}

\section{RLC-Series Controller | Quintic Order}

\begin{align}
V &= Ri + L i' + \cfrac{1}{C} \int_{t_0}^t i(\tau)\,\mathrm{d}\tau \\
V' &= \cfrac{i}{C} + Ri' + L i''  \\
u(t) &= c_0 i(t) + c_1 i'(t) + c_2 i''(t) \text{ has order 2} \\
u &= c_0 i + c_1 i' + c_2 i'' + c_3 i^{(3)} + c_4 i^{(4)} + i^{(5)} \text{ has five impedances}
\end{align}

\textbf{Theorem:} $\forall \epsilon > 0, \exists I = (I_1, I_2), \exists f, \exists g\,;\,M \in I \Rightarrow f(M) < \alpha < g(M) \Rightarrow |i - i_{ref}| < \epsilon$.

\textbf{Proof:}$\,\,\blacktriangleright$

By variation of parameters,

\begin{align}
i &= y_1 \\
y_1' &= y_2 \\
y_2' &= y_3 \\
y_3' &= y_4 \\
y_4' &= y_5 \\
y_5' + c_0 i + c_1 i' + c_2 i'' + c_3 i^{(3)} + c_4 i^{(4)} &= u \\
Y + P\cdot Y &= Q = u \hat e_5 \\
Y &= e^{-tP} \int_{t_0}^t e^{\tau P} u \hat e_5\,\mathrm{d}\tau + \underbrace{y(t_0) + \cdots + y^{(4)}(t_0) (t - t_0)^4/4!}_0 \\
\exp \tau P &\to 5 \times 5: \exp (VDV^{-1}) = V (\exp D) V^{-1} \\
\sim \cdot \hat e_5 &\to \text{ just take the fifth column, times u} \\
\int &\to \text{ just get 5 functions of }t - t_0 \\
e^{-tP}\cdot \sim &\to \text{ left multiply by }5 \times 5 \\
i(t) = \mathrm{d}x (Y) &\to \text{ just take the first line}
\end{align}

\vspace{100mm}

By Riemann's below series,

\begin{align}
  i(u)(k) &= \hat e_1^\top e^{-(a + hk)P}\sum_{n > 0}^k e^{(a + hn)P} u(n) \hat e_5 h \\  \mu &:= \mu_1 \Rightarrow \mu_2 = 1 - \mu \\
  i_{ref}(\mathbb{Z}) &= \text{any limited function} \Rightarrow |i_{ref}| < c_5 \\
  e &= i - i_{ref} \\
  e(k) &\le - E_1 \Rightarrow \mu(k) = 1 \\
  e(k) &\ge E_2 \Rightarrow \mu(k) = 0 \\
  - E_1 &\le e(k) \le E_2 \Rightarrow \mu(k) = p e(k) + q \\
  u_{ref} &= c_0 i_{ref} + c_1 i'_{ref} + c_2 i''_{ref} + c_3 i^{(3)}_{ref} + c_4 i^{(4)}_{ref} + i^{(5)}_{ref} \\
  \Delta u &= u - u_{ref} \\
u(k+1) &= \mu_k q_1(k+1) + (1 - \mu_k) q_2(k+1) \\
&= \mu_k \sum_n \alpha \mu_n e + (1 - \mu_k) \sum_n \alpha (1 - \mu_n) e \\
u(i)(k+1) &= \sum_{n = 0}^k \alpha \varphi(n,k) [i(n) - i_{ref}(n)] \\
\varphi(n, k) &= 2 \mu_n \mu_k + 1 - \mu_k - \mu_n
\end{align}

Equation 51 goes from u to i. We combine with equation 62, which goes from i to u.

\begin{align}
i &\to i(k+ 1) = \hat e_1^\top e^{-(a + hk + h)P}\cfrac{1}{L}\sum_{m > 0}^{k + 1} e^{(a + hm)P}  \sum_{n = 0}^{m - 1} \alpha \varphi(n,m-1) [i(n) - i_{ref}(n)] \hat e_5 h \\
u &\to u(k + 1) = \sum_{m = 0}^k \alpha \varphi(m,k) \bigg[\hat e_1^\top e^{-(a + hm)P}\sum_{n > 0}^m e^{ (a + hn)P} u(n)\hat e_5 h - i_{ref}(m)\bigg]
\end{align}

We want direct ratio, that is:

\begin{align}
  \cfrac{|i - i_{ref}|}{\alpha} &< M \\
  \left\vert \hat e_1^\top e^{-(a + hk + h)P}\sum_{m > 0}^{k + 1} e^{ (a + hm)P}  \sum_{n = 0}^{m - 1} \underbrace{\varphi(n,m-1)}_{\le 3} [i(t_n) - i_{ref}(t_n)] \hat e_5 h - \cfrac{1}{\alpha} \cdot i_{ref}(t_{k+1}) \right\vert &\le  \\
  \left\vert 3h \hat e_1^\top e^{-(a + hk + h)P}\sum_{m > 0}^{k + 1} e^{(a + hm)P} \hat e_5 \underbrace{\sum_{n = 0}^{m - 1} [i(t_n) - i_{ref}(t_n)]}_{<\text{ integral}} - \cfrac{1}{\alpha} \cdot i_{ref}(t_{k+1}) \right\vert &\le
\end{align}

\footnotesize

\begin{align}
  \left\vert 3h \hat e_1^\top e^{- (a + hk + h)P}\sum_{m > 0}^{k + 1} e^{(a + hm)P}\hat e_5  \left\{ i(a) - i_{ref}(a) + \int_a^{a + h(m - 1)} \underbrace{[i(t) - i_{ref}(t)]}_{\text{Fund.T.Calculus}} \mathrm{d}t \right\} - \cfrac{1}{\alpha} \cdot i_{ref}(t_{k+1}) \right\vert &=  \\
  \left\vert 3h \hat e_1^\top e^{- (a + hk + h)P}\sum_{m > 0}^{k + 1} e^{ (a + hm)P}\hat e_5  \left\{ i(a) - i_{ref}(a) + [I(a + hm - h) - I_{ref}(a + hm - h)] - [I(a) - I_{ref}(a)] \right\} - \cfrac{1}{\alpha} \cdot i_{ref}(t_{k+1}) \right\vert &=
\end{align}

\normalsize

\begin{align}
  \left\vert 3h \hat e_1^\top e^{- (a + hk + h)P}\underbrace{\sum_{m > 0}^{k + 1} e^{(a+hm)P}\hat e_5 [p(m) + q]}_{<\text{ integral}}  - \cfrac{1}{\alpha} \cdot i_{ref}(t_{k+1}) \right\vert &\le  \\
  \left\vert 3h \hat e_1^\top e^{- (a + hk + h)P} \underbrace{\int_0^{k + 2} e^{(a+hm)P}\hat e_5 [p(m) + q] \,\mathrm{d}m}_{\text{T.Fund.C\'alculo}}  - \cfrac{1}{\alpha} \cdot i_{ref}(t_{k+1}) \right\vert &=  \\
  \left\vert \underbrace{3h \hat e_1^\top e^{- (a + hk + h)P} [V(k+2) - V(0)]}_A  - \cfrac{1}{\alpha} \cdot \underbrace{i_{ref}(t_{k+1})}_B \right\vert &< M
\end{align}

\begin{align}
  - M < A - \cfrac{B}{\alpha} &< M \\
  - M\alpha < A\alpha - B &< M \alpha \\
  (- M - A)\alpha < - B &< (M - A) \alpha \\
  \cfrac{B}{A + M} &< \alpha < \cfrac{B}{A - M}
\end{align}

Our function is quite simple.

\begin{align}
 f(x) &= A - \cfrac{B}{x} \\
 f(0+) &= - \infty \\
 f(\infty) &= A \\
 f(x) &= - M \Rightarrow Ax - B = - Mx \therefore x_{min} = \cfrac{B}{A + M} \\
 f\bigg(\cfrac{B}{A - M}\bigg) &= M \Leftarrow M < A \Rightarrow x_{max} = \cfrac{B}{A - M} \\
 \\
 M \ge A &\Rightarrow x_{max} = \infty
\end{align}

Now, at line 66, $\epsilon = M\alpha $. We start by FOR ALL $\epsilon > 0$, EXISTS $\alpha(\epsilon) > 0$. We want $M(\epsilon) > 0$.

\begin{align}
 \alpha &= \cfrac{\epsilon}{M} > \cfrac{B}{A + M} \\
 \cfrac{BM - \epsilon(A+M)}{A+M} &< 0 \\
 M_1 &= - A \,;\, M_2 = \cfrac{A \epsilon}{B - \epsilon}
\end{align}

When $B - \epsilon > 0$, the parabola stands up. $M$ is between $M_1$ and $M_2$.

\begin{align}
 \alpha &= \cfrac{\epsilon}{M} < x_{max} \\
 M &\ge A \Rightarrow \text{end of restrictions} \\
 M &< A \Rightarrow \cfrac{BM - \epsilon(A - M)}{A - M} > 0
\end{align}

\begin{align}
 M_3 &= A \,;\, M_4 = \cfrac{A}{B + \epsilon}
\end{align}

When $B + \epsilon > 0$, the parabola reverts down. $M$ is between $M_3$ and $M_4$.

The conclusion is that the maximum output error $M$ is a function of the order of magnitude of $\epsilon$. At Platonic world, there is the limit beyond and less than Quantum Physics. $\,\,\blacksquare$

It's controlled: $y^{(n)} = p_0 y + p_1 y' + \cdots + p_{n-1} y^{(n-1)} + u(t)$, if $p_i$ are constants and if $n < \infty$.

\subsection{MatLab for 5th order}

\begin{verbatim}
under construction
\end{verbatim}

\section{RL with Perturbations}

\begin{align}
 u &= Ri + Li' \\
 i' + \cfrac{R}{L}\cdot i &= \cfrac{u}{L} \\
 i(R, L) &= \cfrac{1}{L}\cdot e^{-Rt/L} \int_{t_0}^t e^{R\tau/L} u(\tau)\,\mathrm{d}\tau \\
 \exp 1/L &= \beta\,;\,\exp 1/(L \cdot \Delta L) = \gamma = \beta^{\Delta L} \\
 i(R, L) &= \cfrac{1}{L\beta^{Rt} }\cdot \int_{t_0}^t \beta^{R\tau} u(\tau)\,\mathrm{d}\tau = i_0 \\
 i(R \cdot \Delta R, L \cdot \Delta L) &= \cfrac{1}{L}\cdot \gamma^{-Rt} \gamma^{-\Delta Rt} \int_{t_0}^t \gamma^{R\tau} \gamma^{\Delta R\tau} u(\tau)\,\mathrm{d}\tau  = i_1 \\
 \cfrac{i_1}{i_0} &= \bigg( \cfrac{\gamma}{\beta}\bigg) ^{-Rt} \gamma^{-\Delta Rt} \cfrac{\int_{t_0}^t \gamma^{R\tau} \gamma^{\Delta R\tau} u(\tau)\,\mathrm{d}\tau}{  \int_{t_0}^t \beta^{R\tau} u(\tau)\,\mathrm{d}\tau} \\
 \cfrac{i_1}{i_0} &= ( \beta^{t} )^{R - R\Delta L - \Delta R\Delta L}\cdot \cfrac{\int_{t_0}^t \beta^{R\tau \Delta L} u(\tau)\,\mathrm{d}\tau\, \beta^{\Delta R\Delta L\tau} }{  \int_{t_0}^t \beta^{R\tau} u(\tau)\,\mathrm{d}\tau}
\end{align}

\subsection{MatLab for almost constant 1st order}

\begin{verbatim}
/*******
 *
 * fun.m
 *
 *******/
function y = fun(x,a,b)

g = @(x) exp(1./((x-a).*(x-b))) ;

 if (x <= a)
     y = 0;
 elseif (x >= b)
     y = quadgk(g,a,b);
 else
     y = quadgk(g,a,x);
 end
end

/*************
 *
 * principal.m
 *
 *************/
close all;
clc;
clear all


a = 1;
b = 3;
m1 = 0;
M2 = 2;
R = 3;
L = 5;
emin = -1;
emax = 1;
delta = 0.001;
x=(a:delta:8*b-delta)' ;
epsilon = 0.00008;


p = -1/(emax-emin);

u(1) = 0;
ulinha(1) = 0;
q1(1) = 0;
q2(1) = 0;
soma = 0;
somalinha = 0;
alfa(1) = 42000;
funfinal = fun(b,a,b);
temp = 0;

for k =1:length(x)
    if (k > 1)
        termo = exp(R*x(k)/L)* delta * u(k)/alfa(k-1);
        somalinha = somalinha + termo;
    end

    termo = exp(-R*x(k)/L)/L;
    ilinha(k) = termo * somalinha;

    const = 1 / funfinal * (M2 - m1);

     if (x(k) <= a)
         temp = 0;
     elseif (x(k) >= b)
         temp = funfinal;
     else
         temp = temp + exp(1./((x(k)-a).*(x(k)-b))) * delta;
     end

    iref(k) = temp * const + m1;

    aa = ilinha(k);
    bb = iref(k);

    mm1 = aa * epsilon / (bb + epsilon);
    mm2 = aa * epsilon / (bb - epsilon);
    if mm1 > mm2
      mm3 = mm1;
      mm1 = mm2;
      mm2 = mm3;
    end
    mm3 = (mm1 + mm2)/2;
    mm4 = mm1 * 1.01;
    mm5 = mm2 * 1.01;

    if abs(aa - bb * mm3/epsilon) < mm3
      mm = mm3;
    elseif abs(aa - bb * mm4/epsilon) < mm4
      mm = mm4;
    elseif abs(aa - bb * mm5/epsilon) < mm5
      mm = mm5;
    else
      mm = epsilon;
    end

    alfa(k) = epsilon/mm;
    if alfa(k) > 42000
      alfa(k) = 42000;
    elseif alfa(k) < 39915
      alfa(k) = 39915;
    end

    % calcular i(k) através da eq. do circuito

    % i = exp(-Rt/L)/L * integral exp(R*t/L) * u dt

    if (k > 1)
        termo = exp(R*x(k)/L)* delta * u(k)/alfa(k);
        soma = soma + termo * alfa(k);
    end

    termo = exp(-R*x(k)/L)/L;
    i(k) = termo * soma;

    % calcular o erro
    e(k) =iref(k) - i(k);

    % calcular mi1

    if e(k) <= emin
        mi1 = 1;
    elseif e(k) >= emax
        mi1 = 0;
    else mi1 = p* e(k) + 0.5;
    end

    mi2=1-mi1;

    % resolva em m: - M < A - BM/epsilon < M
    %               BM/epsilon - M < A < BM/epsilon + M
    %               A/~ < M < A/~

    q1(k+1)=q1(k) + alfa(k)*mi1*e(k);
    q2(k+1)=q2(k) + alfa(k)*mi2*e(k);

    if (k < length(x))
        u(k+1) = mi1*q1(k+1) + mi2*q2(k+1);
    end

    % u = Ri + L di\dt

     if (x(k) <= a)
         z = 0;
     elseif (x(k) >= b)
         z = 0;
     else
       z = exp(1./((x(k)-a).*(x(k)-b)));
     end
     uref(k) = R * iref(k) + L * z * const;
end

figure
plot(x,e)

figure
plot(x,alfa)
\end{verbatim}

\subsection{Sinoidal Perturbation}

\begin{align}
  u &= (R_x + R_y \sin \omega t) i + (L_x + L_y \sin \varphi t) i' \\
  i' + \cfrac{R(t)}{L(t)}\cdot i &= \cfrac{u(t)}{L(t)} \\
  P(t) &= \int_{t_0}^t \cfrac{R_x + R_y \sin \omega \tau}{L_x + L_y \sin \varphi \tau}\,\mathrm{d}\tau \\
  i(t) &= e^{-P(t)} \int_{t_0}^t e^{P(\tau)}\, \cfrac{u(\tau)}{L(\tau)}\,\mathrm{d}\tau
\end{align}

\section{Follower Robot | The Missile Problem}

\begin{flushleft}
Input: a robot named Self is at Cartesian Physical cohordinates $(s_x(k), s_y(k), s_z(k))$ \\
Inputs: speed $(v_x, v_y, v_z)$ and acceleration $(a_x, a_y, a_z)$. \\
Input: The followed object named M is at $m_x(k), m_y(k), m_z(k)$. \\
Restrictions: $|s'(\mathbb{R})| \le 1\,;\,|s''(\mathbb{R})| \le 1$. We don't have how to preview M's next move. \\
We want to minimize $|s - m|$. \\
Outputs: Self's $(k + 1)$ new position, $(k)$ new speed and new acceleration.
\end{flushleft}

\begin{align}
 |s(0) - m(0)| &= f(0) > 0 \\
 X''(t) &\sim a(0) \\
 X'(t) &\sim v(0) + t a(0) \\
 \text{line from }s(0)\text{ to }m(0): X(t) &\sim s(0) + t v(0) + \cfrac{t^2}{2}\cdot a(0) = s(0) + \lambda[ m(0) - s(0) ] \\
 t = 1 &\to \text{ we want at zero } v_x, v_y, v_z, a_x, a_y, a_z, \lambda \text{ then finallly }s(1)
\end{align}

\begin{align}
 v_x + a_x/2 &= \lambda (m_x - s_x) \\
 v_y + a_y/2 &= \lambda (m_y - s_y) \\
 v_z + a_z/2 &= \lambda (m_z - s_z) \\
v_x^2 + v_y^2 + v_z^2 &\le 1 \\
 a_x^2 + a_y^2 + a_z^2 &\le 1 \\
 \text{Difference } D &= (s_x, s_y, s_z) + \lambda(m_x - s_x, m_y - s_y, m_z - s_z) - (m_x, m_y, m_z) \\
 D_x &= s_x + \lambda(m_x - s_x) - m_x \\
 D_y &= s_y + \lambda(m_y - s_y) - m_y \\
 D_z &= s_z + \lambda(m_z - s_z) - m_z \\
\text{Minimize }J(\lambda) &= D_x^2 + D_y^2 + D_z^2 \\
 t = k+1 &\to \text{ we want at k } v_x, v_y, v_z, a_x, a_y, a_z, \lambda \text{ then finallly }s(k+1) \\
 k &= 1, 2, 3, \cdots
\end{align}

		\vspace{3mm}

Out of charity, there is no salvation at all. With charity, there is evolution.

\vspace{3mm}

Vinicius Claudino FERRAZ, 26/Oct/2019, Release $1.0.5$

\end{document}

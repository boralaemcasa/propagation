\documentclass[11pt]{article}
\usepackage{amsmath}
\usepackage{amssymb} %mathbb
\usepackage{graphicx}
\usepackage{hyperref}
\usepackage{colortbl}
\usepackage[latin1]{inputenc}
\usepackage[top=1.0cm,bottom=1.3cm,left=1.0cm,right=1.0cm]{geometry}

\begin{document}

\Large

\begin{center}
\href{https://www.youtube.com/watch?v=BKfWfz4Bah8&list=UUdcL3nma0HpYkWFH52YE06Q}{\color{blue}\underline{Sistemas Nebulosos 7 | Resum\~ao \& E.D.O.}}
\end{center}

\normalsize

\vspace{3mm}

\section{Preliminares}

Caso $B - \epsilon > 0$, a par\'abola vira para cima. $M$ est\'a entre $M_1$ e $M_2$.

\begin{align}
 \alpha &= \cfrac{\epsilon}{M} < x_{max} \\
 M &\ge A \Rightarrow \text{acabaram as restri\c{c}\~oes} \\
 M &< A \Rightarrow \cfrac{BM - \epsilon(A - M)}{A - M} > 0 \\
 M_3 &= A \,;\, M_4 = \cfrac{A}{B + \epsilon}
\end{align}

Caso $B + \epsilon > 0$, a par\'abola vira para baixo. $M$ est\'a entre $M_3$ e $M_4$.

A conclus\~ao \'e que o erro m\'aximo de sa\'ida $M$ \'e fun\c{c}\~ao da ordem de grandeza $\epsilon$. No mundo Plat\^onico, existe o limite que vai al\'em da f\'isica qu\^antica.

Est\'a controlado: $y' = py + u(t)$, se $p$ for constante.

Est\'a controlado: $y'' = p_0 y + p_1 y' + u(t)$, se $p_i$ forem constantes.

Est\'a controlado: $y^{(n)} = p_0 y + p_1 y' + \cdots + p_{n-1} y^{(n-1)} + u(t)$, se $p_i$ forem constantes e se $n < \infty$.

$n \to \infty$, descontrole. N\~ao usamos varia\c{c}\~ao de par\^ametros.

A ideia \'e truncar $p$ por aproxima\c{c}\~ao, por exemplo: $p(t, n) = a_0 + a_1 (t - t_0) + \cdots + a_n (t - t_0)^n$.

\section{NOVIDADE | Controlador qualquer de primeira ordem}

\begin{flushleft}
Entrada vetorial: $\vec u(t) = (u_0,u_1)(t)$ Mai\'usculo $U_0(t) = \int_0^t u(\tau) \,\mathrm{d}\tau$: antiderivada. \\
Sa\'ida escalar: $y(t)$
\end{flushleft}

Se o erro for proporcional como vimos na simula\c{c}\~ao em MatLab, basta repetir a invers\~ao. \\

Resolvemos a equa\c{c}\~ao diferencial $y' + u_0(t) y = u_1(t)$ por varia\c{c}\~ao de par\^ametros.

A equa\c{c}\~ao caracter\'istica \'e $\lambda = u_0$.

\begin{align}
 y(t) &= e^{-U_0(t)} \int_{t_0}^t e^{U_0(\tau)} U_1(\tau)\, \mathrm{d}\tau + y(t_0)
\end{align}

Pela soma de Riemann,

\begin{align}
  \overline{y}(k) &= e^{-U_0(a + hk)} \sum_{n > 0}^k e^{U_0(n)} U_1(n)\, h + \overline{y}(0)  \\
  \mu &:= \mu_1 \Rightarrow \mu_2 = 1 - \mu \\
  y_{ref}(\mathbb{Z}) &= \text{degrau restrito} \\
  e &= y - y_{ref} \\
  e(k) &\le - E_1 \Rightarrow \mu(k) = 1 \\
  e(k) &\ge E_2 \Rightarrow \mu(k) = 0 \\
  - E_1 &\le e(k) \le E_2 \Rightarrow \mu(k) = p e(k) + q \\
  y_{ref}' &= u^0_{ref} y_{ref} + u^1_{ref}
  \end{align}

Queremos isolar $u_{ref}$ na express\~ao acima. Facilmente:

\begin{align}
  u_{ref} &= u^0_{ref}(1,-y_{ref}) + u^1_{ref}(0,y'_{ref}) \\
  \Delta u_0 &= u_0 - u^0_{ref} \\
  \Delta u_1 &= u_1 - u^1_{ref} \\
\overline{u}(k+1) &= \vec\mu_k q_1(k+1) + (1 - \vec\mu_k) q_2(k+1) \\
&= \mu_k \sum_n \alpha \mu_n e + (1 - \mu_k) \sum_n \alpha (1 - \mu_n) e \\
\overline{u}(k+1) &= \sum_{n = 0}^k \alpha_n  [\overline{y}(n) - \overline{y_{ref}}(n)] \vec\varphi(n,k) \\
\varphi_0(n, k) &= 2 \mu_n^0 \mu_k^0 + 1 - \mu_k^0 - \mu_n^0 \\
\varphi_1(n, k) &= 2 \mu_n^1 \mu_k^1 + 1 - \mu_k^1 - \mu_n^1 \\
\overline{u}(\ell) &= \sum_{n = 0}^{\ell - 1} \alpha_n [\overline{y}(n) - \overline{y_{ref}}(n)] \varphi_{01} (n, \ell-1) \\
\overline{U}(m) &= h \sum_{\ell = 0}^{m} \overline{u}(\ell) \\
\overline{y} &\to \overline{y}(k+ 1) = h e^{- h \sum_{\ell = 0}^{k+1} \overline{u_0}(\ell)} \sum_{m > 0}^{k+1} e^{h \sum_{\ell = 0}^{m} \overline{u_0}(\ell)} \cdot \bigg[ h \sum_{\ell = 0}^{m} \overline{u_1}(\ell) \bigg] + \overline{y}(0) = \nonumber
\end{align}

\begin{align}
\overline{y}(k+ 1) &= h e^{- h \sum_{\ell = 0}^{k+1} \sum_{n = 0}^{\ell - 1} \alpha_n [\overline{y}(n) - \overline{y_{ref}}(n)] \varphi_{0} (n, \ell-1)} \sum_{m > 0}^{k+1} e^{h \sum_{\ell = 0}^{m} \sum_{n = 0}^{\ell - 1} \alpha_n [\overline{y}(n) - \overline{y_{ref}}(n)] \varphi_{0} (n, \ell-1)} \cdot \nonumber \\
&\cdot \bigg[ h \sum_{\ell = 0}^{m} \sum_{n = 0}^{\ell - 1} \alpha_n [\overline{y}(n) - \overline{y_{ref}}(n)] \varphi_{1} (n, \ell-1) \bigg] + \overline{y}(0) \\
u &\to u(k + 1) = \sum_{m = 0}^k \alpha_m \bigg[ e^{-U_0(a + hm)} \sum_{n > 0}^m e^{U_0(n)} U_1(n)\, h + \overline{y}(0) - \overline{y_{ref}}(m) \bigg] \varphi_{01} (m, k)
\end{align}

H\'a 2 pend\^encias:

\begin{align}
|u_0 - u_{ref}^0| &< \delta_0 \\
|u_1 - u_{ref}^1| &< \delta_1 \\
  |&\overline{y}(k+1) - \overline{y_{ref}}(k+1)| = \\
\bigg| &h e^{- h \sum_{\ell = 0}^{k+1} \sum_{n = 0}^{\ell - 1} \alpha_n [\overline{y}(n) - \overline{y_{ref}}(n)] \varphi_{0} (n, \ell-1)} \sum_{m > 0}^{k+1} e^{h \sum_{\ell = 0}^{m} \sum_{n = 0}^{\ell - 1} \alpha_n [\overline{y}(n) - \overline{y_{ref}}(n)] \varphi_{0} (n, \ell-1)} \cdot \nonumber \\
&\cdot \bigg[ h \sum_{\ell = 0}^{m} \sum_{n = 0}^{\ell - 1} \alpha_n [\overline{y}(n) - \overline{y_{ref}}(n)] \varphi_{1} (n, \ell-1) \bigg] + \overline{y}(0) - \overline{y_{ref}}(k+1) \bigg| = \sigma =
\end{align}

\begin{align}
\bigg| &h e^{- F_0(\vec \alpha, k+1)} \sum_{m > 0}^{k+1} e^{F_0(\vec \alpha, m)} \cdot F_1(\vec \alpha, m) + \overline{y}(0) - \overline{y_{ref}}(k+1) \bigg| = \\
\bigg| &h F_2( \vec \alpha, k+1) + \overline{y}(0) - \overline{y_{ref}}(k+1) \bigg| \le \\
&h | F_2( \vec \alpha, k+1) | + M_1 \le \\
&hM_2 + M_1 = M_3
\end{align}

Para algum alfa, $|F_2( \vec \alpha, k+1)| < M_2$. Qual? Outra tentativa \'e $\lim |e| = 0$, ou seja:

\begin{align}
  |&\overline{e}(k+1) - \overline{e}(k)| = \\
  |&\overline{y}(k+1) - \overline{y_{ref}}(k+1) - [\overline{y}(k) - \overline{y_{ref}}(k)]| \le \\
  | &h F_2( \vec \alpha, k+1) + \overline{y}(0) - \overline{y_{ref}}(k+1) - h F_2( \vec \alpha, k) - \overline{y}(0) + \overline{y_{ref}}(k) | \le \\
  &h |F_3(\alpha, k +1)| + M_4 \le \\
  &hM_5 + M_4 = M_6
\end{align}

Para algum alfa, $|F_2( \vec \alpha, k+1) - F_2( \vec \alpha, k)| < M_5$. Qual?

$\alpha(k)$ pode ser anal\'itica. A ideia \'e truncar $\alpha$ por aproxima\c{c}\~ao, por exemplo: $\alpha(k, n) = a_0 + a_1 k + \cdots + a_n k^n$.

Na linha (71), h\'a $2+3+3$ somat\'orias. Tomemos, grosseiramente, o m\'odulo de cada parcela.

\begin{align}
  &h |e^{- h \alpha_1 \varphi_1 y_1}e^{ h \alpha_2 \varphi_2 y_2} h \alpha_3 \varphi_3 y_3| \le h |e^{- 3h \alpha_1 y_1}e^{ 3h \alpha_2 y_2} 3h \alpha_3 y_3| = M_7
\end{align}

Exclu\'imos $\alpha$ negativo. Sempre que $e^{-\alpha_1} > 1 \Leftrightarrow e^{\alpha_2} < 1$. O m\'inimo \'e $1 = e^0$. Substitu\'imos $\alpha_1 := 0$.

\begin{align}
  M_7 &\le h |e^{ 3h \alpha_2 y_2} 3h \alpha_3 y_3| = h | \beta e^{\beta} |\,;\, \beta = 3h \alpha y
\end{align}

A sa\'ida deve divergir. Nesse caso, $\alpha \in \emptyset$.

Sen\~ao, \'e poss\'ivel usar a fun\c{c}\~ao W de Lambert: $z e^z = w = W(w) \exp W(w) \Leftrightarrow z = W(w)$.

\begin{align}
  | \beta e^{\beta} | &\le \cfrac{M_8}{h}  \\
  w_1 e^{w_1} = - w e^w \le \beta e^{\beta} &\le w e^w \\
  z_1 \le \beta &\le z_2 \\
  z_1 \le 3h \alpha y &\le z_2 \\
  \cfrac{z_1}{3h\inf y} \le 0 \le \alpha &\le \cfrac{z_2}{3h\sup y} \\
  \sigma &\le | \beta e^{\beta} | (1+2+\cdots+k+1)[1 + (1 + 2) + (1 + 2 + 3) + \cdots + (1 + 2 + \cdots + k + 1)]^2 \\
  &\le \cfrac{M_8}{h} (k + 2)^{2+2+3}
\end{align}

Esta se\c{c}\~ao independe da dimens\~ao interna. Vale para qualquer E.D.O. sol\'uvel no lugar de (47).

Para o caso da dimens\~ao externa $D = \dim \vec y$, suponha que $y_i$ independe de $y_j, 1 \le i \ne j \le D$ e controle cada coordenada.

\section{EDO de ordem 1 | Quadr\'atica}

\begin{align}
 y' &= p_0 +  p_1 y + p_2 y^2 \\
 y &= \sum_0 a_i x^i \\
 p_1 &= \sum_0 b_i x^i \\
 p_1 y &= \sum_{k = 0}^{\infty} \sum_{i = 0}^k a_i b_{k - i} x^k \\
 y' &= \sum_0 (i + 1) a_{i + 1} x^i \\
 y^2 &= \sum_0 a_i x^{2i} + 2 \sum_{k = 0}^{\infty} \underbrace{\sum_{i = 0}^{\lceil k/2 - 1 \rceil} a_i a_{k - i}}_{f(k)} x^k \\
  p_2 &= \sum_0 c_i x^i \\
 p_2 y^2 &= \sum_{k = 0}^{\infty} \sum_{i = 0}^k a_i c_{k - 2i} x^k + 2 \sum_{L = 0}^{\infty}  \sum_{j = 0}^L c_j f(L-j) x^L \\
 p_0 &= \sum_0 d_i x^i \\
 \sum_0 (i + 1) a_{i + 1} x^i &= \sum_0 d_i x^i + \sum_{k = 0}^{\infty} \sum_{i = 0}^k a_i b_{k - i} x^k + \sum_{k = 0}^{\infty} \sum_{i = 0}^k a_i c_{k - 2i} x^k + 2 \sum_{L = 0}^{\infty}  \sum_{j = 0}^L c_j f(L-j) x^L
 \end{align}

 \begin{align}
 1a_1 &= d_0 + a_0 b_0 + a_0 c_0 \\
 2 a_2 &= d_1 + a_0 b_1 + a_1 b_0 + a_0 c_1 + 2 c_0 a_0 a_1 \\
 3 a_3 &= d_2 + a_0 b_2 + a_1 b_1 + a_2 b_0 + a_0 c_2 + a_1 c_0 + 2 c_0 a_0 a_2 + 2 c_1 a_0 a_1
 \end{align}

\section{EDO de ordem 1 | C\'ubica}

Seja $x(t) = t - t_0$. Todas as nossas s\'eries de Taylor est\~ao definidas em $t_0 - \delta < t < t_0 + \delta$.

\begin{align}
 y' &= p_0 +  p_1 y + p_2 y^2 + p_3 y^3 \\
 y &= \sum_0 a_i x^i \\
 p_0 &= \sum_0 b_i x^i \\
 p_1 &= \sum_0 c_i x^i \\
 p_1 y &= \sum_{k = 0}^{\infty} \sum_{i = 0}^k a_i c_{k - i} x^k \\
 y' &= \sum_0 (i + 1) a_{i + 1} x^i \\
 y^2 &= \sum_0 a_i x^{2i} + \sum_{k = 0}^{\infty} \underbrace{2 \sum_{i = 0}^{\lceil k/2 - 1 \rceil} a_i a_{k - i}}_{f_2(k)} x^k \\
  p_2 &= \sum_0 d_i x^i \\
 p_2 y^2 &= \sum_{k = 0}^{\infty} \sum_{i = 0}^k a_i d_{k - 2i} x^k + \sum_{L = 0}^{\infty}  \sum_{j = 0}^L d_j f_2(L-j) x^L
\end{align}

\begin{align}
 y^3 &= \sum_0 a_i x^{3i} + \sum_{k = 0}^{\infty} \underbrace{3 \sum_{i = 0}^{\lceil k/2 - 1 \rceil} a_i^2 a_{k - i} + a_i a_{k - i}^2 + 6 \sum_{i \ne j \ne k} a_i a_j a_{k - i - j} }_{f_3(k)} x^k \\
 p_3 &= \sum_0 g_i x^i \\
  p_3 y^3 &= \sum_{k = 0}^{\infty} \sum_{i = 0}^k a_i g_{k - 3i} x^k + \sum_{L = 0}^{\infty}  \sum_{j = 0}^L g_j f_3(L-j) x^L \\
 \sum_0 (i + 1) a_{i + 1} x^i &= \sum_0 b_i x^i + \sum_{k = 0}^{\infty} \sum_{i = 0}^k a_i c_{k - i} x^k + \sum_{k = 0}^{\infty} \sum_{i = 0}^k a_i d_{k - 2i} x^k + 2 \sum_{L = 0}^{\infty}  \sum_{j = 0}^L d_j f(L-j) x^L + \nonumber \\
 &+ \sum_{k = 0}^{\infty} \sum_{i = 0}^k a_i g_{k - 3i} x^k + \sum_{L = 0}^{\infty}  \sum_{j = 0}^L g_j f_3(L-j) x^L
 \end{align}

Determinar $a(b,c,d,g)$.

\section{EDO de ordem 1 | Grau $N$}

\begin{align}
 y' &= p_0 +  p_1 y + p_2 y^2 + \cdots + p_N y^N \\
 y &= \sum_0 a_i x^i \\
 p_0 &= \sum_0 c_i(0) x^i \\
 &\vdots \\
 p_N &= \sum_0 c_i(N) x^i \\
 p_1 y &= \sum_{k = 0}^{\infty} \sum_{i = 0}^k a_i c_{k - i}(1) x^k \\
 y' &= \sum_0 (i + 1) a_{i + 1} x^i \\
 y^{\lambda} &= \sum_0 a_i x^{i\lambda} + \sum_{k = 0}^{\infty} f_{\lambda}(k) x^k \\
 p_{\lambda} y^{\lambda} &= \sum_{k = 0}^{\infty} \sum_{i = 0}^k a_i c_{k - i\lambda}(\lambda) x^k + \sum_{L = 0}^{\infty}  \sum_{j = 0}^L c_j(\lambda) f_{\lambda}(L-j) x^L \\
\sum_0 (i + 1) a_{i + 1} x^i &= \sum_0 c_i(0) x^i + \sum_{\lambda = 1}^N \sum_{k = 0}^{\infty} \sum_{i = 0}^k a_i c_{k - i\lambda}(\lambda) x^k + \nonumber \\
 & + \sum_{\lambda = 2}^N \sum_{L = 0}^{\infty}  \sum_{j = 0}^L c_j(\lambda) f_{\lambda}(L-j) x^L
 \end{align}

Determinar $a(c)$. Comece aqui e determine at\'e o fim do arquivo.

\section{EDO de ordem 1 | S\'erie}

\begin{align}
 y' &= \varphi(y) = p_0 +  p_1 y + p_2 y^2 + \cdots \\
\sum_0 (i + 1) a_{i + 1} x^i &= \sum_0 c_i(0) x^i + \sum_{\lambda = 1}^{\infty} \sum_{k = 0}^{\infty} \sum_{i = 0}^k a_i c_{k - i\lambda}(\lambda) x^k + \sum_{\lambda = 2}^{\infty} \sum_{L = 0}^{\infty}  \sum_{j = 0}^L c_j(\lambda) f_{\lambda}(L-j) x^L
 \end{align}

\section{EDO de ordem 2 | S\'erie}

\begin{align}
 y'' &= \varphi(y) + \psi(y') \\
 y' &= z \\
 z' &= \psi(z) + \varphi(y) \\
 \begin{pmatrix} y \\ z \end{pmatrix}' &= \begin{pmatrix} z \\ \psi(z) \end{pmatrix} + \begin{pmatrix} 0 \\ \varphi(y) \end{pmatrix} \\
 Y' &= \varphi(Y)
\end{align}

Ou, se preferir,

\begin{align}
 y'' &= p_0^0 + \sum p_i^0 y^i + \sum p_i^1 (y')^i  \\
 p_N^0 &= \sum_0 c_i^0(N) x^i \\
 p_N^1 &= \sum_0 c_i^1(N) x^i \\
 y &= \sum_0 a_i^0 x^i \\
 y' &= \sum_0 a_i^1 x^i \sim (i + 1)a_{i+1}^0 \\
\sum_0 (i + 1) a_{i + 1}^1 x^i &= \sum_0 c_i^0(0) x^i + \sum_{\lambda = 1}^{\infty} \sum_{k = 0}^{\infty} \sum_{i = 0}^k a_i^0 c_{k - i\lambda}^0(\lambda) x^k + \sum_{\lambda = 2}^{\infty} \sum_{L = 0}^{\infty}  \sum_{j = 0}^L c_j^0(\lambda) f_{\lambda}^0(L-j) x^L + \nonumber \\
&+ \sum_{\lambda = 1}^{\infty} \sum_{k = 0}^{\infty} \sum_{i = 0}^k a_i^1 c_{k - i\lambda}^1(\lambda) x^k + \sum_{\lambda = 2}^{\infty} \sum_{L = 0}^{\infty}  \sum_{j = 0}^L c_j^1(\lambda) f_{\lambda}^1(L-j) x^L
\end{align}

\section{EDO de ordem 3 | S\'erie}

\begin{align}
 y''' &= p_0^0 + \sum p_i^0 y^i + \sum p_i^1 (y')^i + \sum p_i^2 (y'')^i  \\
 y''' &= \varphi(y) + \psi(y') + \xi(y'') \\
 y' &= z_1 \\
 z_1' &= z_2 \\
 z_2' &= \xi(z_2) + \psi(z_1) + \varphi(y) \\
 \begin{pmatrix} y \\ z_1 \\ z_2 \end{pmatrix}' &= \begin{pmatrix}0 \\ z_2 \\ \xi(z_2) \end{pmatrix} + \begin{pmatrix} z_1 \\ 0 \\ \psi(z_1) \end{pmatrix}+ \begin{pmatrix} 0 \\ 0 \\ \varphi(y) \end{pmatrix} \\
 Y' &= \varphi(Y) \\
 \sum_0 (i + 1) a_{i + 1}^2 x^i &= \sum_0 c_i^0(0) x^i + \sum_{\theta = 0}^2 \sum_{\lambda = 1}^{\infty} \sum_{k = 0}^{\infty} \sum_{i = 0}^k a_i^{\theta} c_{k - i\lambda}^{\theta}(\lambda) x^k + \nonumber \\
 &+ \sum_{\theta = 0}^2 \sum_{\lambda = 2}^{\infty} \sum_{L = 0}^{\infty}  \sum_{j = 0}^L c_j^{\theta}(\lambda) f_{\lambda}^{\theta}(L-j) x^L
\end{align}

\section{EDO de ordem $n$ | S\'erie}

\begin{align}
y^{(n)} &= p_0^0 + \sum_{j = 0}^{n-1} \sum_{i = 1}^{\infty} p_i^j (y^{(j)})^i \\
 y^{(n)} &= \varphi_0(y) + \varphi_1(y') + \cdots + \varphi_{n-1}(y^{(n-1)}) \\
 y' &= z_1 \\
 z_1' &= z_2 \\
 &\vdots \\
 z_{n-1}' &= \varphi_{n-1}(z_{n-1}) + \cdots + \varphi_1(z_1) + \varphi_0(y)
\end{align}

\begin{align}
 Y' &= \sum_{i=1}^{n-1} z_i e_i + \sum_{i=1}^{n-1} \varphi_i(z_i) e_n + \varphi(y) e_n \\
 Y' &= \varphi(Y) \\
 \sum_0 (i + 1) a_{i + 1}^{n-1} x^i &= \sum_0 c_i^0(0) x^i + \sum_{\theta = 0}^{n-1} \sum_{\lambda = 1}^{\infty} \sum_{k = 0}^{\infty} \sum_{i = 0}^k a_i^{\theta} c_{k - i\lambda}^{\theta}(\lambda) x^k + \nonumber \\
 &+ \sum_{\theta = 0}^{n-1} \sum_{\lambda = 2}^{\infty} \sum_{L = 0}^{\infty}  \sum_{j = 0}^L c_j^{\theta}(\lambda) f_{\lambda}^{\theta}(L-j) x^L
\end{align}

\section{EDO de ordem infinita}


\begin{align}
0 &= p_0^0 + \sum_{j = 0}^{\infty} \sum_{i = 1}^{\infty} p_i^j (y^{(j)})^i \\
 0 &= \varphi_0(y) + \varphi_1(y') + \cdots \\
 0 &= \sum_0 c_i^0(0) x^i + \sum_{\theta = 0}^{\infty} \sum_{\lambda = 1}^{\infty} \sum_{k = 0}^{\infty} \sum_{i = 0}^k a_i^{\theta} c_{k - i\lambda}^{\theta}(\lambda) x^k + \sum_{\theta = 0}^{\infty} \sum_{\lambda = 2}^{\infty} \sum_{L = 0}^{\infty}  \sum_{j = 0}^L c_j^{\theta}(\lambda) f_{\lambda}^{\theta}(L-j) x^L
\end{align}

\vspace{3mm}

Out of charity, there is no salvation at all. \href{https://drive.google.com/file/d/1MvAjsOvsvSIq34uAFz_h16-yVIR3_MvM/view?fbclid=IwAR3cyh3IOcsgGE7M-1TIxh3ESL6IHZ_nAmcF8IGdtnWDZHa1o6mhybQ5eJU}{\underline{With charity}}, there is evolution.

\vspace{3mm}

Vinicius Claudino FERRAZ, 3/Octo/2019, Release $1.0.4$

\end{document}

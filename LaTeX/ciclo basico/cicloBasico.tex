\documentclass[12pt,a4paper]{article}
\usepackage{amsmath}
\usepackage{amssymb} %mathbb
\usepackage{graphicx}
\usepackage{hyperref}
\usepackage[top=1.0cm,bottom=1.3cm,left=1.0cm,right=1.0cm]{geometry}

\begin{document}

\Large

Ciclo B\'asico

\normalsize

\vspace{6mm}

\textbf{At\'e $11$ anos}

Interpreta\c{c}\~ao de texto. As quatro opera\c{c}\~oes, tabuada, fra\c{c}\~oes, n\'umeros primos, mmc, mdc, m\'ultiplos, divisores.

\vspace{6mm}

\textbf{At\'e $12$ anos}

Teoria de conjuntos. Potencia\c{c}\~ao. As cinco opera\c{c}\~oes em $\mathbb{N}, \mathbb{Q}_+$. Vide Matem\'atica e Realidade de IEZZI, Gelson.

\vspace{6mm}

\textbf{At\'e $13$ anos}

Radicia\c{c}\~ao. As seis opera\c{c}\~oes em $\mathbb{Z}, \mathbb{Q}$. Regra de tr\^es simples e composta. Juros simples e compostos. Geometria b\'asica. \^Angulos opostos pelo v\'ertice. Paralelismo. Pol\'igonos. Per\'imetro e \'area. Introduzir \'algebra, equa\c{c}\~oes e inequa\c{c}\~oes de primeiro grau, plano cartesiano, gr\'afico de reta. Sistemas lineares em duas vari\'aveis.

\vspace{6mm}

\textbf{At\'e $14$ anos}

Mon\^omios. mmc, mdc. Polin\^omios. Produtos not\'aveis. Fatora\c{c}\~ao. Equa\c{c}\~oes param\'etricas. Toda a geometria b\'asica: Congru\^encia de tri\^angulos. Teorema do arco capaz.

\vspace{6mm}

\textbf{At\'e $15$ anos}

N\'umeros irracionais e reais. Radicais. Equa\c{c}\~oes e inequa\c{c}\~oes de segundo grau. Semelhan\c{c}a de tri\^angulos. Teorema de Pit\'agoras. $\pi$. C\'irculo, tri\^angulo inscrito, tri\^angulo circunscrito. Circunfer\^encia, quadril\'atero inscrito, quadril\'atero circunscrito. Hex\'agono inscrito, hex\'agono circunscrito. Ap\'otema. Lado e \'area de pol\'igonos regulares. Somas de \^angulos de pol\'igonos. Contar diagonais. Introdu\c{c}\~ao a rela\c{c}\~oes e fun\c{c}\~oes. Introdu\c{c}\~ao \`a trigonometria.

\vspace{6mm}

\textbf{At\'e $16$ anos}

Rela\c{c}\~oes e fun\c{c}\~oes. Reta. Fun\c{c}\~ao modular. Par\'abola. Exponencial. Logaritmo. \^Angulos trigonom\'etricos. Radianos.

\vspace{6mm}

\textbf{At\'e $17$ anos}

N\'umeros complexos. Polin\^omios. Teoremas de ra\'izes, divisores e fatora\c{c}\~ao de polin\^omios. Geometria espacial. Matrizes. Sistemas lineares. Limites e c\'alculo diferencial. Combinat\'oria. Estat\'istica.

\vspace{6mm}

\textbf{At\'e $18$ anos}

Geometria anal\'itica. Retas verticais. Foco e diretriz da par\'abola, hip\'erbole e elipse. C\'irculo. C\'alculo integral. Progress\~ao aritm\'etica, progress\~ao geom\'etrica.

\vspace{6mm}

Fora da caridade n\~ao h\'a salva\c{c}\~ao. Com caridade, h\'a evolu\c{c}\~ao.

\vspace{6mm}

Vers\~ao 0.2 de 19/jan/2020 por Vinicius Claudino Ferraz.

\end{document}

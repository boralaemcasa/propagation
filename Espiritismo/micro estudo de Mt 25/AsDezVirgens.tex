\documentclass{rbfin}
\usepackage{amsmath}
\usepackage{amssymb} %mathbb
\usepackage{gensymb} % \degree
\usepackage{multicol}
\usepackage{graphicx}
\usepackage{hyperref}
\usepackage{cancel}
\newcolumntype{C}{>{$}c<{$}}
\newcommand\doubleoverline[1]{\overline{\overline{#1}}}
\newcommand\doubleunderline[1]{\underline{\underline{#1}}}
\begin{document}
\selectlanguage{brazil}
\shorttitle{$\,$} % appears on header every other page
\rbfe{}
\autor{$\,$}

$\,$

\vspace{40mm}

\Huge

\begin{center}
\textbf{Mas ele respondeu:}

\textbf{A verdade é que não as conheço!}

\textbf{Mateus 25, 12}

\end{center}

\vspace{90mm}

\Large

\begin{flushright}
\textbf{Vinícius Claudino Ferraz}, versão de 26/maio/2023.
\end{flushright}

\LARGE

\newpage

\doublespacing

\begin{multicols}{2}
Oração 1: Mas (conjunção) ele (sujeito) respondeu (algo a alguém --- verbo transitivo direto e indireto).

Elas queriam entrar, mas... --- oração adversativa. Ele contrariou a vontade delas, da mesma forma que no versículo 9, as prudentes estavam se lixando para as insensatas.

Ação: Respondeu --- Diálogo entre Jesus e as mulheres, Jesus responde às preces da humanidade.

Quem respondeu? Ele --- o noivo, em um casamento da humanidade com Jesus.

A quem respondeu? Às 5 virgens insensatas.

Quando respondeu? Depois que compraram óleo, depois da meia-noite, tarde demais! 

Elas até ``compraram a luz'', mas estavam atrasadas.

Onde respondeu? À porta da casa do noivo. No sereno da noite. A noite escura das almas. (Joanna de Ângelis)

Como respondeu? Oração seguinte. Respondeu com Justiça, dizendo que não conhece quem está na escuridão.

Por que respondeu? Porque a humanidade pedia para Jesus abrir a porta para todos.

Vejo os católicos rezarem: levai todas as almas para o céu.

Mas nem todos os que dizem senhor, senhor, serão salvos. E.S.E. capítulo 18, item 9.

Faz-se necessário o mérito de cada uma.

\dotfill

\newpage

Oração 2: A verdade (sujeito) é (predicativo do sujeito) que não as conheço!

Ação: Ser. Essência. A filosofia quer saber: o que é tudo, afinal de contas?

Quando é? Sempre. Não é estar. É permanente.

Quem é? A Verdade. Eu sou o caminho da verdade e da vida. João 14, 6.

O que é a Verdade? Oração seguinte. Jesus só \textbf{REconhece} a luz.

Como é? Justamente.

Onde? A Verdade é universal, em todo lugar, onisciente.

Por que a Verdade é o que é? Lei Natural, que dá origem a todas as outras leis. É uma revelação dos desígnios criados por deus e que se aplica a todos, anjos, humanos, animais, vegetais, minerais: tudo é luz. Vós sois a luz do mundo. Brilhe a vossa luz. Sermão da montanha.

\dotfill

\newpage

Oração 3: Não (advérbio de negação) as (objeto) conheço (verbo transitivo direto)!

Ação: Conhecer. Saber quem é. Deus não identifica uma falsa luz. A luz é como uma impressão digital, ou dos olhos, da íris.

Quem desconhece? O noivo.

A quem desconhece? Às insensatas.

Quando desconhece? No presente do indicativo. Cedo demais. Deus nos reconhecerá na hora certa, em que estivermos preparados.

Como desconhece? Justamente.

Onde conhece? À porta de sua própria casa.

Por que desconhece? Porque a lei não permite que se entre em Sua casa sem a veste nupcial. Deus é o primeiro a respeitar Suas próprias leis.

Exclamação = ênfase.

\dotfill

\newpage

Parábola do festim de bodas

ESE cap 18, item 1. Falando ainda por parábolas, disse-lhes Jesus: O reino dos céus se assemelha a um rei que, querendo festejar as \textbf{bodas} de seu filho — despachou seus servos a chamar para as bodas os que tinham sido convidados; estes, porém, recusaram ir. — O rei despachou outros servos com ordem de dizer da sua parte aos convidados: Preparei o meu jantar; mandei matar os meus bois e todos os meus cevados; tudo está pronto; vinde às bodas. — Eles, porém, sem se incomodarem com isso, lá se foram, um para a sua casa de campo, outro para o seu negócio.

— Os outros pegaram dos servos e os mataram, depois de lhes haverem feito muitos ultrajes. — Sabendo disso, o rei se tomou de cólera e, mandando contra eles seus exércitos, exterminou os assassinos e lhes queimou a cidade.

Então, disse a seus servos: O festim das bodas está inteiramente preparado; mas, os que para ele foram chamados \textbf{não eram dignos dele}. Ide, pois, às encruzilhadas e chamai para as bodas todos quantos encontrardes. — Os servos então saíram pelas ruas e trouxeram todos os que iam encontrando, bons e maus; a sala das bodas se encheu de pessoas que se puseram à mesa.

\newpage

Entrou, em seguida, o rei para ver os que estavam à mesa, e, dando com um homem que não vestia a túnica nupcial — disse-lhe: Meu amigo, como entraste aqui sem a túnica nupcial? O homem guardou silêncio. — Então, disse o rei à sua gente: Atai-lhe as mãos e os pés e lançai-o nas trevas exteriores: aí é que haverá prantos e ranger de dentes, — porquanto muitos há chamados, mas \textbf{poucos escolhidos}. (S. Mateus, 22:1 a 14.)

2. O incrédulo sorri a esta parábola, que lhe parece de pueril ingenuidade, por não compreender que se possa opor tanta dificuldade para assistir a um festim e, ainda menos, que convidados levem a resistência a ponto de massacrarem os enviados do dono da casa. “As parábolas”, diz ele, o incrédulo, “são, sem dúvida, imagens; mas, ainda assim, mister se torna que não ultrapassem os limites do verossímil”.

Fora, contudo, injusto acusar-se o povo inteiro de tal estado de coisas. A responsabilidade tocava principalmente aos fariseus e saduceus, que sacrificaram a nação por efeito do orgulho e do fanatismo de uns e pela incredulidade dos outros. São, pois, eles, sobretudo, que Jesus identifica nos convidados que recusam comparecer ao festim das bodas. Depois, acrescenta: “Vendo isso, o Senhor mandou convidar a todos os que fossem encontrados nas encruzilhadas, bons e maus.” 

\newpage

Queria dizer desse modo que a palavra ia ser pregada a todos os outros povos, pagãos e idólatras, e estes, acolhendo-a, seriam admitidos ao festim, em lugar dos primeiros convidados.

Mas não basta a ninguém ser convidado; não basta dizer-se cristão, nem sentar-se à mesa para tomar parte no banquete celestial. É preciso, antes de tudo e sob condição expressa, estar revestido da túnica nupcial, isto é, ter puro o coração e cumprir a lei segundo o espírito. Ora, a lei toda se contém nestas palavras: Fora da caridade não há salvação. Entre todos, porém, que ouvem a palavra divina, quão poucos são os que a guardam e a aplicam proveitosamente! Quão poucos se tornam dignos de entrar no reino dos céus! Eis por que disse Jesus: Chamados haverá muitos; poucos, no entanto, serão os escolhidos.

\dotfill

ESE cap 18, item 6. Nem todos os que me dizem: Senhor! Senhor! entrarão no reino dos céus; apenas entrará aquele que faz a vontade de meu Pai, que está nos céus. — Muitos, nesse dia, me dirão: Senhor! Senhor! não profetizamos em teu nome? Não expulsamos em teu nome o demônio? Não fizemos muitos milagres em teu nome? — Eu então lhes direi em altas vozes: Afastai-vos de mim, \textbf{vós que fazeis obras de iniquidade}. (S. Mateus, 7:21 a 23.)

\newpage

9. Em vão dirão eles a Jesus: “Senhor! não profetizamos, isto é, não ensinamos em teu nome; não expulsamos em teu nome os demônios; não comemos e bebemos contigo?” Ele lhes responderá: “Não sei quem sois; afastai-vos de mim, vós que cometeis iniquidades, vós que desmentis com os atos o que dizeis com os lábios, que caluniais o vosso próximo, que espoliais as viúvas e cometeis adultério. Afastai-vos de mim, vós cujo coração destila ódio e fel, que derramais o sangue dos vossos irmãos em meu nome, que fazeis corram lágrimas, em vez de secá-las. Para vós, haverá prantos e ranger de dentes, porquanto o reino de Deus é para os que são brandos, humildes e caridosos. Não espereis dobrar a justiça do Senhor pela multiplicidade das vossas palavras e das vossas genuflexões. O caminho único que vos está aberto, para achardes graça perante ele, é o da prática sincera da lei de amor e de caridade.”

\dotfill

Caminho, Verdade e Vida, cap. 63

QUEM SOIS?

       “Mas o espírito maligno lhes respondeu: Conheço a Jesus e bem sei quem é Paulo; mas vós, quem sois?” — (ATOS, capítulo 19, versículo 15.)

\newpage

Qualquer expressão de comércio tem sua base no poder aquisitivo. Para obter, é preciso possuir.
No intercâmbio dos dois mundos, terrestre e espiritual, o fenômeno obedece ao mesmo princípio.
Nas operações comerciais de César, requerem-se moedas ou expressões fiduciárias com efígies e identificações que lhes digam respeito. Nas operações de permuta espiritual requisitam-se valores individualíssimos, com os sinais do Cristo.
O dinheiro de Jesus é o amor. Sem ele, não é lícito aventurar-se alguém ao sagrado comércio das almas.
O versículo aqui nomeado constitui benéfica advertência a quantos, para o esclarecimento dos outros, invocam o Mestre, sem títulos vivos de sua escola sacrificial.
Mormente no que se refere às relações com o plano invisível, mantendo cuidado por evitar afirmativas a esmo.
Não vos aventureis ao movimento, sem o poder aquisitivo do amor de Jesus.
O Mestre é igualmente conhecido de seus infelizes adversários. Os discípulos sinceros do Senhor são observados por eles também. Os inimigos da luz reconhecem-lhes o sublime valor.
Quando vos dispuserdes, portanto, a esse gênero de trabalho, não olvideis vossa própria identificação, porque, provavelmente, sereis interpelados pelos representantes do mal, que vos perguntarão quem sois.

\dotfill

\newpage

Pastorino, sabedoria do evangelho, vol. 7, p. 99

O parabolista mostra a chegada extemporânea e apressada das cinco “tolas", com a resposta aparentemente dura do noivo: "não vos conheço". Não nos esqueçamos de que o verbo "conhecer" tem
sentido especial na Escritura, exprimindo a união sexual (vol. 1), a união íntima (volume 4), ou a
gnose total (vol. 5); daí dizer Jesus que “conhece o Pai" (vol. 4).
Finaliza a parábola com a repetição da advertência: "Despertai, porque não sabeis o dia nem a
hora". A consciência, mantida alerta e acordada no plano superior do espírito, constitui a lâmpada
que deve estar alimentada pelo azeite da prece ininterrupta. Assim não se desfaz a sintoma com Seu
Espírito, conservando-o preparado para a recepção suprema ao Noivo Divino.

\dotfill

Mateus 6, 9-13. O pão sobressubstancial nos dai hoje.

Obrigação do trabalho

Coragem e forças para obedecer

Ociosidade, supérfluo $\rightarrow$ não ajuda

Culpa. Própria incúria, imprevidência, ambição, descontentamento

Prudência, previdência e moderação para não perder o fruto

Não possamos trabalhar $\rightarrow$ divina providência

Provações $\rightarrow$ aceitamos como justa expiação

Inveja $\rightarrow$ perdoar o esquecimento da lei de amor/caridade

\newpage

Negar a justiça $\rightarrow$ afastar. Eterno será o júbilo daquele que sofre resignado

Quarta petição: “dá-nos hoje o pão sobressubstancial” (Orígenes)

Intelecto, conhecimento da Espiritualidade, sabedoria

Contato com a eternidade, com o eu superior, com o Infinito

Mergulhar na consciência cósmica

Visões intuitivas sobre a Verdade = Deus

\textbf{Esponsalício} Místico = Unificação

Sair das trevas para a luz

Da divisão para a união

Do egoísmo para o amor total

\dotfill

Fonte viva — 15 --- Fraternidade

“Nisto todos \textbf{conhecerão} que sois meus discípulos: se vos amardes uns aos outros.” — Jesus. (JOÃO, 13.35)

Desde a vitória de Constantino, que descerrou ao mundo cristão as portas da hegemonia política, temos ensaiado diversas experiências para demonstrar na Terra a nossa condição de discípulos de Jesus.

Organizamos concílios célebres, formulando atrevidas conclusões acerca da natureza de Deus e da Alma, do Universo e da Vida.

\newpage

Incentivamos guerras arrasadoras que implantaram a miséria e o terror naqueles que não podiam crer pelo diapasão da nossa fé.

Disputamos o sepulcro do Divino Mestre, brandindo a espada mortífera e ateando o fogo devorador.

Criamos comendas e cargos religiosos, distribuindo o veneno e manejando o punhal.

Acendemos fogueiras e erigimos cadafalsos, inventamos suplícios e construímos prisões para quantos discordassem dos nossos pontos de vista.

Estimulamos insurreições que operaram o embate de irmãos contra irmãos, em nome do Senhor que testemunhou na cruz o devotamento à Humanidade inteira.

Edificamos palácios e basílicas, famosos pela suntuosidade e beleza, pretendendo reverenciar-lhe a memória, esquecidos de que ele, em verdade, não possuía uma pedra onde repousar a cabeça.

E, ainda hoje, alimentamos a separação e a discórdia, erguendo trincheiras de incompreensão e animosidade, uns contra os outros, nos variados setores da interpretação.

Entretanto, a palavra do Cristo é insofismável. Não nos faremos titulares da Boa Nova simplesmente através das atitudes exteriores…

Precisamos, sim, da cultura que aprimora a inteligência, da justiça que sustenta a ordem, do progresso material que enriquece o trabalho e de assembleias que favoreçam o estudo; no entanto, toda a movimentação humana, sem a luz do amor, pode perder-se nas sombras…

Seremos admitidos ao aprendizado do Evangelho, cultivando o Reino de Deus que começa na vida íntima.

Estendamos, assim, a fraternidade pura e simples, amparando-nos mutuamente… Fraternidade que trabalha e ajuda, compreende e perdoa, entre a humildade e o serviço que asseguram a vitória do bem. 14 Atendamo-la, onde estivermos, recordando a palavra do Senhor que afirmou com clareza e segurança: — “Nisto todos conhecerão que sois meus discípulos: se vos amardes uns aos outros.”

\dotfill

\textbf{Verdade e Justiça:}

Vinha de luz — 162 --- A luz inextinguível

“A caridade jamais se acaba.” — Paulo. (1 CORÍNTIOS, 13.8)

Permaneces no campo da experiência humana, em plena atividade transformadora.

Todas as situações de que te envaideces, comumente, são apenas ângulos necessários mas instáveis de tua luta.

A fortuna material, se não a fundamentas no trabalho edificante e contínuo, é patrimônio inseguro.

A família humana, sem laços de verdadeira afinidade espiritual, é ajuntamento de almas, em experimentação de fraternidade, da qual te afastarás, um dia, com extremas desilusões.

\newpage

A eminência diretiva, quando não solidificada em alicerces robustos de justiça e sabedoria, de trabalho e consagração ao bem, é antecâmara do desencanto.

A posição social é sempre um jogo transitório.

As emoções da Esfera física, em sua maior parte, apagam-se como a chama duma vela.

A mocidade do corpo denso é floração passageira.

A fama e a popularidade costumam ser processos de tortura incessante.

A tranquilidade mentirosa é introdução a tormentos morais.

A festa desequilibrante é véspera de laborioso reparo.

O abuso de qualquer natureza compele ao reajustamento apressado.

Tudo, ao redor de teus passos, na vida exterior, é obscuro e problemático.

O amor, porém, é a luz inextinguível.

A caridade jamais se acaba.

\textbf{O bem que praticares, em algum lugar, é teu advogado em toda parte.}

Através do amor que nos eleva, o mundo se aprimora. Ama, pois, em Cristo, e alcançarás a glória eterna. $\blacksquare$
\end{multicols}

\newpage

$\,$

\vspace{75mm}

\begin{center}
\textbf{OBRIGADO.}
\end{center}

\end{document}

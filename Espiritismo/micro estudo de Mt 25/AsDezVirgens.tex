\documentclass{rbfin}
\usepackage{amsmath}
\usepackage{amssymb} %mathbb
\usepackage{gensymb} % \degree
\usepackage{multicol}
\usepackage{graphicx}
\usepackage{hyperref}
\usepackage{cancel}
\newcolumntype{C}{>{$}c<{$}}
\newcommand\doubleoverline[1]{\overline{\overline{#1}}}
\newcommand\doubleunderline[1]{\underline{\underline{#1}}}
\begin{document}
\selectlanguage{brazil}
\shorttitle{$\,$} % appears on header every other page
\rbfe{}
\autor{$\,$}

$\,$

\vspace{40mm}

\Huge

\begin{center}
\textbf{Mas ele respondeu:}

\textbf{A verdade é que não as conheço!}

\textbf{Mateus 25, 12}

\end{center}

\vspace{90mm}

\Large

\begin{flushright}
\textbf{Vinícius Claudino Ferraz}, versão de 9/maio/2023.
\end{flushright}

\LARGE

\newpage

\doublespacing

Oração 1: Mas (conjunção) ele (sujeito) respondeu (algo a alguém --- verbo transitivo direto e indireto).

Elas queriam entrar, mas... --- oração adversativa. Ele contrariou a vontade delas, da mesma forma que no versículo 9, as prudentes estavam se lixando para as insensatas.

O que? Respondeu --- Diálogo entre deus e as mulheres, Deus responde às preces da humanidade.

Quem respondeu? Ele --- o noivo, em um casamento da humanidade com Deus.

Quando respondeu? Depois que compraram óleo, depois da meia-noite, tarde demais! 

Elas até ``compraram a luz'', mas estavam atrasadas.

Onde respondeu? À porta da casa do noivo. No sereno da noite. A noite escura das almas. (Joanna de Ângelis)

Como respondeu? Oração seguinte. Respondeu com Justiça, dizendo que não conhece quem está na escuridão.

Por que respondeu? Porque a humanidade pedia para Deus abrir a porta para todos.

Vejo os católicos rezarem: levai todas as almas para o céu.

Mas nem todos os que dizem senhor, senhor, serão salvos. E.S.E. capítulo 18, item 9.

\newpage

Oração 2: A verdade (sujeito) é (predicativo do sujeito) que não as conheço!

O que? Ser. Essência. A filosofia quer saber: o que é tudo, afinal de contas?

Quando é? Sempre. Não é estar. É permanente.

Quem é? A Verdade. Eu sou o caminho da verdade e da vida. João 14, 6.

O que é a Verdade? Oração seguinte. Deus só \textbf{REconhece} a luz.

Como é? Justamente.

Onde? A Verdade é universal, em todo lugar, onisciente.

Por que a Verdade é o que é? Lei Natural, que dá origem a todas as outras leis. É uma revelação dos desígnios criados por deus e que se aplica a todos, anjos, humanos, animais, vegetais, minerais: tudo é luz. Vós sois a luz do mundo. Brilhe a vossa luz. Sermão da montanha.

\newpage

Oração 3: Não (advérbio de negação) as (objeto) conheço (verbo transitivo direto).

O que? Conhecer. Saber quem é. Deus não identifica uma falsa luz. A luz é como uma impressão digital, ou dos olhos, da íris.

Quem desconhece? O noivo.

A quem desconhece? Às insensatas.

Quando desconhece? No presente do indicativo. Cedo demais. Deus nos reconhecerá na hora certa, em que estivermos preparados.

Como desconhece? Justamente.

Onde conhece? À porta de sua própria casa.

Por que desconhece? Porque a lei não permite que se entre em Sua casa sem a veste nupcial. Deus é o primeiro a respeitar Suas próprias leis.

\newpage

reconhecer

exclamação

Justiça

Pastorino.

\end{document}

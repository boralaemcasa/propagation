\documentclass{rbfin}
\usepackage{amsmath}
\usepackage{amssymb} %mathbb
\usepackage{gensymb} % \degree
\usepackage{multicol}
\usepackage{graphicx}
\usepackage{hyperref}
\usepackage{cancel}
\newcolumntype{C}{>{$}c<{$}}
\newcommand\doubleoverline[1]{\overline{\overline{#1}}}
\newcommand\doubleunderline[1]{\underline{\underline{#1}}}
\begin{document}
\selectlanguage{brazil}
\shorttitle{$\,$} % appears on header every other page
\rbfe{}
\autor{$\,$}

$\,$

\vspace{40mm}

\Huge

\begin{center}
\textbf{Portanto, vigiai,}

\textbf{porque não sabeis o dia nem a hora.}

\textbf{Mateus 25, 13}

\end{center}

\vspace{90mm}

\Large

\begin{flushright}
\textbf{Vinícius Claudino Ferraz}, versão de 02/junho/2023.
\end{flushright}

\LARGE

\newpage

\doublespacing

\begin{multicols}{2}
Oração 1: Portanto (oração conclusiva), vigiai.

Ação: vigiai vós. Imperativo.

Quem deve vigiar? Quem tem ouvidos de ouvir.

Vigiar o que? No sentido intransitivo, equivale a tende cuidado! Com vocês mesmos!

Quando vigiar? Agora.

Por que vigiar? Oração seguinte.

\dotfill

Oração 2: Porque (oração explicativa) não sabeis (adverbo de negação $\Rightarrow$ ignorais) o dia nem a hora (2 objetos diretos).

Ação: Saber. Dizia Jesus que nem ele sabia. Só o Pai sabe. Revelação. Com margem de erro.

Quem ignora? Vós, os interlocutores. Nós.

Ignoramos o que? A hora, o dia, o mês, o ano, o século, o milênio.

Não sabemos nem o que, nem o como, nem o onde. É na Terra, mas para onde? 

O reino dos céus vem como um ladrão. O ladrão sabe o que faz?

``Pai, perdoai-lhes, eles não sabem o que fazem.'' 

Quando ignoramos? No presente do indicativo.

Por que ignoramos? Apocalipse. Por que é vedado o acesso ao futuro? Procurar no {\color{blue}Livro dos Espíritos}.

\dotfill

\newpage

Vigilância (de acordo com a bíblia do caminho):

$\rightarrow$ para não ser mundano, e sim em consonância com o espiritual;

$\rightarrow$ para não perder a hora (a ovelha perdida se encontrará);

$\rightarrow$ para não dormir no ponto, e sim permanecermos despertos e preparados;

$\rightarrow$ quando despertarmos, não estarmos sem luz;

$\rightarrow$ Não temas, porque Deus vigia.

$\rightarrow$ Cada um deve velar por si.

$\rightarrow$ Para não cair em tentação. De pensamentos negativos, por exemplo.

$\rightarrow$ Apóstolos invigilantes no Getsêmani e na sexta da Paixão. Mateus 26, 40. Nem uma hora pudeste velar comigo?

$\rightarrow$ O fracasso, depois o choro de Pedro. Cair em si.

$\rightarrow$ Nós vamos dar de cara com Jesus uma hora dessas.

$\rightarrow$ Vigiar a boca. As nascentes do coração.

\dotfill

\newpage

Palavras de vida eterna — Emmanuel --- 3

Evitando a tentação

“Vigiai e orai para não entrardes em tentação.” — JESUS (Marcos, 14.38)

Vigiar não quer dizer apenas guardar. Significa também precaver-se e cuidar. E quem diz cuidar afirma igualmente trabalhar e defender-se.

Orar, a seu turno, não exprime somente adorar e aquietar-se, mas, acima de tudo, comungar com o Poder Divino, que é crescimento incessante para a luz, e com o Divino Amor, que é serviço infatigável no bem.

Tudo o que repousa em excesso é relegado pela Natureza à inutilidade.

O tesouro escondido transforma-se em cadeia de usura. A água estagnada cria larvas de insetos patogênicos.

Não te admitas na atitude de vigilância e oração, fugindo à luta com que a Terra te desafia.

Inteligência parada e mãos paradas impõem paralisia ao coração que, da inércia, cai na cegueira.

Vibra com a vida que estua, sublime, ao redor de ti, e trabalha infatigavelmente, dilatando as fronteiras do bem, aprendendo e ajudando aos outros em teu próprio favor. Essa é a mais alta fórmula de vigiar e orar para não cairmos em tentação.

\dotfill

\newpage

Mais Luz --- Batuíra --- 13

Oração e vigilância

1 Disse-nos Jesus: “Orai e vigiai.” ( † )

Sim, oremos e vigiemos.


2 Orai amando.

Vigiai servindo.


3 Orai compreendendo.

Vigiai auxiliando.


4 Orai confiando.

Vigiai esclarecendo.


5 Orai refletindo.

Vigiai realizando.


6 Orai abençoando.

Vigiai construindo.


7 Orai esperando.

Vigiai aprendendo.


8 Orai ouvindo.

Vigiai semeando.


9 Orai, pacíficos.

Vigiai, operosos.


10 Orai, tranquilos.

Vigiai, seguros.


11 Abraçando a oração e a vigilância, dignifiquemos a nossa edificação espírita-cristã, ofertando-lhe o melhor de nossas vidas. 

\newpage

12 E integrados nessas duas forças da alma, sem as quais se nos fará impraticável o aprimoramento íntimo, para atender aos desígnios do Eterno, permaneçamos, cada dia e cada hora, no refúgio da fé renovadora que nos enobrece a esperança, com a felicidade de trabalhar e com o privilégio de servir.

\dotfill

Palavras de vida eterna — Emmanuel --- 20

Vigiando

“… Se alguma virtude há e se algum louvor existe, seja isso o que ocupe o vosso pensamento.” — PAULO (Filipenses, 4.8)

1 Trabalhemos vigiando.

2 Aquilo que nos ocupa o pensamento é a substância de que se nos constituirá a própria vida.

3 Retiremos, dessa forma, o coração de tudo o que não seja material de edificação do Reino Divino, em nós próprios.

4 Em verdade, muita sugestão criminosa buscará enevoar-nos a mente, muito lodo da estrada procurar-nos-á as mãos na jornada de cada dia e muito detrito do mundo tentará imobilizar-nos os pés.

5 É a nuvem da incompreensão conturbando-nos o ambiente doméstico…

6 É a injúria nascida na palavra inconsciente dos desafetos gratuitos…

\newpage

7 É a acusação indébita de permeio com a calúnia destruidora…

8 É a maledicência convidando-nos à mentira e à leviandade…

9 É o amigo de ontem que se rende às requisições da treva, passando à condição de censor das nossas qualidades ainda em processo de melhoria…

10 Entretanto, à frente de todos os percalços, não te prendas às teias da perturbação e da sombra.

11 Em todas as situações e em todos os assuntos, guardemos a alma nos ângulos em que algo surja digno de louvor, fixando o bem e procurando realizá-lo com todas as energias ao nosso alcance.

12 Aos mais infelizes, mais amparo.

Aos mais doentes, mais socorro.

13 E, ocupando o nosso pensamento com os valores autênticos da vida, aprenderemos a sorrir para as dificuldades, quaisquer que sejam, construindo gradativamente, em nós mesmos, o templo vivo da luz para a comunhão constante com o nosso Mestre e Senhor. $\blacksquare$
\end{multicols}

\newpage

$\,$

\vspace{75mm}

\begin{center}
\textbf{OBRIGADO.}
\end{center}

\end{document}
